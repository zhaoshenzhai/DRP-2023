\documentclass{beamer}

\usepackage[T1]{fontenc}                             % make fonts work? not sure
\usepackage[hidelinks]{hyperref}                     % links
\usepackage[side]{footmisc}                          % footnotes
\usepackage{geometry}                                % page layout
\usepackage{fancyhdr}                                % headers and footers
\usepackage{titlesec}                                % title page
\usepackage{tocloft}                                 % custom table of contents
\usepackage{amsfonts, amsmath, amssymb, amsthm}      % basic maths commands
\usepackage{mathtools}                               % more maths commands
\usepackage{mathrsfs}                                % more math commands
\usepackage{mdframed}                                % framed environments
\usepackage{graphicx}                                % images and other graphics
\usepackage{tikz}                                    % maths figures
\usepackage{tikz-3dplot}                             % needed by tikz
\usepackage{tikzpagenodes}                           % needed by tikz
\usepackage{pgfplots}\pgfplotsset{compat=1.7}        % plots
\usepackage{contour}                                 % underlining
\usepackage{ulem}                                    % needed by contour
\usepackage{xcolor}                                  % more colors
\usepackage{caption}                                 % captions outside float
\usepackage{enumitem}                                % enumerate and itemize indents
\usepackage{nicematrix}                              % matrices and tables
\usepackage{bm}                                      % bold math
\usepackage{esint}                                   % integrals
\usepackage{marginnote}                              % margin notes

\usetikzlibrary{matrix}
\usetikzlibrary{positioning}
\usetikzlibrary{patterns}
\usetikzlibrary{decorations.markings}
\usetikzlibrary{arrows}
\usetikzlibrary{arrows.meta}
\usetikzlibrary{backgrounds}
\usetikzlibrary{math}
\usetikzlibrary{cd}

\titleformat{\chapter}[display]
{\normalfont\huge\bfseries}{\chaptertitlename\ \thechapter}{20pt}{\Huge}   
\titlespacing*{\chapter}{0pt}{-20pt}{40pt}
\synctex=1
\setlength{\cftsecindent}{10pt}
\makeatletter\newcommand*{\toccontents}{\@starttoc{toc}}\makeatother

\definecolor{gray}{rgb}{0.8, 0.8, 0.8}
\definecolor{lightGray}{rgb}{0.9, 0.9, 0.9}
\mdfdefinestyle{defstyle}{backgroundcolor = lightGray, innertopmargin = -2pt, linewidth = 1pt}
\newmdtheoremenv[style = defstyle]{theorem}{Theorem}[chapter]
\newmdtheoremenv[style = defstyle]{axiom}[theorem]{Axiom}
\newmdtheoremenv[style = defstyle]{definition}[theorem]{Definition}
\newmdtheoremenv[style = defstyle]{defprop}[theorem]{Definition/Proposition}
\newmdtheoremenv[style = defstyle]{proposition}[theorem]{Proposition}
\newmdtheoremenv[style = defstyle]{lemmath}[theorem]{Lemma}
\newmdtheoremenv[style = defstyle]{corollary}{Corollary}[theorem]
\theoremstyle{definition}\newmdtheoremenv[style = defstyle]{exercise}{Exercise}
\newmdtheoremenv[style = defstyle]{lemmaex}{Lemma}[exercise]
\theoremstyle{definition}\newtheorem{example}[theorem]{Example}
\theoremstyle{definition}\newtheorem*{remark}{Remark}
\theoremstyle{remark}\newtheorem*{solution}{Solution}
\makeatletter\renewenvironment{proof}[1][\proofname]{\par\normalfont\topsep6\p@\@plus6\p@\relax\trivlist\item[\hskip\labelsep\itshape#1\@addpunct{.}]\ignorespaces}{\endtrivlist\@endpefalse}\renewcommand\qedhere{}\makeatother

% Math notations
    % Set Theory
        \newcommand{\fa}{\forall}
        \newcommand{\ex}{\exists}
        \newcommand{\tpl}[1]{\l(#1\r)}
        \newcommand{\pow}{\mathcal{P}}
        \newcommand{\id}{\operatorname{id}}
        \newcommand{\iv}{\operatorname{iv}}
        \newcommand{\dom}{\operatorname{dom}}
        \newcommand{\ran}{\operatorname{ran}}
        \newcommand{\cdm}{\operatorname{cdm}}
        \newcommand{\fld}{\operatorname{fld}}
        \newcommand{\im}{\operatorname{im}}
        \newcommand{\preim}{\operatorname{preim}}
        \newcommand{\rest}{\!\upharpoonright\!}
        \newcommand{\comp}{\setminus}
        \newcommand{\rfcl}{\operatorname{rfcl}}
        \newcommand{\ordinal}{\textrm{ON}}
        \newcommand{\iso}{\cong}
        \newcommand{\into}{\hookrightarrow}
        \newcommand{\onto}{\twoheadrightarrow}
        \newcommand{\simto}{\rightleftarrows}
        \newcommand{\eqnum}{\approx}
        \renewcommand{\em}{\varnothing}
        \renewcommand{\mid}{\,\middle\vert\,}
        \renewcommand{\min}{\operatorname{min}}
        \renewcommand{\max}{\operatorname{max}}

    % Category Theory
        \newcommand{\cat}[1]{\textbf{#1}}
        \newcommand{\catset}{\cat{Set}}
        \newcommand{\catrel}{\cat{Rel}}
        \newcommand{\catgrp}{\cat{Grp}}
        \newcommand{\catvect}[1][K]{\cat{Vect}_#1}
        \newcommand{\Id}{\operatorname{Id}}
        \newcommand{\natid}{\mc{I}\textrm{d}}
        \newcommand{\obj}{\operatorname{Obj}}
        \newcommand{\aut}{\operatorname{Aut}}
        \newcommand{\edm}{\operatorname{End}}
        \renewcommand{\hom}{\operatorname{Hom}}

    % Geometry
        % Analysis
            \newcommand{\diam}{\operatorname{diam}}
            \newcommand{\del}{\partial}
            \renewcommand{\d}{\mathrm{d}}
            \renewcommand{\Re}{\operatorname{Re}}
            \renewcommand{\Im}{\operatorname{Im}}

        % Topology
            \newcommand{\sttopR}[1][]{\mc{T}^\textrm{st}_{\R^{#1}}}
            \newcommand{\ULtopR}{\mc{T}^\textrm{UL}_\R}
            \newcommand{\LLtopR}{\mc{T}^\textrm{LL}_\R}
            \newcommand{\KtopR}{\mc{T}^K_\R}

        % Complex Analysis
            \newcommand{\RS}{\hat{\C}}
            \newcommand{\mult}{\operatorname{mult}}
            \newcommand{\ord}{\operatorname{ord}}

    % Algebra
        % Linear Algebra
            % Core
                \newcommand{\rank}{\operatorname{rank}}
                \newcommand{\nullity}{\operatorname{nullity}}
                \newcommand{\bilform}[2]{\l\langle{#1},{#2}\r\rangle}
                \newcommand{\proj}{\operatorname{proj}}
                \renewcommand{\span}{\operatorname{span}}
            % Classes of Matrices
                \newcommand{\mat}[2]{\mathcal{M}_{#1}\!\l(#2\r)}
                \newcommand{\GL}{\operatorname{GL}}
                \newcommand{\SL}{\operatorname{SL}}
                \newcommand{\Diag}{\operatorname{Diag}}
                \newcommand{\Skew}{\operatorname{Skew}}
            % Matrix
                \newcommand{\diag}{\operatorname{diag}}
                \newcommand{\trans}{\mathsf{T}}
                \newcommand{\rref}{\operatorname{rref}}
                \newcommand{\row}{\operatorname{row}}
                \newcommand{\col}{\operatorname{col}}
                \newcommand{\nullsp}{\operatorname{null}}
            % Vector
                \newcommand{\dotpro}{\boldsymbol\cdot}
                \newcommand{\cropro}{\boldsymbol\times}
                \newcommand{\unitv}[1]{\mathbf{\hat{#1}}}
                \newcommand{\unitvalt}[1]{\bm{\hat{#1}}}
                \newcommand{\ihat}{\bm\hat{\textbf{\i}}}
                \newcommand{\jhat}{\bm\hat{\textbf{\j}}}
                \newcommand{\valt}[1]{\bm{#1}}
                \renewcommand{\v}[1]{\mathbf{#1}}

        % Group Theory
            \newcommand{\Sym}{\operatorname{Sym}}
            \newcommand{\nsubgrpeq}{\trianglelefteq}
            \newcommand{\cyclic}[1]{\l\langle#1\r\rangle}

        % Number Theory
            \newcommand{\lcm}{\operatorname{lcm}}
            \newcommand{\divides}{\,|\,}
            \newcommand{\ndivides}{\nmid}
            \renewcommand{\mod}[1]{\equiv_{#1}}

        % Ring Theory
            \newcommand{\charac}{\operatorname{char}}
            \newcommand{\eval}{\operatorname{eval}}

    % Others
        % Projective Geometry
            \newcommand{\proje}[1]{\l[#1\r]}

% Math others
    % Number Systems
        \newcommand{\N}{\mathbb{N}}
        \newcommand{\Z}{\mathbb{Z}}
        \newcommand{\Q}{\mathbb{Q}}
        \newcommand{\R}{\mathbb{R}}
        \newcommand{\C}{\mathbb{C}}
        \newcommand{\F}{\mathbb{F}}
        \renewcommand{\P}{\mathbb{P}}

% LaTeX/MathJax
    % Fonts
        \newcommand{\mc}[1]{\mathcal{#1}}
        \newcommand{\ms}[1]{\mathscr{#1}}
        \newcommand{\mb}[1]{\mathbb{#1}}
        \newcommand{\mf}[1]{\mathfrak{#1}}
        \renewcommand{\it}[1]{\textit{#1}}
        \renewcommand{\bf}[1]{\textbf{#1}}
        \renewcommand{\phi}{\varphi}
        \renewcommand{\epsilon}{\varepsilon}
    % Meta
        \renewcommand{\qed}{\phantom\qedhere\hfill$\blacksquare$}
        \newcommand{\qedin}{\tag*{$\blacksquare$}}
        \newcommand{\exqed}{\phantom\qedhere\hfill$\blacklozenge$}
        \newcommand{\exqedin}{\tag*{$\blacklozenge$}}
        \newcommand{\blob}{\bullet}
        \newcommand{\slot}{-}
        \newcommand{\cref}[1]{\tag{$\,#1\,$}}
        \newcommand{\ulthm}[1]{\uline{\phantom{#1}}\llap{\contour{gray}{#1}}}
        \newcommand{\uldef}[1]{\uline{\phantom{#1}}\llap{\contour{lightGray}{#1}}}
        \newcommand{\ul}[1]{\uline{\phantom{#1}}\llap{\contour{white}{#1}}}
        \newcommand{\side}[2][0in]{\marginpar{\vspace{#1 - 0.07in}\footnotesize{\normalfont{#2}}}}
        \renewcommand{\ref}[1]{\l(\,#1\,\r)}
        \renewcommand{\bar}{\overline}
        \renewcommand{\l}{\left}
        \renewcommand{\r}{\right}


\title{\normalsize{Possible Choices of Complex Co-ordinates for a Surface}}
\subtitle{\small{Moduli Spaces of Riemann Surfaces}}
\author{Zhaoshen Zhai}
\institute{McGill University}
\date{\today}

\usecolortheme{seahorse}
\usefonttheme{serif}
\useinnertheme{rounded}

\begin{document}\frame{\titlepage}
    \begin{frame}{How many complex coordinates on $\R^2$?}
        \begin{center}
            \begin{tikzpicture}
                \draw[<->, thick] (-2,0) -- (2,0);
                \draw[<->, thick] (0,-1.5) -- (0,1.5);
                \fill (-2,1.5) circle (0in) node{$\R^2$};
                \fill (1,0.75) circle (0.02in) node[above right]{$\l(x,y\r)$};

                \begin{scope}[xshift=2.25in]
                    \draw[<->, thick] (-2,0) -- (2,0);
                    \draw[<->, thick] (0,-1.5) -- (0,1.5);
                    \fill (-2,1.5) circle (0in) node{$\C$};
                    \fill (1,0.75) circle (0.02in) node[above right]{$x+iy$};
                \end{scope}

                \draw[dashed] (0.25in,2) edge[out=20,in=160,->] (2in,2);
                \fill (1.125in,2.8) circle (0in) node{$\l(x,y\r)\mapsto x+iy$};
            \end{tikzpicture}
        \end{center}
    \end{frame}
    \begin{frame}{How many complex coordinates on $\R^2$?}
        \begin{center}
            \begin{tikzpicture}
                \draw[<->, thick] (-2,0) -- (2,0);
                \draw[<->, thick] (0,-1.5) -- (0,1.5);
                \fill (-2,1.5) circle (0in) node{$\R^2$};
                \fill (1,0.75) circle (0.02in) node[above right]{$\l(x,y\r)$};

                \begin{scope}[xshift=2.25in]
                    \draw[<->, thick] (-2,0) -- (2,0);
                    \draw[<->, thick] (0,-1.5) -- (0,1.5);
                    \fill (-2,1.5) circle (0in) node{$\C$};
                    \fill (1,-0.75) circle (0.02in) node[above right]{$x-iy$};
                \end{scope}

                \draw[dashed] (0.25in,2) edge[out=20,in=160,->] (2in,2);
                \fill (1.125in,2.8) circle (0in) node{$\l(x,y\r)\mapsto x-iy$};
            \end{tikzpicture}
        \end{center}
    \end{frame}
    \begin{frame}{How many complex coordinates on $\R^2$?}
        \begin{center}
            \begin{tikzpicture}[scale=0.8]
                \draw[<->, thick] (-2,0) -- (2,0);
                \draw[<->, thick] (0,-1.5) -- (0,1.5);
                \fill (-2,1.5) circle (0in) node{$\R^2$};
                \fill (1,0.75) circle (0.02in) node[above right]{$\l(x,y\r)$};

                \begin{scope}[xshift=3in, yshift=1in]
                    \draw[<->, thick] (-2,0) -- (2,0);
                    \draw[<->, thick] (0,-1.5) -- (0,1.5);
                    \fill (-2,1.5) circle (0in) node{$\C$};
                    \fill (1,0.75) circle (0.02in) node[above right]{$x+iy$};
                \end{scope}

                \begin{scope}[xshift=3in, yshift=-1in]
                    \draw[<->, thick] (-2,0) -- (2,0);
                    \draw[<->, thick] (0,-1.5) -- (0,1.5);
                    \fill (-2,1.5) circle (0in) node{$\C$};
                    \fill (1,-0.75) circle (0.02in) node[above right]{$x-iy$};
                \end{scope}

                \draw[dashed] (0.25in,1.5) edge[out=40, in=180, ->] (2in,3) node[xshift=0.55in, yshift=0.45in, rotate=20]{\scriptsize$\l(x,y\r)\mapsto x+iy$};
                \draw[dashed] (0.25in,-1.5) edge[out=-40, in=180, ->] (2in,-3) node[xshift=0.55in, yshift=-0.45in, rotate=-20]{\scriptsize$\l(x,y\r)\mapsto x-iy$};

                \pause

                \draw[dashed] (1.25in,1.25) edge[out=340, in=20,<->] (1.25in,-1.25) node[xshift=0.8in, yshift=-0.4in]{$\l(x,y\r)\mapsto\l(x,-y\r)$};
            \end{tikzpicture}
        \end{center}
    \end{frame}
    \begin{frame}{Refined question}
        \begin{center}
            How many complex coordinates on $\R^2$?
        \end{center}
    \end{frame}
    \begin{frame}{Refined question}
        \begin{center}
            \sout{How many complex coordinates on $\R^2$?}\\
            How many complex coordinates on $\R^2$ \textit{up to biholomorphism}?
        \end{center}
    \end{frame}
    \begin{frame}{$\C$ and $\H$}
        \begin{center}
            \begin{tikzpicture}[scale=0.8]
                \draw[<->, thick] (-2,0) -- (2,0);
                \draw[<->, thick] (0,-1.5) -- (0,1.5);
                \fill (-2,1.5) circle (0in) node{$\R^2$};
                \fill (1,0.75) circle (0.02in) node[above right]{$\l(x,y\r)$};

                \begin{scope}[xshift=3in, yshift=1in]
                    \draw[<->, thick] (-2,0) -- (2,0);
                    \draw[<->, thick] (0,-1.5) -- (0,1.5);
                    \fill (-2,1.5) circle (0in) node{$\C$};
                    \fill (1,0.75) circle (0.02in) node[above right]{$x+iy$};
                \end{scope}

                \begin{scope}[xshift=3in, yshift=-1.25in]
                    \draw[<->, thick, dashed, dash pattern={on 3pt off 2pt}] (-2,0) -- (2,0);
                    \draw[->, thick] (0,0) -- (0,1.5);
                    \fill (-2,1.5) circle (0in) node{$\H$};
                    \fill (1,1) circle (0.02in) node[above right]{$x+ie^y$};
                \end{scope}

                \draw[dashed] (0.25in,1.5) edge[out=40, in=180, ->] (2in,3) node[xshift=0.55in, yshift=0.45in, rotate=20]{\scriptsize$\l(x,y\r)\mapsto x+iy$};
                \draw[dashed] (0.25in,-1.5) edge[out=-40, in=180, ->] (2in,-3) node[xshift=0.55in, yshift=-0.45in, rotate=-20]{\scriptsize$\l(x,y\r)\mapsto x+ie^y$};

                \pause

                \draw[dashed, red] (1.25in,1.25) edge[out=340, in=20,<->] (1.25in,-1.25) node[xshift=1in, yshift=-0.4in]{No biholomorphism!};
            \end{tikzpicture}
        \end{center}
    \end{frame}
    \begin{frame}{Riemann surfaces}
        \begin{center}
            What about for manifolds?\\\ \\
            \pause
            \begin{definition}
                \textit{A \ul{Riemann surface} is a connected 1-dimensional complex manifold.}
            \end{definition}
            \includegraphics[width=0.8\textwidth]{figures/manifolds.png}
        \end{center}
    \end{frame}
    \begin{frame}{Compact Riemann surfaces}
        \begin{theorem}[Classification]
            Every compact Riemann surface is classified by its genus.
        \end{theorem}
        \pause
        \begin{center}
            \includegraphics[width=0.8\textwidth]{figures/torus_handle.png}
        \end{center}
    \end{frame}
    \begin{frame}{Topological torus}
        \begin{theorem}
            Every torus is the quotient $\C/\Gamma$ where $\Gamma=\Z\omega_1\oplus\Z\omega_2$ for some linearly independent $\omega_1,\omega_2\in\C$.
        \end{theorem}
        \pause
        \ \\
        \begin{center}
            \includegraphics[width=0.9\textwidth]{figures/torus_glue.png}
        \end{center}
    \end{frame}
    \begin{frame}{Complex tori}
        \begin{theorem}
            There is, up to homeomorphism, only one topological torus.
        \end{theorem}

        \pause

        \begin{block}{Question}
            \textit{In how many ways can we give the torus complex coordinates?}
        \end{block}
    \end{frame}
    \begin{frame}{Complex tori}
        \begin{center}
            \includegraphics[width=\textwidth]{figures/torus_scale.png}
        \end{center}
    \end{frame}
    \begin{frame}{Complex tori}
        \begin{theorem}
            Every complex tori is the quotient $X_\tau\coloneqq\C/\Gamma$ where $\Gamma=\Z\oplus\Z\tau$ for some $\tau\in\H$. \pause Indeed, for all linearly independent $\omega_1,\omega_2\in\C$,
            \begin{equation*}
                \C/\!\l(\Z\omega_1\oplus\Z\omega_2\r)\iso X_\tau
            \end{equation*}
            for $\tau\coloneqq\omega_2/\omega_1\in\H$.
        \end{theorem}
    \end{frame}
    \begin{frame}{Complex tori}
        \begin{center}
            \includegraphics[width=\textwidth]{figures/torus_fundamental_domain_1.png}
        \end{center}
    \end{frame}
    \begin{frame}{Complex tori}
        \begin{center}
            \includegraphics[width=\textwidth]{figures/torus_fundamental_domain_2.png}
        \end{center}
    \end{frame}
    \begin{frame}{Complex tori}
        \begin{center}
            \includegraphics[width=\textwidth]{figures/torus_fundamental_domain_3.png}
        \end{center}
    \end{frame}
    \begin{frame}{Complex tori}
        \begin{definition}
            \textit{The \ul{modular group} $\PSL{2}{\Z}$ is the group of functions $\gamma:\H\to\H$ mapping}
            \begin{equation*}
                \tau\mapsto\frac{a\tau+b}{c\tau+d}
            \end{equation*}
            \textit{for some $a,b,c,d\in\Z$ with $ad-bc=1$.}
        \end{definition}

        \pause

        \begin{theorem}
            For any $\tau,\tau'\in\H$, the tori $X_\tau$ and $X_{\tau'}$ are biholomorphic iff there exists some $\gamma\in\PSL{2}{\Z}$ such that $\tau'=\gamma\l(\tau\r)$.
        \end{theorem}

        \pause

        \begin{corollary}
            The moduli space of complex tori is $\H/\PSL{2}{\Z}$.
        \end{corollary}
    \end{frame}
\end{document}
