\documentclass[../Moduli_Spaces_of_Riemann_Surfaces.tex]{subfiles}
\begin{document}
    \section{Charts and Atlases}
    We assume that the reader is familiar with the basic notions of real manifolds. The case for complex manifolds is similar, so our exposition will be brief.
    \begin{definition}
        Let $X$ be a second-countable Hausdorff space. A \uldef{$d$-dimensional complex} \uldef{chart on $X$} is a pair $\tpl{U,\phi}$ where $\phi:U\to V$ is a homeomorphism from an open subset $U\subseteq X$ onto an open subset $V\subseteq\C^d$ for some $d$. Two $d$-dimensional charts $\l(U_1,\phi_1\r)$ and $\l(U_2,\phi_2\r)$ are said to be \uldef{holomorphically compatible} if either $U_1\cap U_2=\em$, or the map
        \begin{equation*}
            \phi_2\circ\phi_1^{-1}:\phi_1\l(U_1\cap U_2\r)\to\phi_2\l(U_1\cap U_2\r)
        \end{equation*}
        is biholomorphic. A \uldef{$d$-dimensional complex atlas on $X$} is a collection $\ms{A}\coloneqq\l\{\l(U_i,\phi_i\r)\r\}_{i\in I}$ of $d$-dimensional complex charts such that every two charts $\l(U_i,\phi_i\r)$ and $\l(U_j,\phi_j\r)$ are holomorphically compatible and $X=\bigcup_{i\in I}U_i$.
    \end{definition}
    \side[-1.31in]{Charts provide us a way of making $X$ $\textrm{`}$look like$\textrm{'}$ an open set in $\C^d$. Indeed, they provide local coordinates for every point in $X$ in such a way that the $\textrm{`}$change of coordinates$\textrm{'}$ map $\phi_2\circ\phi_1^{-1}$ ensures that local notions of functions in $\C^d$ are well-defined on $X$ too.
        \begin{equation*}
            \begin{tikzcd}[ampersand replacement=\&, column sep=0.1in]
                \& U_1\cap U_2 \ar[dl, "\phi_1"'] \ar[dr, "\phi_2"] \\
                \phi_1(U_1\cap U_2) \ar[rr, "\phi_2\circ\phi_1^{-1}"'] \& \& \phi_2(U_1\cap U_2)
            \end{tikzcd}       
        \end{equation*}
    It is clear that one only needs $\phi_2\circ\phi_1^{-1}$ to be holomorphic for it to be biholomorphic.}
    \begin{remark}
        Two atlases $\ms{A}$ and $\ms{B}$ on a manifold $X$ are said to be \ul{analytically equivalent} if every chart in $\ms{A}$ is compatible with every chart in $\ms{B}$. By Zorn's Lemma, every atlas $\ms{A}$ of a manifold $X$ is contained in a unique maximal atlas $\mf{U}$ on $X$. Moreover, two atlases are equivalent iff they are contained in the same maximal atlas, which justifies the following definition.\exqed
    \end{remark}
    \begin{definition}
        Let $X$ be a second-countable Hausdorff space. A \uldef{$d$-dimensional complex} \uldef{structure} on $X$ is a $d$-dimensional maximal atlas $\mf{U}$ on $X$, or, equivalently, an equivalence class of $d$-dimensional complex atlases on $X$. The pair $\tpl{X,\mf{U}}$ is then called a \uldef{$d$-dimensional} \uldef{complex manifold}.
    \end{definition}
    \side[-0.54in]{To give a complex structure $\mf{U}$ to $X$, it suffices to give $X$ a complex atlas since it extends to a \textit{unique} complex structure.}
    \begin{definition}
        A \uldef{Riemann surface} is a connected $1$-dimensional complex manifold.
    \end{definition}
    \begin{example}\side[-0.15in]{Every Riemann surface can be regarded as a (connected) $2$-dimensional real manifold by $\textrm{`}$forgetting$\textrm{'}$ its complex structure; indeed all holomorphic maps are real $\mc{C}^\infty$ functions.}
        Some elementary examples of Riemann surfaces.
        \begin{itemize}
            \item The complex plane $\C$, equipped with its standard topology, can be given a complex structure $\mf{U}$ by choosing the atlas containing a single chart $\tpl{\C,\id_\C}$. We may, however, also give $\C$ a different complex structure $\mf{U}'$ by choosing the chart map $\phi:z\mapsto\bar{z}$ instead. Indeed, $\mf{U}\neq\mf{U}'$ since the map $\phi\circ\id_\C^{-1}=\phi$ is not holomorphic and hence the atlases $\l\{\tpl{\C,\id_\C}\r\}$ and $\l\{\tpl{\C,\phi}\r\}$ are not equivalent. This example generalizes to any domain $D\subseteq\C$.
            \item Let $D\subseteq\C$ be a domain and consider any holomorphic function $f:D\to\C$. Then the graph $\Gamma\!_f\coloneqq\l\{\tpl{z,f\l(z\r)}\mid z\in D\r\}$, equipped with the subspace topology inherited from $\C^2$, can be given a complex structure by choosing the chart map $\pi:\Gamma\!_f\to D:\tpl{z,f\l(z\r)}\mapsto z$.\exqed
        \end{itemize}
    \end{example}
    \subsection{The Riemann Sphere $\RS$}
    A particularly important Riemann surface is the Riemann sphere\side[0.01in]{Showing that \textit{every} Riemann surface that is topologically a sphere is biholomorphic to $\RS$ is a non-trivial task, and it will be the first goal of this paper to establish this fact.} $\RS$, which admits several constructions. Here, we give three; see Example \ref{1.2:exa:biholomorphisms_between_Riemann_spheres} for a proof that they are all biholomorphic (in the sense of Definition $\ref{1.2:def:biholomorphic_Riemann_surfaces}$).
    \begin{example}[One-point Compactification of $\C$]\label{1.1:exa:one_point_compactification_of_C}
        Let $\infty$ be a symbol not belonging to $\C$ and set $\C_\infty\!\coloneqq\C\cup\l\{\infty\r\}$. We declare a set $U\subseteq\C_\infty\!$ to be open if either $U\subseteq\C$ is open or $U=K^c\cup\l\{\infty\r\}$ where $K\subseteq\C$ is compact.\side{This makes $\C_\infty\!$, equipped with the collection $\mc{T}$ of all such open sets, a second-countable Hausdorff space. Indeed, the fact that $\mc{T}$ is a topology on $\C_\infty\!$ follows from De Morgan's Laws and the Heine-Borel Theorem. It is trivially Hausdorff, and it is second-countable since we may append, to any countable basis for the standard topology of $\C$, the countable collection $\l\{B_r\l(0\r)^c\cup\l\{\infty\r\}\r\}_{r\in\Q^+}$.} We employ two charts
        \begin{equation*}
            \begin{alignedat}{2}
                U_1&\coloneqq\C_\infty\!\comp\l\{\infty\r\}=\C\ \ \ \ \ \ \ \ \ \ \ \ \ \ \ \ \ \ \ \ &&\phi_1:U_1\to\C:z\mapsto z\ \ \ \ \l(\phi_1\coloneqq\id_\C\r) \\
                U_2&\coloneqq\C_\infty\!\comp\l\{0\r\}=\C^\ast\cup\l\{\infty\r\}&&\phi_2:U_2\to\C:z\mapsto
                \begin{dcases}
                    1/z & \textrm{if }z\in\C^\ast \\
                    0 & \textrm{else.}
                \end{dcases}
            \end{alignedat}
        \end{equation*}
        Clearly $\phi_1$ is a homeomorphism. Since $\phi_2$ is invertible with $\phi_2^{-1}\!\l(z\r)\coloneqq1/z$ for all $z\in\C^\ast$ and $\phi_2^{-1}\!\l(0\r)\coloneqq\infty$, and
        \begin{equation*}
            \lim\limits_{z\to\infty}\phi_2\!\l(z\r)=0=\phi_2\!\l(\infty\r)\ \ \ \ \ \ \ \ \textrm{and}\ \ \ \ \ \ \ \ \lim\limits_{z\to0}\phi_2^{-1}\!\l(z\r)=\infty=\phi_2^{-1}\l(0\r),
        \end{equation*}
        we see that $\phi_2$ is a homeomorphism too. Furthermore,
        \begin{equation*}
            \phi_2\circ\phi_1^{-1}:\C^\ast\to\C^\ast:z\mapsto\frac{1}{z}
        \end{equation*}
        is holomorphic, so the atlas $\l\{\tpl{U_1,\phi_1},\tpl{U_2,\phi_2}\r\}$ defines a complex structure on $\C_\infty\!$.\exqed
    \end{example}
    \begin{example}[Stereographic Projection]\label{1.1:exa:stereographic_projection}
        Consider the unit sphere $S^2\subseteq\R^3$ as a topological subspace of $\R^3$, which makes it a second-countable Hausdorff space. Identifying the plane $w=0$ as $\C$, we employ the charts
        \begin{equation*}
            \begin{alignedat}{2}
                U_1&\coloneqq S^2\comp\l\{\l(0,0,1\r)\r\}\ \ \ \ \ \ \ \ \ \ \ \ \ \ \ \ &&\phi_1:U_1\to\C:\tpl{x,y,w}\mapsto\frac{x+iy}{1-w} \\
                U_2&\coloneqq S^2\comp\l\{\l(0,0,-1\r)\r\}&&\phi_2:U_2\to\C:\tpl{x,y,w}\mapsto\frac{x-iy}{1+w}.
            \end{alignedat}
        \end{equation*}
        Clearly $\phi_1$ and $\phi_2$ are continuous, and it can be verified that they are invertible with continuous inverses
        \begin{equation*}
            \phi_1^{-1}\!\l(z\r)\coloneqq\tpl{\frac{2\Re z}{\l|z\r|^2+1},\frac{2\Im z}{\l|z\r|^2+1},\frac{\l|z\r|^2-1}{\l|z\r|^2+1}}\ \ \ \ \textrm{and}\ \ \ \ \phi_2^{-1}\!\l(z\r)\coloneqq\tpl{\frac{2\Re z}{\l|z\r|^2+1},\frac{-2\Im z}{\l|z\r|^2+1},\frac{1-\l|z\r|^2}{\l|z\r|^2+1}}.
        \end{equation*}
        Observe that $U_1\cap U_2=S^2\comp\l\{\tpl{0,0,\pm1}\r\}$ and $\phi_2\circ\phi_1^{-1}:\C^\ast\to\C^\ast:z\mapsto1/z$, which is holomorphic, so the atlas $\l\{\tpl{U_1,\phi_1},\tpl{U_2,\phi_2}\r\}$ defines a complex structure on $\RS$.\exqed
    \end{example}
    \begin{example}[Complex Projective Line]\label{1.1:exa:complex_projective_line}
        Consider the equivalence relation $\sim$ on $\C^2\comp\l\{\tpl{0,0}\r\}$ defined by $\tpl{z_1,w_1}\sim\tpl{z_2,w_2}$ iff $\tpl{z_1,w_1}=\lambda\tpl{z_2,w_2}$ for some $\lambda\in\C^\ast$. Set $\P^1\coloneqq\l(\C^2\comp\l\{\tpl{0,0}\r\}\r)/\!\sim$ and equip it with the quotient topology. Since $\sim$ is an open equivalence relation whose graph is closed in $\l(\C^2\comp\l\{\tpl{0,0}\r\}\r)^2$, we see that $\P^1$ is a second-countable Hausdorff space.\side{See \cite[][section 7.5]{tu}.} Denoting the equivalence class of $\tpl{z,w}$ by $\proje{z:w}$, we employ the charts
        \begin{equation*}
            \begin{alignedat}{2}
                U_1&\coloneqq\P^1\comp\l\{\proje{0:w}\mid w\in\C\r\}\ \ \ \ \ \ \ \ \ \ \ \ &&\phi_1:U_1\to\C:\proje{z:w}\mapsto w/z \\
                U_2&\coloneqq\P^1\comp\l\{\proje{z:0}\mid z\in\C\r\}&&\phi_2:U_2\to\C:\proje{z:w}\mapsto z/w.
            \end{alignedat}
        \end{equation*}
        Clearly $\phi_2$ and $\phi_2$ are continuous, and it is easily verified that they are invertible with continuous inverses
        \begin{equation*}
            \phi_1^{-1}\!\l(z\r)\coloneqq\proje{1:z}\ \ \ \ \ \ \ \ \textrm{and}\ \ \ \ \ \ \ \ \phi_2^{-1}\!\l(z\r)\coloneqq\proje{z:1}.
        \end{equation*}
        Furthermore, $\phi_2\circ\phi_1^{-1}:\C^\ast\to\C^\ast:1\mapsto1/z$ is holomorphic, so the atlas $\l\{\tpl{U_1,\phi_1},\tpl{U_2,\phi_2}\r\}$ defines a complex structure on $\P^1$.\exqed
    \end{example}
    \subsection{Complex Tori}
    Recall that a torus is any manifold homeomorphic to $T^2\coloneqq S^1\times S^1$, which admits a representation as a quotient $\C/\Gamma$ by the lattice $\Gamma\coloneqq\Z\oplus\Z$. Thus (by definition) there is only one torus up to homeomorphism, but it turns out that we can equip it with many different complex structures.\side[0.01in]{They manifest by quotienting $\C$ by different lattices, and we shall derive a criterion on $\Gamma_1\coloneqq\Z\omega_1\oplus\Z\omega_2$ and $\Gamma_2\coloneqq\Z\eta_1\oplus\Z\eta_2$ for the tori $\C/\Gamma_1$ and $\C/\Gamma_2$ to be biholomorphic.}
    \begin{example}[Complex Tori]
        Let $\omega_1,\omega_2\in\C$ be linearly independent over $\R$ and consider the lattice $\Gamma\coloneqq\Z\omega_1\oplus\Z\omega_2$. Then the quotient $\C/\Gamma$ is a torus in the topological sense since the map
        \begin{equation*}
            \phi:\C/\Gamma\to T^2\ \ \ \ \ \ \ \ \textrm{mapping}\ \ \ \ \ \ \ \ \l[z\r]\mapsto\tpl{e^{2\pi i\lambda_1},e^{2\pi i\lambda_2}},
        \end{equation*}
        where $z=\lambda_1\omega_1+\lambda_2\omega_2$ for unique $\lambda_1,\lambda_2\in\R$, is a homeomorphism. Indeed, $\phi$ is well-defined since for any $\lambda_1\omega_1+\lambda_2\omega_2\sim\mu_1\omega_1+\mu_2\omega_2$ in $\C$, we have $\l(\lambda_1-\mu_1\r)\omega_1+\l(\lambda_2-\mu_2\r)\omega_2\in\Gamma$ and so $\lambda_i-\mu_i\in\Z$ for $i=1,2$. The fact that it is a homeomorphism is clear. This makes $\C/\Gamma$ a second-countable Hausdorff space, which we now endow with the following complex structure.\\\ \\
        Since $\Gamma$ is discrete, there exists some $\epsilon>0$ such that $\epsilon<\l|\omega\r|/2$ for every non-zero $\omega\in\Gamma$.\side{This exposition follows \cite[][Section I.2]{miranda}.} Fix any such $\epsilon$, which ensures that no two points in any open ball with radius $\epsilon$ can be equivalent. Indeed, take any $z\in\C$ and $w_1,w_2\in B\l(z,\epsilon\r)\eqqcolon V_z$. For $\omega_1\sim\omega_2$, we need some $n,m\in\Z$ such that $w_1-w_2=n\omega_1+m\omega_2$. But
        \begin{equation*}
            \l|w_1-w_2\r|\leq\l|z-w_1\r|+\l|z-w_2\r|<2\epsilon<\l|n\omega_1+m\omega_2\r|
        \end{equation*}
        for any $n,m\in\Z$, so this is impossible. Fixing any such $\epsilon$, this gives us a family $\l\{V_z\r\}_{z\in\C}$ of open sets in $\C$ for which the projections $\l.\pi\r|_{V_z}:V_z\to\pi\l(V_z\r)$ are homeomorphisms.\side[0.01in]{The choice of $\epsilon$ ensures that no two points in $V_z$ are equivalent, which make all such projections injective.} Letting $U_z\coloneqq\pi\l(V_z\r)$ and $\phi_z:U_z\to V_z$ be the inverse of $\l.\pi\r|_{V_z}$, we obtain complex charts $\tpl{U_z,\phi_z}$ for all $z\in\C$. We claim that the collection $\mf{U}\coloneqq\l\{\tpl{U_z,\phi_z}\r\}_{z\in\C}$ form an atlas, for which it suffices to take $\tpl{U_1,\phi_1},\tpl{U_2,\phi_2}\in\mf{U}$ and show that the transition map $T\coloneqq\phi_2\circ\phi_1^{-1}:\phi_1\!\l(U\r)\to\phi_2\!\l(U\r)$, where $U\coloneqq U_1\cap U_2$, is holomorphic. Observe that the diagram
        \begin{equation*}
            \begin{tikzcd}[column sep=0.1in]
                & U \ar[dl, "\phi_1"] \ar[dr, "\phi_2"'] \\
                V_1=\phi_1(U) \ar[ur, "\l.\pi\r|_{V_1}", bend left=20, start anchor={[xshift=-20px]}] \ar[rr, "T"'] & & \phi_2(U)=V_2 \ar[ul, "\l.\pi\r|_{V_2}"', bend right=20, start anchor={[xshift=20px]}]
            \end{tikzcd}
        \end{equation*}
        commutes, so $\l.\pi\r|_{V_2}\circ T=\l.\pi\r|_{V_1}$ on $\phi_1\!\l(U\r)$. Then $\pi\l(T\l(z\r)\r)=\pi\l(z\r)$ for every $z\in\phi_1\!\l(U\r)$, so $T\l(z\r)\sim z$ and hence $\ell\l(z\r)\coloneqq T\l(z\r)-z\in\Gamma$. This holds for all $z\in\phi_1\!\l(U\r)$, so we obtain a continuous function $\ell:\phi_1\!\l(U\r)\to\Gamma:z\mapsto T\l(z\r)-z$.\side[0.01in]{Since $U=\pi\l(V_1\r)\cap\pi\l(V_2\r)$, it may not be connected. Hence $\phi_1\!\l(U\r)$ may not be connected, so $\ell$ may take on multiple values. What matters, hoverer, is that they coincide within every connected component of of $\phi_1\!\l(U\r)$.} Note that $\Gamma\subseteq\C$ is equipped with the subspace topology, but since it is discrete, every $L\subseteq\Gamma$ is open. In particular, fix $z_0\in\phi_1\!\l(U\r)$ and set $\omega_0\coloneqq T\l(z_0\r)-z_0$. With $L\coloneqq\l\{\omega_0\r\}$, continuity of $\ell$ shows that $\ell^{-1}\!\l(L\r)$ is open. Thus $\ell\l(B\l(z_0,\delta_1\r)\r)\subseteq\l\{\omega_0\r\}$ for some $\delta_1>0$, so $\ell\l(w\r)=\omega_0$ for all $w\in B\l(z_0,\delta_1\r)$. But then $\ell\l(B\l(\omega_0,\delta_2\r)\r)\subseteq\l\{\omega_0\r\}$ for some $\delta_2>0$ too, so we may repeat this process to show that $\ell$ is constant on every connected component of $\phi_1\!\l(U\r)$. Thus $T\l(z\r)=z+\omega_0$ for all $z\in\phi_1\!\l(U\r)$ in a local neighborhood around $z_0$, so $T$ is locally holomorphic. But this holds for all $z_0\in\phi_1\!\l(U\r)$, so $T$ is holomorphic on $\phi_1\!\l(U\r)$.\exqed
    \end{example}
\end{document}
