\documentclass[../Moduli_Spaces_of_Riemann_Surfaces.tex]{subfiles}
\begin{document}
    \section{Maps on Riemann Surfaces}
    \subsection{Holomorphic Functions and Maps}
    \begin{definition}
        Let $X$ be a Riemann surface and let $W\subseteq X$ be open. For a fixed $p\in W$, a function $f:W\to\C$ is said to be \uldef{holomorphic at $p$} if there exists a chart $\tpl{U,\phi}$ of $X$ containing $p$ such that $f\circ\phi^{-1}:\phi\l(U\r)\to\C$ is holomorphic at $\phi\l(p\r)$. If $f$ is holomorphic at every point of $W$, then $f$ is said to be \uldef{holomorphic on $W$}.
    \end{definition}
    \begin{remark}\side[-0.57in]{Defining some property $P$ of $f$ using charts by transporting $f$ to a function $f\circ\phi^{-1}$ on a subset of $\C$, and borrowing $P$ from $f\circ\phi^{-1}$, will be a common theme. However, one must check that $P$ is \textit{independent of charts}; that is, if $f\circ\phi^{-1}$ satisfies $P$, then so does $f\circ\psi^{-1}$ for any other chart $\tpl{V,\psi}$.}
        It must be checked that $\textrm{`}$being holomorphic$\textrm{'}$ does not depend on the choice of chart. This is indeed the case, for if $\tpl{V,\psi}$ is another chart containing $p$, then, since
        \begin{equation}\label{1.2:eqn:holomorphic_function_well_defined}
            f\circ\psi^{-1}=f\circ\l(\phi^{-1}\circ\phi\r)\circ\psi^{-1}=\l(f\circ\phi^{-1}\r)\circ\l(\phi\circ\psi^{-1}\r):\psi\l(U\cap V\r)\to\C
        \end{equation}\side[-0.45in]{
            \begin{equation*}
                \begin{tikzcd}[ampersand replacement=\&, column sep = 0.1in]
                    \phi\l(U\cap V\r) \ar[drrrr, "f\circ\phi^{-1}", bend left = 20] \\
                    \& U\cap V \ar[ul, "\phi"'] \ar[dl, "\psi"] \ar[rrr, "f"] \& \& \& \C \\
                    \psi\l(U\cap V\r) \ar [urrrr, "f\circ\psi^{-1}"', bend right = 20]
                \end{tikzcd}
            \end{equation*}
        }
        on the intersection $U\cap V$, we see that $f\circ\psi^{-1}:\psi\l(V\r)\to\C$ is also holomorphic at $p$.\exqed
    \end{remark}
    \begin{example}\label{1.2:exa:elementary_holomorphic_functions}
        Some elementary examples of holomorphic functions.
        \begin{itemize}
            \item Any holomorphic function $f:W\to\C$ from an open set $W\subseteq\C$, considering $\C$ as a Riemann surface with the standard chart $\l(\C,\id_\C\r)$, is holomorphic in the classical sense.
            \item Any chart map $\phi:U\to\C$ of a Riemann surface is (tautologically) holomorphic in the above sense.
            \item If $f,g:W\to\C$ are both holomorphic at some $p\in W$, then\side{This makes the set $\mc{O}\l(W\r)$ of all holomorphic functions $f:W\to\C$ into a $\C$-algebra.} so are $f\pm g$ and $f\cdot g$. If $g\l(p\r)\neq0$, then so is $f/g$.\exqed
        \end{itemize}
    \end{example}
    \begin{definition}
        Let $X$ and $Y$ be Riemann surfaces and let $W\subseteq X$ be open. For a fixed $p\in W$, a mapping $F:W\to Y$ is said to be \uldef{holomorphic at $p$} if there exists a chart $\tpl{U,\phi}$ of $X$ containing $p$ and a chart $\tpl{V,\psi}$ of $Y$ containing $F\l(p\r)$ such that $\psi\circ F\circ\phi^{-1}\!\!:\phi\l(U\r)\to\psi\l(V\r)$ is holomorphic at $\phi\l(p\r)$. If $F$ is holomorphic at every point of $W$, then $F$ is \uldef{holomorphic on $W$}.
    \end{definition}\side[-0.7in]{For $Y\coloneqq\C$ regarded as a Riemann surface, this definition agrees with the above. Again, we must check that $\textrm{`}$being holomorphic$\textrm{'}$ is well-defined, but it follows from the commutativity of the diagram below.
        \begin{equation*}
            \begin{tikzcd}[ampersand replacement=\&, column sep = 0.05in]
                \phi_1\!\l(U_1\cap U_2\r) \ar[rrrrrrr, "\psi_1\circ F\circ\phi_1^{-1}", bend left = 10] \ar[dd, "\phi_2\circ\phi_1^{-1}"'] \& \& \& \& \& \& \& \psi_1\!\l(V_1\cap V_2\r) \ar[dd, "\psi_2\circ\psi_1^{-1}"] \\
                \& \hspace{-0.23in}U_1\cap U_2 \ar[ul, "\phi_1"'] \ar [dl, "\phi_2"] \ar[rrrrr, "F"] \& \& \& \& \& V_1\cap V_2\hspace{-0.23in} \ar[ur, "\psi_1"] \ar[dr, "\psi_2"'] \\
                \phi_2\!\l(U_1\cap U_2\r) \ar[rrrrrrr, "\psi_2\circ F\circ\phi_2^{-1}"', bend right = 10] \& \& \& \& \& \& \& \psi_2\!\l(V_1\cap V_2\r)
            \end{tikzcd}
        \end{equation*}}
    \begin{example}
        It is easy to show that the identity map $\id_X$ on a Riemann surface $X$ is a holomorphic map. Furthermore, for all Riemann surfaces $X$, $Y$ and $Z$ and holomorphic maps $F:X\to Y$ and $G:Y\to Z$, their composite $G\circ F:X\to Z$ is also a holomorphic map. This shows that the collection of all Riemann surfaces is a \textit{category}.\exqed
    \end{example}
    \begin{definition}\label{1.2:def:biholomorphic_Riemann_surfaces}
        Let $X$ and $Y$ be Riemann surfaces. A \uldef{biholomorphism between $X$ and $Y$} is an invertible holomorphic map $F:X\to Y$ whose inverse $F^{-1}:Y\to X$ is also holomorphic. Two Riemann surfaces $X$ and $Y$ are said to be \uldef{biholomorphic} if there exists a biholomorphism $F:X\to Y$.
    \end{definition}
    \begin{example}[Biholomorphisms between Riemann spheres]\label{1.2:exa:biholomorphisms_between_Riemann_spheres}
        Let $\C_\infty$, $S^2$, and $\P^1$ denote the three constructions for the Riemann sphere $\RS$ presented in Examples \ref{1.1:exa:one_point_compactification_of_C}, \ref{1.1:exa:stereographic_projection}, and \ref{1.1:exa:complex_projective_line}, respectively. We claim that the maps\side[0.28in]{Take $G\l(0,0,1\r)\coloneqq\infty$.}
        \begin{equation*}
            F:S^2\to\P^1:\tpl{x,y,w}\mapsto\proje{1-w:x+iy}\ \ \ \ \ \ \ \ \textrm{and}\ \ \ \ \ \ \ \ G:S^2\to\C_\infty:\tpl{x,y,w}\mapsto\frac{x+iy}{1-w}
        \end{equation*}
        are biholomorphisms, which shows that all three constructions are biholomorphic.\side{Since the collection of Riemann surfaces form a category, the $\textrm{`}$is isomorphic to$\textrm{'}$ relation is an equivalence relation. Thus we are justified to call all three constructions $\textrm{`}$the$\textrm{'}$ Riemann sphere, and, henceforth, we shall denote all three by $\RS$.} Indeed $F$ is holomorphic since with the charts
        \begin{equation*}
            \begin{alignedat}{2}
                U&\coloneqq S^2\comp\l\{\l(0,0,1\r)\r\}\ \ \ \ \ \ \ \ \ \ \ \ \ \ \ \ \ \ \ \ &&\phi:U\to\C:\tpl{x,y,w}\mapsto\frac{x+iy}{1-w} \\
                V&\coloneqq\P^1\comp\l\{\proje{0:w}\mid w\in\C\r\}&&\psi:V\to\C:\proje{z:w}\mapsto\frac{w}{z},
            \end{alignedat}
        \end{equation*}
        we see that\side{A similar calculation shows that $G$ is biholomorphic. Indeed, we choose the same chart $\tpl{U,\phi}$, and choose $V\coloneqq\C_\infty\comp\l\{\infty\r\}=\C$ with $\psi\coloneqq\id_\C$. Then $\l(\psi\circ G\circ\phi^{-1}\r)\!\l(z\r)=z$ for all $z\in\phi\l(U\r)=\C$, and $G$ is invertible with inverse
            \begin{equation*}
                G^{-1}\!\l(z\r)\coloneqq
                \begin{dcases}
                    \phi^{-1}\!\l(z\r) & \textrm{if }z\in\C \\
                    \tpl{0,0,1} & \textrm{else.}
                \end{dcases}
            \end{equation*}}
        \begin{equation*}
            \begin{aligned}
                \l(\psi\circ F\circ\phi^{-1}\r)\l(z\r)&=\psi\l(F\l(\frac{2\Re z}{\l|z\r|^2+1},\frac{2\Im z}{\l|z\r|^2+1},\frac{\l|z\r|^2-1}{\l|z\r|^2+1}\r)\r) \\
                                                      &=\psi\l(\proje{1-\frac{\l|z\r|^2-1}{\l|z\r|^2+1}:\frac{2z}{\l|z\r|^2+1}}\r) \\
                                                      &=\psi\l(\proje{1:z}\r) \\
                                                      &=z
            \end{aligned}
        \end{equation*}
        for all $z\in\phi\l(U\r)=\C$, which is clearly holomorphic. Furthermore, it can be checked that $F$ is invertible with inverse\side{}
        \begin{equation*}
            F^{-1}\l(\proje{z:w}\r)\coloneqq\frac{\tpl{2\Re\l(z\bar{w}\r),2\Im\l(z\bar{w}\r),\l|z\r|^2-\l|w\r|^2}}{\l|z\r|^2+\l|w\r|^2},
        \end{equation*}
        which is well-defined, and since $\l(\psi\circ F\circ\phi^{-1}\r)^{-1}=\phi\circ F^{-1}\circ\psi^{-1}$, we see that $F^{-1}$ is holomorphic too.\exqed
    \end{example}
    \begin{proposition}\label{1.2:prp:holomorphic_compact_constant}
        Any holomorphic function $f:X\to\C$ on a compact Riemann surface $X$ is constant.
    \end{proposition}
    \begin{proof}
        Since $f$ is holomorphic, the function $\l|f\r|:X\to\R$ defined by $\l|f\r|\l(x\r)\coloneqq\l|f\l(x\r)\r|$ is continuous on $X$. But $X$ is compact, so $\l|f\r|$ achieves its maximum at some point $p\in X$. Choosing a connected\side{Such a connected $U$ can always be found since we may let $V$ be a chart around $p$ and choose $\epsilon>0$ small enough so that $U\coloneqq B\l(p,\epsilon\r)\subseteq V$.} chart $\tpl{U,\phi}$ centered at $p$, we see that $f\circ\phi:U\to\C$ is holomorphic. Then $\l|f\circ\phi\r|:U\to\R$ has a local maximum at $0$, so, since $U$ is connected, $f\circ\phi$ is constant by the Maximum Principle. Then $f$ is locally constant around $p$, so, since $X$ is connected, $f$ is constant on $X$.\qed
    \end{proof}
    \subsection{Singularities of Functions}
    Throughout this section, let $X$ be a Riemann surface, let $p\in X$, and let $f:W\to\C$ be defined and holomorphic on a punctured neighborhood $W$ of $p$.\side[0.01in]{That is, let $f$ be defined and holomorphic on $B\l(p,\epsilon\r)\comp\l\{p\r\}$ for some $\epsilon>0$.} As above, we can transport the behaviour of $f$ at $p$ from its chart representation $f\circ\phi^{-1}$.
    \begin{definition}
        Let $f:W\to\C$ be a holomorphic function in a punctured neighborhood of $p$. We say that $f$ has a \uldef{removable singularity} (resp. \uldef{pole}, \uldef{essential singularity}) \uldef{at $p$} if there exists a chart $\tpl{U,\phi}$ of $X$ containing $p$ such that $f\circ\phi^{-1}:\phi\l(U\r)\to\C$ has a removable singularity (resp. pole, essential singularity) at $\phi\l(p\r)$.
    \end{definition}\side[-0.56in]{We recall those notions from complex analysis. Let $f:W\subseteq\C\to\C$ be a holomorphic function (in the regular sense) in a punctured neighborhood of $p$. Suppose that $f$ is not holomorphic at $p$.
        \begin{itemize}
            \item If $\lim_{z\to p}f\l(z\r)$ exists, then $f$ has a \ul{removable singularity at $p$}.
            \item If $\lim_{z\to p}f\l(z\r)=\pm\infty$, then $f$ has a \ul{pole at $p$}. This is equivalent to the existence of some $n>0$ such that the limit $\lim_{z\to p}\l(z-p\r)^nf\l(z\r)$ exists. See Definition \ref{1.2:def:order}.
            \item Otherwise, $f$ has an \ul{essential singularity} \ul{at $p$}.
        \end{itemize}
    }
    \vspace{-0.05in}
    \begin{proof}
        (Well-defined): Equation (\ref{1.2:eqn:holomorphic_function_well_defined}) shows that those notions are chart independent; the composition of $f\circ\phi^{-1}$ having a singularity at $p$ with a transition map that is holomorphic at $p$ yields a function with the same type of singularity at $p$.\qed
    \end{proof}
    \begin{remark}
        Functions having an essential singularity at $p$ are very ill-behaved. Indeed, this occurs iff $\l|f\l(x\r)\r|$ has a non-zero oscillation near $p$. Other singularities behave much better:
        \begin{itemize}
            \item A removable singularity occurs iff $\l|f\l(x\r)\r|$ is bounded in a neighborhood of $p$, and can be $\textrm{`}$filled in$\textrm{'}$ by defining $\tilde{f}\l(p\r)\coloneqq\lim_{x\to p}f\l(x\r)$. This makes $\tilde{f}:\tilde{W}\to\C$ into\side{$\tilde{W}\coloneqq W\cup\l\{p\r\}$.} a holomorphic function.
            \vspace{-0.08in}
            \item A pole occurs iff $\l|f\l(x\r)\r|\to\infty$ as $x\to p$, which can also be $\textrm{`}$filled in$\textrm{'}$ by defining the map
                \begin{equation*}
                    F:W\to\RS\ \ \ \ \ \ \ \ \textrm{mapping}\ \ \ \ \ \ \ \ x\mapsto
                    \begin{dcases}
                        \infty & \textrm{if }x=p \\
                        f\l(x\r) & \textrm{else}
                    \end{dcases}
                \end{equation*}
                that extends the codomain of $f$ to the Riemann sphere $\RS$;\side{Here, we consider $\RS=\C_\infty$.} it is clear that $F$ is holomorphic.
        \end{itemize}
        Thus we see that every such function $f:W\to\C$ having pole at $p$ can be holomorphically extended to a map $F:W\to\RS$. Conversely, every holomorphic map $F:W\to\RS$ (that is not identically zero) can be regarded as a function $f:W\comp F^{-1}\l(\infty\r)\to\C$ that is holomorphic everywhere except where $F\l(x\r)=\infty$, in which case it either has a pole. This motivates the following definition.\exqed
    \end{remark}
    \begin{definition}
        A function $f:W\to\C$ is said to be \uldef{meromorphic at $p$} if it does not have an essential singularity at $p$; that is, if it is either holomorphic, has a removable singularity, or has a pole at $p$. If $f$ is meromorphic at every point of $W$, then $f$ is \uldef{meromorphic on $W$}.
    \end{definition}\side[-0.44in]{As in Example \ref{1.2:exa:elementary_holomorphic_functions}, if $f,g:W\to\C$ are both meromorphic at $p$, then so are $f\pm g$ and $f\cdot g$. If $g$ is not identically $0$, then so is $f/g$. This makes the set $\mc{M}\l(W\r)$ of all meromorphic functions $f:W\to\C$ into a $\C$-algebra.}
    \begin{remark}
        The previous remark can now be rephrased by saying that the set of all meromorphic functions $f:W\to\C$ are in one-to-one correspondence with the set of all holomorphic maps $F:W\to\RS$ (which are not identically zero). That is, meromorphic functions are the holomorphic maps to the Riemann sphere.\exqed
    \end{remark}
    \begin{definition}\label{1.2:def:order}
        Let $f:W\to\C$ be meromorphic at $p$ and consider its Laurent series $f_\phi\!\l(z\r)\coloneqq\l(f\circ\phi^{-1}\r)\l(z\r)=\sum_{i}c_i\l(z-z_0\r)^i$ under a chart $\l(U,\phi\r)$ of $X$ with $z_0\coloneqq\phi\l(p\r)$. The \uldef{order of $f$ at $p$} is
        \begin{equation*}
            \ord_p\!\l(f\r)\coloneqq\min\l\{n\in\Z\mid0\neq\l(z-z_0\r)^n\!f_\phi\!\l(z\r)\in\mc{O}\l(W\r)\r\}.
        \end{equation*}
    \end{definition}
    \side[-0.53in]{Note that $f$, being meromorphic, ensures that its Laurent series has finitely-many negative terms. Thus the set $\l\{n\in\Z\mid c_n\neq0\r\}$ achieves its minimum, so the definition makes sense. If $f$ is not meromorphic, we take $\ord_p\!\l(f\r)\coloneqq\infty$.\\\ \\
        The arithmetic of $\ord_p$ is straightforward. Indeed, if $f,g:W\to\C$ are meromorphic at $p$, then
        \begin{itemize}
            \item $\ord_p\!\l(fg\r)=\ord_p\!\l(f\r)+\ord_p\!\l(g\r)$.
            \item $\ord_p\!\l(f/g\r)=\ord_p\!\l(f\r)-\ord_p\!\l(g\r)$, if $g\neq0$.
            \item $\ord_p\!\l(1/f\r)=-\ord_p\!\l(f\r)$, if $f\neq0$.
            \item $\ord_p\!\l(f\pm g\r)\geq\min\l\{\ord_p\!\l(f\r),\ord_p\!\l(g\r)\r\}$.
        \end{itemize}
    }\vspace{-0.05in}
    \begin{proof}
        (Well-defined). Let $z$ be the local coordinates given by $\l(U,\phi\r)$ and suppose that $\l(V,\psi\r)$ is another chart with $w_0\coloneqq\psi\l(p\r)$ giving another local coordinate $w$. Then the transition function $T:\phi\circ\psi^{-1}$ is holomorphic, so it admits a power series representation
        \begin{equation*}
            z=T\l(w\r)=\sum_{n\geq0}a_n\l(w-w_0\r)^n=z_0+\sum_{n\geq1}a_n\l(w-w_0\r)^n.
        \end{equation*}
        Since $T'\l(w_0\r)\neq0$, we see that $a_1\neq0$. Suppose now that the Laurent series of $f$ at $p$ in the coordinate $z$ is $c_{-n_0}\l(z-z_0\r)^{-n_0}+$ higher order terms, so that the order of $f$ at $p$ computed by employing $z$ is $n_0$. Then the Laurent series of $f$ at $p$ in the coordinate $w$ is
        \begin{equation*}
            c_{-n_0}\l(\sum_{n\geq1}a_n\l(w-w_0\r)^n\r)^{-n_0}+\textrm{ higher order terms},
        \end{equation*}
        whose lowest order term is $c_{-n_0}a_1^{-n_0}\l(w-w_0\r)^{-n_0}$. Observe that $b_{-n_0}\coloneqq c_{-n_0}a_1^{-n_0}\neq0$, so the order of $f$ at $p$ computed via $w$ is also $n_0$.\qed
    \end{proof}
    \begin{remark}
        The order $\ord_p\!\l(f\r)$ can be used to classify the behaviour of $f$ at $p$. Indeed, it is readily verified that $f$ is holomorphic at $p$ iff $\ord_p\!\l(f\r)\leq0$, in which case $f\l(p\r)=0$ iff $\ord_p\!\l(f\r)<0$. Similarly, $f$ has a pole at $p$ iff $\ord_p\!\l(f\r)>0$, so $f$ has neither a zero nor a pole at $p$ iff $\ord_p\!\l(f\r)=0$. This motivates the following definition.\exqed
    \end{remark}
    \begin{definition}
        Let $f:W\to\C$ be meromorphic at $p$. We say that $f$ has a \uldef{zero} (resp. \uldef{pole}) \uldef{of order $n$ at $p$} if $\ord_p\!\l(f\r)=n<0$ (resp. $n>0$).
    \end{definition}
    \subsection{Meromorphic Functions on $\RS$}
    \begin{example}
        Let $f:W\subseteq\RS\to\C$ be a non-zero rational function $f\l(z\r)\coloneqq p\l(z\r)/q\l(z\r)$. Then $f$ is holomorphic at all points $z\in\C$ such that $q\l(z\r)\neq0$, and has a pole otherwise. Also, $f\l(\infty\r)\in\C$ if $\deg p=\deg q$, vanishes if $\deg p<\deg q$, and has a pole otherwise. In any case, $f$ is meromorphic on $\RS$\side{In fact, \textit{any} meromorphic function on the Riemann sphere is a rational function; see Proposition \ref{1.2:prp:meromorphic_Riemann_sphere_rational}.}. To compute $\ord_z\!\l(f\r)$ at all $z\in\RS$, we split $p$ and $q$ into linear factors to write $f$ uniquely as
        \begin{equation*}
            f\l(z\r)=c\prod\l(z-\lambda_i\r)^{\alpha_i}
        \end{equation*}
        where $c\neq0$ and each $\lambda_i$ is distinct. Fix $i$. Setting $g_j\!\l(z\r)\coloneqq\l(z-\lambda_j\r)^{\alpha_j}$ for all $j$, we see that $\ord_{\lambda_i}\!\l(g_i\r)=-\alpha_i$ and $\ord_{\lambda_j}\!\l(g_i\r)=0$ for all $i\neq j$. Thus
        \begin{equation*}
            \ord_{\lambda_i}\!\l(f\r)=\sum_j\ord_{\lambda_i}\!\l(g_j\r)=-\alpha_i.
        \end{equation*}
        Moreover, if $\alpha_i>0$ (resp. $\alpha_i<0$), then $g_i$ has a pole (resp. zero) of order $\l|\alpha_i\r|$ at $\infty$. It follows then that $\ord_\infty\!\l(g_i\r)=\alpha_i$, so
        \begin{equation*}
            \ord_\infty\!\l(f\r)=\sum_i\ord_\infty\!\l(g_i\r)=\sum_i\alpha_i.
        \end{equation*}
        Lastly, it is clear that $\ord_z\!\l(f\r)=0$ for all $z\neq \lambda_i,\infty$.\exqed
    \end{example}
    \begin{proposition}\label{1.2:prp:meromorphic_Riemann_sphere_rational}
        Any meromorphic function on $\RS$ is a rational function.
    \end{proposition}
    \begin{proof}
        Let $f:\RS\to\C$ be meromorphic. Since $\RS$ is compact, it has finitely-many poles.\side[0.01in]{Otherwise, the set of poles would have a limit point, contradicting the discreteness of poles.} W.l.o.g., assume that $\infty$ is not a pole of $f$ (since we may consider $1/f$ instead). Now, for each pole $\lambda_i\in\C$ of $f$, consider its principle part
        \begin{equation*}
            h_i\!\l(z\r)=\sum_{\mathclap{j=-m_i}}^{-1}c_{ij}\l(z-\lambda_i\r)^j
        \end{equation*}
        for some $m_i>1$. Then the function $g\coloneqq f-\sum_ih_i$ is holomorphic function on $\RS$, and since $\RS$ is compact, it is constant by Proposition \ref{1.2:prp:holomorphic_compact_constant}. Thus $f=g+\sum_ih_i$, which is a rational function.\qed
    \end{proof}
    \begin{remark}
        Together with the above computation, this shows that if $f$ is a meromorphic function on $\RS$, then $\sum_{z\in\RS}\ord_z\!\l(f\r)=0$. As we shall see, this is a general fact for all compact Riemann surfaces.\exqed
    \end{remark}
    \subsection{Local Normal Form}
    \begin{theorem}[Local Normal Form]\label{1.2:thm:local_normal_form}
        Let $X$ and $Y$ be Riemann surfaces and let $F:X\to Y$ be a non-constant holomorphic map. Then, for every $p\in X$, there exists a unique $m\geq1$ such that for any chart $\tpl{U_2,\phi_2}$ of $Y$ centered at $F\l(p\r)$, there exists a chart $\tpl{U_1,\phi_1}$ of $X$ centered at $p$ such that $\phi_2\circ F\circ\phi_1^{-1}:z\mapsto z^m$ for all $z\in\phi_1\l(U_1\r)$.
    \end{theorem}
    \side[-0.53in]{This theorem also give easy proofs of some elementary properties of holomorphic maps, which we collect here; see \cite[][Section 1.2]{otto} for details. Throughout, $F:X\to Y$ is a non-constant holomorphic map between Riemann surfaces $X$ and $Y$.
        \begin{itemize}
            \item $F$ is an open map.
            \item If $F$ is injective, then it is biholomorphic onto its image.
            \item If $Y=\C$, then $\l|F\r|$ does not attain its maximum.
            \item If $X$ is compact, then $F$ is surjective and $Y$ is compact.
        \end{itemize}
        Together, the last two claims give an alternative proof for Proposition \ref{1.2:prp:holomorphic_compact_constant}.}\vspace{-0.05in}
    \begin{proof}
        Let $\tpl{U_2,\phi_2}$ be a chart of $Y$ centered at $F\l(p\r)$ and consider any chart $\tpl{V,\psi}$ of $X$ centered at $p$. Then the function $h\coloneqq\phi_2\circ F\circ\psi^{-1}$ is holomorphic, so it admits a power series representation $h\l(w\r)=\sum_{i=0}^{\infty}c_iw^i$ for all $w\in\psi\l(V\r)$. Note that $h\l(0\r)=\phi_2\l(F\l(p\r)\r)=0$, so $c_0=0$. Let $m\geq1$ be the smallest integer such that $c_m\neq0$, so
        \begin{equation*}
            h\l(w\r)=\sum_{i\geq m}c_iw^i=w^m\sum_{i\geq0}c_{i-m}w^i\eqqcolon w^mg\l(w\r).
        \end{equation*}
        Then $g$ is holomorphic at $0$ with $g\l(0\r)=c_m\neq0$, so there is a function $h$ holomorphic on some neighborhood $W$ of $0$ such that $\l(h\l(w\r)\r)^m=g\l(w\r)$ for all $w\in W$. Thus $h\l(w\r)=\l(wh\l(w\r)\r)^m$, so set $\eta\l(w\r)\coloneqq wh\l(w\r)$ for all $w\in W$. Note that $\eta'\l(0\r)=h\l(0\r)\neq0$, so $\eta$ is invertible on some neighborhood $W'\subseteq W$ of $0$. Set $U_1\coloneqq\psi^{-1}\l(W'\r)$ and $\phi_1\coloneqq\eta\circ\psi$. Then $\l(U_1,\phi_1\r)$ is a chart of $X$ centered at $p$ such that
        \begin{equation*}
            \l(\phi_2\circ F\circ\phi_1^{-1}\r)\l(z\r)=\l(\phi_2\circ F\circ\psi^{-1}\circ\eta^{-1}\r)\l(z\r)=h\l(\eta^{-1}\l(z\r)\r)=\l[\eta\l(\eta^{-1}\l(z\r)\r)\r]^m=z^m
        \end{equation*}
        for all $z\in\phi_1\l(U_1\r)$. To show uniqueness, it suffices to show that such an $m$ is chart-independent. But this is clear, for if a different chart $U_2'$ is chosen such that $F$ acts as $z\mapsto z^n$ for some neighborhood $U_1'$ of $p$, then $z^n=z^m$ on $\phi_1\l(U_1\r)\cap\phi_1'\l(U_1'\r)$ forces $n=m$.\qed
    \end{proof}
    \begin{definition}
        With the above notation, the unique $m\geq1$ such that there are local coordinates around $p$ and $F\l(p\r)$ where $F$ acts like $z\mapsto z^m$ is called the \uldef{multiplicity of $F$ at $p$}, denoted $\mult_p\!\l(F\r)$.
    \end{definition}
    \begin{remark}\side[-0.43in]{Consider the power function $f\l(z\r)\coloneqq z^m$ where $m\coloneqq\mult_p\!\l(F\r)$. Then, for all $z\in\C^*$, we see that $f^{-1}\l(z\r)$ has exactly $m$ elements given by the $m$ distinct $m^\textrm{th}$ roots of $z^m$. Thus the map $f$ causes $\C$ to `cover itself $m$ times', and those coverings meet at the fixed point $0$. But $f^{-1}\l(0\r)=\l\{0\r\}$ has only $1$ element, which prevents $f$ to be a $n$-sheeted covering of $\C$. To remedy this, we count $0$ \textit{with multiplicity $m$}; see Chapter \ref{cpt:covering_spaces} for a more formal discussion. Since $F$ is locally represented by $f$, and $\tpl{U_1,\phi_1}$ is centered at $p$, we see that $m$ counts the multiplicity at which neighbors of $p$ are mapped to $F\l(p\r)$.}
        We give a simple way of computing $\mult_p\!\l(F\r)$ that does not involve casting $F$ into Local Normal Form, or even having to find local coordinates centered at $p$ and $F\l(p\r)$. Indeed, let $\tpl{U_1,\phi_1}$ and $\tpl{U_2,\phi_2}$ be charts around $p$ and $F\l(p\r)$, say with $z_0\coloneqq\phi_1\l(p\r)$ and $w_0\coloneqq\phi_2\l(F\l(p\r)\r)$. Letting $f\coloneqq\phi_2\circ F\circ\phi_1^{-1}$, we see that $f\l(z_0\r)=w_0$ and hence its power series representation has the form
        \begin{equation*}
            f\l(z\r)=f\l(z_0\r)+\sum_{i\geq m}c_i\l(z-z_0\r)^i
        \end{equation*}
        for some $m\geq1$ with $c_m\neq0$. Then, since $z-z_0$ and $w-w_0=f\l(z\r)-f\l(z_0\r)$ are local coordinates centered at $p$ and $F\l(p\r)$, respectively, we see from the above proof that $\mult_p\!\l(F\r)=m$. Thus to compute $\mult_p\!\l(F\r)$, it suffices to case $F$ into local coordinates $\l(U_1,\phi_1\r)$ around $p$ and $\l(U_2,\phi_2\r)$ around $F\l(p\r)$ and find the lowest non-zero power of the Taylor series of $f\coloneqq\phi_2\circ F\circ\phi_1^{-1}$.\exqed
    \end{remark}
    \begin{proposition}\label{1.2:prp:multiplicity_and_order}
        Let $f$ be a meromorphic function on a Riemann surface $X$ and let $F:X\to\RS$ be its associated holomorphic map. Fix $p\in X$.
        \begin{itemize}
            \item If $p$ is a not a pole of $f$, then $\mult_p\!\l(F\r)=-\ord_p\!\l(f-f\l(p\r)\r)$.
            \vspace{-0.05in}
            \item If $p$ is a pole of $f$, then $\mult_p\!\l(F\r)=\ord_p\!\l(f\r)$.
        \end{itemize}
    \end{proposition}
    \begin{proof}
        Suppose that $p$ is not a pole of $f$, so $f\l(p\r)=F\l(p\r)\in\C$. Since the set of all poles of a meromorphic function forms a discrete set, let $p\in U\subseteq X$ be small enough so that $\l.f\r|_U$ is holomorphic. Let $\tpl{U,\phi}$ be a chart of $X$ and consider the chart $\l(\C,\psi\r)$ of $\RS$ around $F\l(p\r)$ defined by $\psi\l(z\r)\coloneqq z-F\l(p\r)$. Then $f-f\l(p\r)=\psi\circ F$ on $U$, so
        \begin{equation*}
            \l(f-f\l(p\r)\r)_\phi\!\coloneqq\l(f-f\l(p\r)\r)\circ\phi^{-1}=\psi\circ F\circ\phi^{-1}
        \end{equation*}
        on $\phi\l(U\r)$. Expanding in power series around $z_0\coloneqq\phi\l(p\r)\in\phi\l(U\r)$, we see that
        \begin{equation*}
            \l(\psi\circ F\circ\phi^{-1}\r)\l(z\r)=\l(f-f\l(p\r)\r)_\phi\!\l(z\r)=\sum_{i\geq m}c_i\l(z-z_0\r)^i
        \end{equation*}
        for some $m\in\N$ with $c_m\neq0$. Note that $\l(f-f\l(p\r)\r)_\phi\!\l(z_0\r)=\l(f-f\l(p\r)\r)\l(p\r)=0$, so $m>0$ and hence $\mult_p\!\l(F\r)=m$. But $m$ is also the smallest integer such that
        \begin{equation*}
            0\neq\l(z-z_0\r)^{-m}\!\l(f-f\l(p\r)\r)_\phi\!\l(z\r)\in\mc{O}\l(U\r),
        \end{equation*}
        so $\ord_p\!\l(f-f\l(p\r)\r)=-m$. Suppose now that $p$ is a pole of $f$, so $F\l(p\r)=\infty$. Since $\lim_{z\to p}1/f\l(z\r)=0$, we may let $p\in U\subseteq X$ be small enough so that the function $\tilde{f}:U\to\C$ defined by
        \begin{equation*}
            \tilde{f}\l(x\r)\coloneqq
            \begin{dcases}
                0 & \textrm{if }x=p \\
                1/f\l(x\r) & \textrm{else}
            \end{dcases}
        \end{equation*}
        is holomorphic. Let $\tpl{U,\phi}$ be a chart of $X$ and consider the chart $\tpl{\RS\comp\l\{0\r\},\psi}$ of $\RS$ defined by\side{
        \begin{equation*}
            \psi\l(z\r)\coloneqq
            \begin{dcases}
                0 & \textrm{if }z=\infty \\
                1/z & \textrm{else.}
            \end{dcases}
        \end{equation*}
        } $\psi\l(z\r)\coloneqq1/z$. Then $\tilde{f}=\psi\circ F$ on $U$, so $\tilde{f}_\phi\!\coloneqq\tilde{f}\circ\phi^{-1}=\psi\circ F\circ\phi^{-1}$ on $\phi\l(U\r)$. By the same argument as above, we see that $\mult_p\!\l(F\r)=-\ord_p(\tilde{f})$. Now $\ord_p\l(f\r)=-\ord_p(\tilde{f})$, so the result follows.\qed
    \end{proof}
\end{document}
