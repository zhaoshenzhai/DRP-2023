\documentclass[../Moduli_Spaces_of_Riemann_Surfaces.tex]{subfiles}
\begin{document}
    \section{Liftings of Curves}\label{sec:liftings_of_curves}
    \begin{definition}
        \textit{4.13: Curve lifting property.}
    \end{definition}
    \begin{theorem}
        \textit{4.14: Every covering map has the curve lifting property.}
    \end{theorem}
    \begin{theorem}
        \textit{4.16: Preimages have the same cardinality.}
    \end{theorem}
    \begin{theorem}
        \textit{4.17: Existence of lifting.}
    \end{theorem}
    \begin{theorem}
        \textit{4.19: Curve lifting property implies covering map.}
    \end{theorem}
    \subsection{Degree of Proper Holomorphic Maps}\label{sec:degree_proper_holomorphic_map}
    \begin{defthm}
        Let $X$ and $Y$ be compact Riemann surfaces and let $F:X\to Y$ be a non-constant holomorphic map. For each $y\in Y$, define the number
        \begin{equation*}
            d_y\!\l(F\r)\coloneqq\sum_{p\in F^{-1}\l(y\r)}\mult_p\!\l(F\r).
        \end{equation*}
        Then $d_y\!\l(F\r)\in\Z$ is independent of $y$, and we define the \uldef{degree of $F$} as $\deg F\coloneqq d_y\!\l(F\r)$ for any $y\in Y$.
    \end{defthm}
    \begin{proof}
        
    \end{proof}
\end{document}
