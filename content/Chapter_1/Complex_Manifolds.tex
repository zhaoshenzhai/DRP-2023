\chapter{Complex Manifolds}
\section{Charts and Atlases}
We assume that the reader is familiar with the basic notions of real manifolds. The case for complex manifolds is similar, so our exposition will be brief.
\begin{definition}
    Let $M$ be a second-countable Hausdorff space. A \uldef{$d$-dimensional complex} \uldef{chart on $M$} is a pair $\tpl{U,\phi}$ where $\phi:U\to V$ is a homeomorphism from an open subset $U\subseteq M$ onto an open subset $V\subseteq\C^d$ for some $d$. Two $d$-dimensional charts $\l(U_1,\phi_1\r)$ and $\l(U_2,\phi_2\r)$ are said to be \uldef{holomorphically compatible} if either $U_1\cap U_2=\em$, or the map
    \begin{equation*}
        \phi_2\circ\phi_1^{-1}:\phi_1\l(U_1\cap U_2\r)\to\phi_2\l(U_1\cap U_2\r)
    \end{equation*}
    is biholomorphic. A \uldef{$d$-dimensional complex atlas on $M$} is a collection $\ms{A}\coloneqq\l\{\l(U_i,\phi_i\r)\r\}_{i\in I}$ of $d$-dimensional complex charts such that every two charts $\l(U_i,\phi_i\r)$ and $\l(U_j,\phi_j\r)$ are holomorphically compatible and $M=\bigcup_{i\in I}U_i$.
\end{definition}
\side{Charts provide us a way of making $M$ $\textrm{`}$look like$\textrm{'}$ an open set in $\C^d$. Indeed, they provide local coordinates for every point in $M$ in such a way that the $\textrm{`}$change of coordinates$\textrm{'}$ map $\phi_2\circ\phi_1^{-1}$ ensures that local notions of functions in $\C^d$ are well-defined on $M$ too.
    \begin{equation*}
        \begin{tikzcd}[ampersand replacement=\&, column sep=0.1in]
            \& U_1\cap U_2 \ar[dl, "\phi_1"'] \ar[dr, "\phi_2"] \\
            \phi_1(U_1\cap U_2) \ar[rr, "\phi_2\circ\phi_1^{-1}"'] \& \& \phi_2(U_1\cap U_2)
        \end{tikzcd}       
    \end{equation*}
It is clear that one only needs $\phi_2\circ\phi_1^{-1}$ to be holomorphic for it to be biholomorphic.}{-1.41in}
\begin{remark}
    Two atlases $\ms{A}$ and $\ms{B}$ on a manifold $M$ are said to be \ul{analytically equivalent} if every chart in $\ms{A}$ is compatible with every chart in $\ms{B}$. By Zorn's Lemma, every atlas $\ms{A}$ of a manifold $M$ is contained in a unique maximal atlas $\mf{U}$ on $M$. Moreover, two atlases are equivalent iff they are contained in the same maximal atlas, which justifies the following definition.\exqed
\end{remark}
\begin{definition}
    Let $M$ be a second-countable Hausdorff space. A \uldef{$d$-dimensional complex} \uldef{structure} on $M$ is a $d$-dimensional maximal atlas $\mf{U}$ on $M$, or, equivalently, an equivalence class of $d$-dimensional complex atlases on $M$. The pair $\tpl{M,\mf{U}}$ is then called a \uldef{$d$-dimensional} \uldef{complex manifold}.
\end{definition}
\side{To give a complex structure $\mf{U}$ to $M$, it suffices to give $M$ a complex atlas since it extends to a \textit{unique} complex structure.}{-0.61in}
\begin{definition}
    A \uldef{Riemann surface} is a connected $1$-dimensional complex manifold.
\end{definition}
\begin{example}\side{Every Riemann surface can be regarded as a (connected) $2$-dimensional real manifold by $\textrm{`}$forgetting$\textrm{'}$ its complex structure; indeed all holomorphic maps are real $\mc{C}^\infty$ functions.}{-0.22in}
    Some elementary examples of Riemann surfaces.
    \begin{itemize}
        \item The complex plane $\C$, equipped with its standard topology, can be given a complex structure $\mf{U}$ by choosing the atlas containing a single chart $\tpl{\C,\id_\C}$. We may, however, also give $\C$ a different complex structure $\mf{U}'$ by choosing the chart map $\phi:z\mapsto\bar{z}$ instead. Indeed, $\mf{U}\neq\mf{U}'$ since the map $\phi\circ\id_\C^{-1}=\phi$ is not holomorphic and hence the atlases $\l\{\tpl{\C,\id_\C}\r\}$ and $\l\{\tpl{\C,\phi}\r\}$ are not equivalent. This example generalizes to any domain $D\subseteq\C$.
        \item Let $D\subseteq\C$ be a domain and consider any holomorphic function $f:D\to\C$. Then the graph $\Gamma\!_f\coloneqq\l\{\tpl{z,f\l(z\r)}\mid z\in D\r\}$, equipped with the subspace topology inherited from $\C^2$, can be given a complex structure by choosing the chart map $\pi:\Gamma\!_f\to D:\tpl{z,f\l(z\r)}\mapsto z$.\exqed
    \end{itemize}
\end{example}
\subsection{The Riemann Sphere}
A particularly important Riemann surface is the Riemann sphere $\RS$, which admits several constructions.\side{The fact that they are all holomorphic (in the sense of Definition $\ref{1:def:holomorphic_maps}$) is non-trivial, and the first goal of this paper will be to show that every Riemann surface homeomorphic to $\RS$ is in fact holomorphic to $\RS$.}{-0.18in} Here, we give three constructions.
\begin{example}[One-point Compactification of $\C$]
    Let $\infty$ be a symbol not belonging to $\C$ and set $\RS\coloneqq\C\cup\l\{\infty\r\}$. We declare a set $U\subseteq\RS$ to be open if either $U\subseteq\C$ is open or $U=K^c\cup\l\{\infty\r\}$ where $K\subseteq\C$ is compact.\side{This makes $\RS$, equipped with the collection $\mc{T}$ of all such open sets, a second-countable Hausdorff space. Indeed, the fact that $\mc{T}$ is a topology on $\RS$ follows from De Morgan's Laws and the Heine-Borel Theorem. It is trivially Hausdorff, and it is second-countable since we may append, to any countable basis for the standard topology of $\C$, the countable collection $\l\{B_r\l(0\r)^c\cup\l\{\infty\r\}\r\}_{r\in\Q^+}$.}{-0.08in} We employ two charts
    \begin{equation*}
        \begin{alignedat}{2}
            U_1&\coloneqq\RS\comp\l\{\infty\r\}=\C\ \ \ \ \ \ \ \ \ \ \ \ \ \ \ \ \ \ \ \ &&\phi_1:U_1\to\C:z\mapsto z\ \ \ \ \l(\phi_1\coloneqq\id_\C\r) \\
            U_2&\coloneqq\RS\comp\l\{0\r\}=\C^\ast\cup\l\{\infty\r\}&&\phi_2:U_2\to\C:z\mapsto
            \begin{dcases}
                1/z & \textrm{if }z\in\C^\ast \\
                0 & \textrm{else.}
            \end{dcases}
        \end{alignedat}
    \end{equation*}
    Clearly $\phi_1$ is a homeomorphism. Since $\phi_2$ is invertible with $\phi_2^{-1}\!\l(z\r)\coloneqq1/z$ for all $z\in\C^\ast$ and $\phi_2^{-1}\!\l(0\r)\coloneqq\infty$, and
    \begin{equation*}
        \lim\limits_{z\to\infty}\phi_2\!\l(z\r)=0=\phi_2\!\l(\infty\r)\ \ \ \ \ \ \ \ \textrm{and}\ \ \ \ \ \ \ \ \lim\limits_{z\to0}\phi_2^{-1}\!\l(z\r)=\infty=\phi_2^{-1}\l(0\r),
    \end{equation*}
    we see that $\phi_2$ is a homeomorphism too. Furthermore,
    \begin{equation*}
        \phi_2\circ\phi_1^{-1}:\C^\ast\to\C^\ast:z\mapsto\frac{1}{z}
    \end{equation*}
    is holomorphic, so the atlas $\l\{\tpl{U_1,\phi_1},\tpl{U_2,\phi_2}\r\}$ defines a complex structure on $\RS$.\exqed
\end{example}
\begin{example}[Stereographic Projection]
    Consider the unit sphere $\RS\coloneqq S^2\subseteq\R^3$ as a topological subspace of $\R^3$, which makes it a second-countable Hausdorff space. Identifying the plane $w=0$ as $\C$, we employ the charts
    \begin{equation*}
        \begin{alignedat}{2}
            U_1&\coloneqq S^2\comp\l\{\l(0,0,1\r)\r\}\ \ \ \ \ \ \ \ \ \ \ \ \ \ \ \ &&\phi_1:U_1\to\C:\tpl{x,y,w}\mapsto\frac{x+iy}{1-w} \\
            U_2&\coloneqq S^2\comp\l\{\l(0,0,-1\r)\r\}&&\phi_2:U_2\to\C:\tpl{x,y,w}\mapsto\frac{x-iy}{1+w}.
        \end{alignedat}
    \end{equation*}
    Clearly $\phi_1$ and $\phi_2$ are continuous, and it can be verified that they are invertible with continuous inverses
    \begin{equation*}
        \phi_1^{-1}\!\l(z\r)\coloneqq\tpl{\frac{2\Re z}{\l|z\r|^2+1},\frac{2\Im z}{\l|z\r|^2+1},\frac{\l|z\r|^2-1}{\l|z\r|^2+1}}\ \ \ \ \textrm{and}\ \ \ \ \phi_2^{-1}\!\l(z\r)\coloneqq\tpl{\frac{2\Re z}{\l|z\r|^2+1},\frac{-2\Im z}{\l|z\r|^2+1},\frac{1-\l|z\r|^2}{\l|z\r|^2+1}}.
    \end{equation*}
    Observe that $U_1\cap U_2=S^2\comp\l\{\tpl{0,0,\pm1}\r\}$, so $\phi_2\circ\phi_1^{-1}:\C^\ast\to\C^\ast:z\mapsto1/z$, which is holomorphic. Thus the atlas $\l\{\tpl{U_1,\phi_1},\tpl{U_2,\phi_2}\r\}$ defines a complex structure on $\RS$.\exqed
\end{example}
\begin{example}[Complex Projective Line]
    
\end{example}
\subsection{Complex Tori}
\section{Holomorphic Maps}
\begin{definition}\label{1:def:holomorphic_maps}

\end{definition}
\subsection{Holomorphic functions on $\RS$}
