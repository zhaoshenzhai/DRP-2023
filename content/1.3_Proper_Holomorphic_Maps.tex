\documentclass[../Moduli_Spaces_of_Riemann_Surfaces.tex]{subfiles}
\begin{document}
    \section{Proper Holomorphic Maps}
    \subsection{Local Normal Form}
    \begin{theorem}[Local Normal Form]
        Let $X$ and $Y$ be Riemann surfaces and let $F:X\to Y$ be a non-constant holomorphic map. Then, for every $p\in X$, there exists a unique $m\geq1$ such that for any chart $\tpl{U_2,\phi_2}$ of $Y$ centered at $F\l(p\r)$, there exists a chart $\tpl{U_1,\phi_1}$ of $X$ centered at $p$ such that $\phi_2\circ F\circ\phi_1^{-1}:z\mapsto z^m$ for all $z\in\phi_1\l(U_1\r)$.
    \end{theorem}
    \side[-0.53in]{This theorem also give easy proofs of some elementary properties of holomorphic maps, which we collect here; see \cite[][section 1.2]{otto} for details. Throughout, $F:X\to Y$ is a non-constant holomorphic map between Riemann surfaces $X$ and $Y$.
        \begin{itemize}
            \item $F$ is an open map.
            \item If $F$ is injective, then it is biholomorphic onto its image.
            \item If $Y=\C$, then $\l|F\r|$ does not attain its maximum.
            \item If $X$ is compact, then $F$ is surjective and $Y$ is compact.
        \end{itemize}
        Together, the last two claims give an alternative proof for Theorem \ref{1.2:thm:holomorphic_compact_constant}.}\vspace{-0.05in}
    \begin{proof}
        Let $\tpl{U_2,\phi_2}$ be a chart of $Y$ centered at $F\l(p\r)$ and consider any chart $\tpl{V,\psi}$ of $X$ centered at $p$. Then the function $h\coloneqq\phi_2\circ F\circ\psi^{-1}$ is holomorphic, so it admits a power series representation $h\l(w\r)=\sum_{i=0}^{\infty}c_iw^i$ for all $w\in\psi\l(V\r)$. Note that $h\l(0\r)=\phi_2\l(F\l(p\r)\r)=0$, so $c_0=0$. Let $m\geq1$ be the smallest integer such that $c_m\neq0$, so
        \begin{equation*}
            h\l(w\r)=\sum_{i\geq m}c_iw^i=w^m\sum_{i\geq0}c_{i-m}w^i\eqqcolon w^mg\l(w\r).
        \end{equation*}
        Then $g$ is holomorphic at $0$ with $g\l(0\r)=c_m\neq0$, so there is a function $h$ holomorphic on some neighborhood $W$ of $0$ such that $\l(h\l(w\r)\r)^m=g\l(w\r)$ for all $w\in W$. Thus $h\l(w\r)=\l(wh\l(w\r)\r)^m$, so set $\eta\l(w\r)\coloneqq wh\l(w\r)$ for all $w\in W$. Note that $\eta'\l(0\r)=h\l(0\r)\neq0$, so $\eta$ is invertible on some neighborhood $W'\subseteq W$ of $0$. Set $U_1\coloneqq\psi^{-1}\l(W'\r)$ and $\phi_1\coloneqq\eta\circ\psi$. Then $\l(U_1,\phi_1\r)$ is a chart of $X$ centered at $p$ such that
        \begin{equation*}
            \l(\phi_2\circ F\circ\phi_1^{-1}\r)\l(z\r)=\l(\phi_2\circ F\circ\psi^{-1}\circ\eta^{-1}\r)\l(z\r)=h\l(\eta^{-1}\l(z\r)\r)=\l[\eta\l(\eta^{-1}\l(z\r)\r)\r]^m=z^m
        \end{equation*}
        for all $z\in\phi_1\l(U_1\r)$. To show uniqueness, it suffices to show that such an $m$ is chart-independent. But this is clear, for if a different chart $U_2'$ is chosen such that $F$ acts as $z\mapsto z^n$ for some neighborhood $U_1'$ of $p$, then $z^n=z^m$ on $\phi_1\l(U_1\r)\cap\phi_1'\l(U_1'\r)$ forces $n=m$.\qed
    \end{proof}
    \begin{definition}
        With the above notation, the unique $m\geq1$ such that there are local coordinates around $p$ and $F\l(p\r)$ where $F$ acts like $z\mapsto z^m$ is called the \uldef{multiplicity of $F$ at $p$}, denoted $\mult_p\!\l(F\r)$.
    \end{definition}
    \begin{remark}\side[-0.43in]{Consider the power function $h\l(z\r)\coloneqq z^m$ where $m\coloneqq\mult_p\!\l(F\r)$. Then, for all $z\in\C^*$, we see that $h^{-1}\l(z\r)$ has exactly $m$ elements given by the $m$ distinct $m^\textrm{th}$ roots of $z^m$. Thus the map $h$ causes $\C$ to `cover itself $m$ times', and those coverings meet at the fixed point $0$. But $h^{-1}\l(0\r)=\l\{0\r\}$ has only $1$ element, which prevents $h$ to be a $n$-sheeted covering of $\C$. To remedy this, we count $0$ \textit{with multiplicity $m$}; see Example \ref{1.3:exa:covering_map_power} for a more formal discussion. Since $F$ is locally represented by $h$, and $\tpl{U_1,\phi_1}$ is centered at $p$, we see that $m$ counts the multiplicity at which neighbors of $p$ are mapped to $F\l(p\r)$.}
        We give a simple way of computing $\mult_p\!\l(F\r)$ that does not involve casting $F$ into Local Normal Form, or even having to find local coordinates centered at $p$ and $F\l(p\r)$. Indeed, let $\tpl{U_1,\phi_1}$ and $\tpl{U_2,\phi_2}$ be charts around $p$ and $F\l(p\r)$, say with $z_0\coloneqq\phi_1\l(p\r)$ and $w_0\coloneqq\phi_2\l(F\l(p\r)\r)$. Letting $f\coloneqq\phi_2\circ F\circ\phi_1^{-1}$, we see that $f\l(z_0\r)=w_0$ and hence its power series representation has the form
        \begin{equation*}
            f\l(z\r)=f\l(z_0\r)+\sum_{i\geq m}c_i\l(z-z_0\r)^i
        \end{equation*}
        for some $m\geq1$ with $c_m\neq0$. Then, since $z-z_0$ and $w-w_0=f\l(z\r)-f\l(z_0\r)$ are local coordinates centered at $p$ and $F\l(p\r)$, respectively, we see from the above proof that $\mult_p\!\l(F\r)=m$. Thus to compute $\mult_p\!\l(F\r)$, it suffices to case $F$ into local coordinates $\l(U_1,\phi_1\r)$ around $p$ and $\l(U_2,\phi_2\r)$ around $F\l(p\r)$ and find the lowest non-zero power of the Taylor series of $f\coloneqq\phi_2\circ F\circ\phi_1^{-1}$.\exqed
    \end{remark}
    \begin{theorem}
        Let $f$ be a meromorphic function on a Riemann surface $X$ and let $F:X\to\RS$ be its associated holomorphic map. Fix $p\in X$.
        \begin{itemize}
            \item If $p$ is a not a pole of $f$, then $\mult_p\!\l(F\r)=-\ord_p\!\l(f-f\l(p\r)\r)$.
            \vspace{-0.05in}
            \item If $p$ is a pole of $f$, then $\mult_p\!\l(F\r)=\ord_p\!\l(f\r)$.
        \end{itemize}
    \end{theorem}
    \begin{proof}
        Suppose that $p$ is not a pole of $f$, so $f\l(p\r)=F\l(p\r)\in\C$. Since the set of all poles of a meromorphic function forms a discrete set, let $p\in U\subseteq X$ be small enough so that $\l.f\r|_U$ is holomorphic. Let $\tpl{U,\phi}$ be a chart of $X$ and consider the chart $\l(\C,\psi\r)$ of $\RS$ around $F\l(p\r)$ defined by $\psi\l(z\r)\coloneqq z-F\l(p\r)$. Then $f-f\l(p\r)=\psi\circ F$ on $U$, so
        \begin{equation*}
            \l(f-f\l(p\r)\r)_\phi\!\coloneqq\l(f-f\l(p\r)\r)\circ\phi^{-1}=\psi\circ F\circ\phi^{-1}
        \end{equation*}
        on $\phi\l(U\r)$. Expanding in power series around $z_0\coloneqq\phi\l(p\r)\in\phi\l(U\r)$, we see that
        \begin{equation*}
            \l(\psi\circ F\circ\phi^{-1}\r)\l(z\r)=\l(f-f\l(p\r)\r)_\phi\!\l(z\r)=\sum_{i\geq m}c_i\l(z-z_0\r)^i
        \end{equation*}
        for some $m\in\N$ with $c_m\neq0$. Note that $\l(f-f\l(p\r)\r)_\phi\!\l(z_0\r)=\l(f-f\l(p\r)\r)\l(p\r)=0$, so $m>0$ and hence $\mult_p\!\l(F\r)=m$. But $m$ is also the smallest integer such that
        \begin{equation*}
            0\neq\l(z-z_0\r)^{-m}\!\l(f-f\l(p\r)\r)_\phi\!\l(z\r)\in\mc{O}\l(U\r),
        \end{equation*}
        so $\ord_p\!\l(f-f\l(p\r)\r)=-m$.\\\ \\
        Suppose now that $p$ is a pole of $f$, so $F\l(p\r)=\infty$. Since $\lim_{z\to p}1/f\l(z\r)=0$, we may let $p\in U\subseteq X$ be small enough so that the function $\tilde{f}:U\to\C$ defined by
        \begin{equation*}
            \tilde{f}\l(x\r)\coloneqq
            \begin{dcases}
                0 & \textrm{if }x=p \\
                1/f\l(x\r) & \textrm{else}
            \end{dcases}
        \end{equation*}
        is holomorphic. Let $\tpl{U,\phi}$ be a chart of $X$ and consider the chart $\tpl{\RS\comp\l\{0\r\},\psi}$ of $\RS$ defined by\side{
        \begin{equation*}
            \psi\l(z\r)\coloneqq
            \begin{dcases}
                0 & \textrm{if }z=\infty \\
                1/z & \textrm{else.}
            \end{dcases}
        \end{equation*}
        } $\psi\l(z\r)\coloneqq1/z$. Then $\tilde{f}=\psi\circ F$ on $U$, so $\tilde{f}_\phi\!\coloneqq\tilde{f}\circ\phi^{-1}=\psi\circ F\circ\phi^{-1}$ on $\phi\l(U\r)$. By the same argument as above, we see that $\mult_p\!\l(F\r)=-\ord_p(\tilde{f})$. Now $\ord_p\l(f\r)=-\ord_p(\tilde{f})$, so the result follows.\qed
    \end{proof}
    \subsection{Covering Maps}
    \begin{definition}
        Let $X$ and $Y$ be topological spaces. A map $f:X\to Y$ is said to be a \uldef{covering map} if every point $y\in Y$ has a neighborhood $V$ such that $f^{-1}\l(V\r)=\bigcup_{j\in J}U_j$ where $U_j$ are disjoint open sets in $X$, each homeomorphic to $V$ via $\l.f\r|_{U_j}$.
    \end{definition}
    \begin{example}\label{1.3:exa:covering_map_power}
        Let $m\geq2$ be a natural number and consider the power map $h:\C^\ast\to\C^\ast$ mapping $z\mapsto z^m$. We claim that $h$ is a covering map, so take $b\in\C^\ast$ and let $a\in\C^\ast$ be any one of its $m^\textrm{th}$ roots. Since $h$ is a local homeomorphism, there exist neighborhoods $U_0$ of $a$ and $V$ of $b$ such that $\l.h\r|_{U_0}:U_0\to V$ is a homeomorphism. It is clear then that\side{Indeed, for all $c\in h^{-1}\l(V\r)$, $h\l(c\r)\in V$ and so there exists some $a'\in U_0$ such that $h\l(a'\r)=h\l(c\r)$. Then $c=\omega^ja'$ for some $0\leq j\leq m-1$, so $c\in\omega^jU_0$. Conversely, if $\in\omega^jU_0$ for some $0\leq j\leq m-1$, then $c=\omega^ja'$ for some $a'\in U_0$ and hence $h\l(c\r)=h\l(\omega^ja'\r)=h\l(a'\r)\in V$.}
        \vspace{-0.1in}
        \begin{equation*}
            h^{-1}\l(V\r)=\bigcup_{j=0}^{m-1}\omega^jU_0,
        \end{equation*}
        \pagebreak
        where $\omega$ is an $m^\textrm{th}$ root of unity, and since $h^{-1}\l(b\r)$ is discrete, the sets $U_j\coloneqq\omega^jU_0$ can be made small enough so that they are pairwise disjoint. Then each $\l.h\r|_{U_j}:U_j\to V$ is a homeomorphism, as desired.\exqed
    \end{example}
    \subsection{Proper Maps}
    \subsection{Degree of Proper Holomorphic Maps}
    \begin{defprop}
        Let $X$ and $Y$ be compact Riemann surfaces and let $F:X\to Y$ be a non-constant holomorphic map. For each $y\in Y$, define the number
        \begin{equation*}
            d_y\!\l(F\r)\coloneqq\sum_{p\in F^{-1}\l(y\r)}\mult_p\!\l(F\r).
        \end{equation*}
        Then $d_y\!\l(F\r)\in\Z$ is independent of $y$, and we define the \uldef{degree of $F$} as $\deg F\coloneqq d_y\!\l(F\r)$ for any $y\in Y$.
    \end{defprop}
    \begin{proof}
        
    \end{proof}
\end{document}
