\documentclass[../Moduli_Spaces_of_Riemann_Surfaces.tex]{subfiles}
\begin{document}
    \section{Analytic Continuation}
    We now restrict to when $X$ is a Riemann surface with a fixed point $p\in X$. For convenience, we write\side[0.01in]{Thus if $U\subseteq X$ is an open set containing $p$, then any $f\in\HOLO_p\!\l[p\r]\l(U\r)$ has at most a single simple pole. Otherwise, if $p\not\in U$, then $f$ is holomorphic.} $\HOLO\l[p\r]\coloneqq\HOLO\l[D\r]$ where $D$ is the divisor on $X$ defined by $D\l(p\r)\coloneqq1$ and zero everywhere else. Throughout, $\alpha:\l[0,1\r]\to X$ is a curve with $p=\alpha\l(0\r)$ and $q\coloneqq\alpha\l(1\r)\neq p$, and $\phi_0\in\HOLO_p\!\l[p\r]$ is a fixed germ.
    \subsection{Analytic Continuation of Germs along Curves}
    \begin{definition}
        A germ $\psi\in\HOLO_q$ is said to be the \uldef{analytic continuation of $\phi_0$ along $\alpha$} if there exist a family $\phi_t\in\HOLO_{\alpha\l(t\r)}\!\l[p\r]$ of germs for all $t\in\l[0,1\r]$ with $\psi=\phi_1$ such that for all $\tau\in\l[0,1\r]$, there exists a neighborhood $T\subseteq\l[0,1\r]$ of $\tau$, an open set $U\subseteq X$ with $\alpha\l(T\r)\subseteq U$, and a function $f\in\HOLO\l[p\r]\l(U\r)$ such that $\l[f\r]_{\alpha\l(t\r)}=\phi_t$ for all $t\in T$.
    \end{definition}
    \begin{proposition}
        A germ $\psi\in\HOLO_q$ is an analytic continuation of $\phi_0$ along $\alpha$ iff there exists a lifting $\tilde{\alpha}:\l[0,1\r]\to\l|\HOLO\l[p\r]\r|$ such that $\tilde{\alpha}\l(0\r)=\phi_0$ and $\tilde{\alpha}\l(1\r)=\psi$.
    \end{proposition}
    \begin{proof}
        If\side[-0.42in]{In particular, the uniqueness of liftings shows that if an analytic continuation of $\phi_0$ along $\alpha$ exists, then it is unique.} $\psi\in\HOLO_q$ is an analytic continuation of $\phi_0$ along $\alpha$, let $\l\{\phi_t\r\}$ be a family of germs as defined above. We claim that the curve $\tilde{\alpha}:\l[0,1\r]\to\l|\HOLO\l[p\r]\r|$ mapping $t\mapsto\phi_t$ is a lifting of $\alpha$.
        \begin{itemize}
            \item First, note that $\phi_t\in\HOLO_{\alpha\l(t\r)}\!\l[p\r]$ for all $t\in\l[0,1\r]$, so\side[-0.22in]{Clearly $\tilde{\alpha}\l(0\r)=\phi_0$ and $\tilde{\alpha}\l(1\r)=\psi$.
                \begin{equation*}
                    \begin{tikzcd}[ampersand replacement=\&]
                        \& \l|\HOLO\l[p\r]\r| \ar[d, "\pi"] \\
                        \l[0,1\r] \ar[r, "\alpha"'] \ar[ur, "\tilde{\alpha}"] \& X
                    \end{tikzcd}
                \end{equation*}
                } $\l(\pi\circ\tilde{\alpha}\r)\l(t\r)=\pi\l(\tilde{\alpha}\l(t\r)\r)=\pi\l(\phi_t\r)=\alpha\l(t\r)$. It remains to show that $\tilde{\alpha}$ is continuous, so fix a basis element $\l[U,f\r]\subseteq\l|\HOLO\l[p\r]\r|$ and take $\tau\in\tilde{\alpha}^{-1}\!\l(\l[U,f\r]\r)$. Then $\tau\in\l[0,1\r]$, so there exists a neighborhood $T\subseteq\l[0,1\r]$ of $\tau$ such that $\l[f\r]_{\alpha\l(t\r)}=\phi_t$ for all $t\in T$. Observe that $\tilde{\alpha}\l(T\r)\subseteq\l[U,f\r]$ since for all $\phi_t\in\tilde{\alpha}\l(T\r)$, we have $\alpha\l(t\r)\in U$ and hence $\phi_t=\l[f\r]_{\alpha\l(t\r)}\in\l[U,f\r]$.
        \end{itemize}
        Conversely, suppose that there is a lifting $\tilde{\alpha}:\l[0,1\r]\to\l|\HOLO\l[p\r]\r|$ of $\alpha$ with $\tilde{\alpha}\l(0\r)=\phi_0$ and $\tilde{\alpha}\l(1\r)=\psi$. For all $t\in\l[0,1\r]$, we define $\phi_t\coloneqq\tilde{\alpha}\l(t\r)$, so $\phi_1=\psi$. Fix $\tau\in\l[0,1\r]$, so there exists a basis neighborhood $\l[U,f\r]\subseteq\l|\HOLO\l[p\r]\r|$ of $\tilde{\alpha}\l(\tau\r)$. But $\tilde{\alpha}$ is continuous, so there exists a neighborhood $T\subseteq\l[0,1\r]$ of $\tau$ such that $\tilde{\alpha}\l(T\r)\subseteq\l[U,f\r]$. Projecting, we see that $\alpha\l(T\r)\subseteq\pi\l(\l[U,f\r]\r)=U$. Finally, the commutativity of the diagram gives $\l[f\r]_{\alpha\l(t\r)}\!=\phi_t$ for all $t\in T$, so $\psi$ is an analytic continuation of $\phi_0$ along $\alpha$.\qed
    \end{proof}
    \begin{corollary}[Monodromy Theorem]
        Let $\alpha_0,\alpha_1:\l[0,1\r]\to X$ be homotopic curves from $p$ to $q$. If the germ $\phi_0\in\HOLO_p\!\l[p\r]$ admits an analytic continuation along every deformation of $\alpha_0$ to $\alpha_1$, then the analytic continuations of $\phi_0$ along $\alpha_0$ and $\alpha_1$ coincide.
    \end{corollary}
    \begin{proof}
        By Propositions \ref{2.2:prp:basis_stalk} and \ref{2.2:prp:stalk_topology_Hausdorff}, $\l|\HOLO\l[p\r]\r|$ is Hausdorff whose projection $\pi:\l|\HOLO\l[p\r]\r|\to X$ is a local homeomorphism. Since each deformation admits a lifting starting at $\phi_0$, the\side{This is a standard result in algebraic topology. For a proof, see \cite[][Proposition 4.10]{forster}.} liftings $\tilde{\alpha}_0$ and $\tilde{\alpha}_1$ have the same endpoints; that is, the analytic continuations along $\alpha_0$ and $\alpha_1$ coincide.\qed
    \end{proof}
    \begin{corollary}
        Suppose $X$ is simply-connected. If the germ $\phi_0\in\HOLO_p\!\l[p\r]$ admits an analytic continuation along every curve starting at $p$, then there exists a unique (globally-defined) function $f\in\HOLO\l[p\r]\l(X\r)$ with $\l[f\r]_p=\phi_0$.
    \end{corollary}
    \begin{proof}
        Uniqueness follows from the Identity Theorem. For existence, we define\side{Note that $\psi_q$ is a function \textit{germ}, so by $\psi_q\!\l(q\r)$ we mean $g\l(q\r)$ where $V\ni q$ is any open set and $g\in\HOLO\l[p\r]\l(V\r)$ is any function with $\psi_q=\l[g\r]_q$.} $f\l(q\r)\coloneqq\psi_q\!\l(q\r)$ where $\psi_q\!\in\HOLO_q\!\l[p\r]$ is the analytic continuation along any curve from $p$ to $q$. Since $X$ is simply-connected, the Monodromy Theorem ensures that $\psi_q$ is well-defined. Clearly $f\l(p\r)=\phi_0\!\l(p\r)$, so $\l[f\r]_p=\phi_0$. Finally, since $\l[f\r]_q=\psi_q\in\HOLO_q\!\l[p\r]$ for all $q\in X$, we see that $f\in\HOLO\l[p\r]\l(X\r)$.\qed
    \end{proof}
    \subsection{Analytic Continuation and Covering Spaces}
\end{document}

    % \begin{definition}
    %     Let $X$ and $E$ be connected topological spaces. A covering map $\pi:E\to X$ if said to be the \uldef{universal covering of $X$} if for every covering $\pi':E'\to X$ on a connected topological space $E'$ and every $e\in E$ and $e'\in E'$ such that $\pi\l(e\r)=\pi'\l(e'\r)$, there exists a unique continuous map $\sigma:E\to E'$ with $\sigma\l(e\r)=e'$ making the below diagram commute.
    %     \begin{equation*}
    %         \begin{tikzcd}
    %             E \ar[rr, "\sigma"] \ar[dr, "\pi"'] & & E' \ar[dl, "\pi'"] \\
    %                                                 & X
    %         \end{tikzcd}
    %     \end{equation*}
    % \end{definition}
    % \side[-1.21in]{As with all `universal properties', the universal covering of $X$ is unique up to isomorphism.}
    % \vspace{-0.05in}
    % \begin{remark}
    %     Note that $\sigma$ is the lifting of of $\pi$ along $\pi'$. Recall that if $E$ is simply-connected, such a lifting exists and is unique, so in this case \textit{any} covering map is the universal covering of $X$. We quote the following theorem that guarantees the existence of such a simply-connected space.\exqed
    % \end{remark}
    % \begin{theorem}[{\cite[][Theorem 5.3]{forster}}]
    %     Suppose $X$ is a connected manifold. Then there exists a connected, simply-connected manifold $\tilde{X}$ and a covering map $\pi:\tilde{X}\to X$.
    % \end{theorem}
    % \begin{example}
    %     Recall from Example \ref{2.1:exa:covering_of_torus} that for any lattice $\Gamma\subseteq\C$, the projection $\pi:\C\to\C/\Gamma$ is a covering map. Since $\C$ is simply-connected, we see that $\pi$ is the universal covering of $\C/\Gamma$.\exqed
    % \end{example}
