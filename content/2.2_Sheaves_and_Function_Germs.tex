\documentclass[../Moduli_Spaces_of_Riemann_Surfaces.tex]{subfiles}
\begin{document}
    \section{Sheaves and Function Germs}
    Unless otherwise stated, in this section, $X$ denotes a topological space with $\mc{T}$ its system of open sets. Our exposition on sheaves $-$ which will only be used in Section \ref{sec:analytic_continuation} $-$ roughly follows \cite[][Section 6]{forster} and \cite[][Chapter IX]{miranda}.
    \subsection{Presheaves and Sheaves}
    \begin{definition}
        A \uldef{presheaf of Abelian groups on $X$} is a pair $\tpl{\mc{F},\rho}$ consisting of
        \begin{itemize}
            \item a family $\mc{F}\coloneqq\l\{\mc{F}\l(U\r)\r\}$ of Abelian groups $\mc{F}\l(U\r)$ for every $U\in\mc{T}$,
                \vspace{-0.05in}
            \item a family $\rho\coloneqq\l\{\rho^U_V\r\}$ of group homomorphisms $\rho^U_V:\mc{F}\l(U\r)\to\mc{F}\l(V\r)$ for every $U,V\in\mc{T}$ with $V\subseteq U$,
        \end{itemize}
        such that the following two properties hold:
        \begin{itemize}
            \item For every $U\in\mc{T}$, we have $\rho^U_U=\id_{\mc{F}\l(U\r)}$.
                \vspace{-0.05in}
            \item For every $U,V,W\in\mc{T}$ with $W\subseteq V\subseteq U$, we have $\rho^V_W\circ\rho^U_V=\rho^U_W$.
        \end{itemize}
    \end{definition}
    \begin{remark}\side[-1.36in]{Analogously, we define the presheaf of sets, rings, vector spaces, algebras, etc, on a topological space $X$.\\\ \\For those with some category theory background, fix any category $\cat{C}$. A \ul{$\cat{C}$-valued presheaf on $X$} is simply a contravariant functor $\mc{F}:\mc{T}\to\cat{C}$ where $\mc{T}$ is the preorder category induced by $\tpl{\mc{T},\subseteq}$.}
        Presheaves give us a way of tracking data associated with open sets of a topological space in such a way that makes restricting to a smaller open set $V\subseteq U$ well-behaved. Consider, for instance, a Riemann surface $X$ and the presheaf of all holomorphic functions $\HOLO$ on $X$.
        \begin{itemize}
            \item To every open set $U\subseteq X$\side{Similarly, consider the (multiplicative) group $\HOLO^\ast\!\l(U\r)$ of all holomorphic functions $f:U\to\C^\ast$, which defines a presheaf $\HOLO^\ast$ of Abelian groups. We define $\MERO^\ast\!\l(U\r)$ similarly, but instead restrict to all meromorphic functions $f:U\to\C$ that do not vanish identically on any connected-component of $U$.} we consider the $\C$-algebra $\HOLO\l(U\r)$ of all holomorphic functions $f:U\to\C$. For any open $V\subseteq U$, we define $\rho^U_V\!\l(f\r)\coloneqq\l.f\r|_V$. The two properties are then trivial, which respectively states that restricting to the domain does nothing, and that restricting once to $V$ and then to $W\subseteq V$ yields the same function as restricting to $W$ directly.
        \end{itemize}
        Similarly, we have the presheaf of all meromorphic functions $\MERO$ on $X$. However, those two examples are much more than presheaves since global information about elements in $\mc{F}\l(X\r)$ can be obtained locally by `restricting' to $U$. The notion of a sheaf makes this precise.\exqed
    \end{remark}
    \begin{definition}
        A presheaf $\mc{F}$ on $X$ is said to be a \uldef{sheaf} if for every open set $U\subseteq X$ and every family $\l\{U_i\r\}_{i\in I}$ of open subsets that cover $U$, the following two properties hold:
        \begin{itemize}
            \item (Identity): For every $f,g\in\mc{F}\l(U\r)$, if $\rho^U_{U_i}\!\l(f\r)=\rho^U_{U_i}\!\l(g\r)$ for every $i\in I$, then $f=g$.
                \vspace{-0.05in}
            \item (Gluing): For every family $\l\{f_i\r\}_{i\in I}$ with $f_i\in\mc{F}\l(U_i\r)$, if $\rho^{U_i}_{U_i\cap U_j}\!\l(f_i\r)=\rho^{U_j}_{U_i\cap U_j}\!\l(f_j\r)$ for all $i,j\in I$, then there is a function $f\in\mc{F}\l(U\r)$ such that $\rho^U_{U_i}\!\l(f\r)=f_i$ for every $i\in I$.
        \end{itemize}
    \end{definition}
    \side[-1in]{It is immediate that $\HOLO$, $\HOLO^\ast$, $\MERO$, and $\MERO^\ast$ are all sheaves on $X$. Indeed, if we have a family $\l\{f_i\r\}$ that agree on all pairwise common domains, then there exists a globally defined function $f$ whose restrictions are $f_i$'s. We only need to show that this globally defined function is of the `right type', but this can be checked easily.}
    \vspace{-0.05in}
    \begin{example}
        We give an example of a presheaf that is \textit{not} a sheaf. Let $X$ be a normed vector space. For all $U\subseteq X$, let $\mc{B}\l(U\r)$ be the vector space of all bounded functions $f:U\to\R$.
        \begin{itemize}
            \item It is clear from our remarks above that $\mc{B}$ is a presheaf. In fact, since $\mc{B}\l(U\r)$ contains functions, we see that if $f,g\in\mc{B}\l(U\r)$ agree on all restrictions, then they agree on $U$.
        \end{itemize}
        The problem arises when we consider glueing. For instance, let $U_i\coloneqq\l\{p\in X\mid\l\|p\r\|<i\r\}$ and observe that $\l\{U_i\r\}_{i\in\R^+}$ covers $X$. Consider the family $\l\{f_i\r\}$ where each $f_i\coloneqq\id_{U_i}$, which clearly agree on their pairwise intersections. But no function $f:X\to\R$ such that $\l.f\r|_{U_i}=f_i$ for all $i\in\R^+$ can be bounded, so $\mc{B}$ is not a sheaf.\exqed
    \end{example}
    \begin{example}
        We give two examples of sheaves relating to \textit{divisors}\side{For compact Riemann surfaces, we see that a function $D:X\to\Z$ is a divisor iff it has finite support, so its set of divisors is the free Abelian group of the points of $X$.} on a Riemann surface $X$; that is, a function $D:X\to\Z$ whose support $\l\{p\in X\mid D\l(p\r)\neq0\r\}$ is a discrete subset of $X$.
        \begin{itemize}
            \item Let $D$ be a divisor on $X$. For every $U\in\mc{T}$, let $\HOLO\l[D\r]\l(U\r)$ denote the set of all meromorphic functions $f:X\to\C$ such that $\ord_p\!\l(f\r)\leq D\l(p\r)$ for all $p\in X$. The usual restriction homomorphisms make $\HOLO\l[D\r]$ is a sheaf of Abelian groups, \side[0.02in]{This construction generalizes both $\HOLO$ and $\MERO$. Indeed, $\HOLO=\HOLO\l[0\r]$ and $\MERO=\bigsqcup_{D\neq0}\HOLO\l[D\r]$. Intuitively, the use of divisors here allow us to `bound' the orders of the poles of $f$ at specific points $p$, thereby restricting how badly-behaved it can be.} for if $\l\{f_i\r\}$ is a family of meromorphic functions having poles bounded by $D$, then the meromorphic function $f:X\to\C$ that glues them together also has poles bounded by $D$.
            \item For every $U\in\mc{T}$, let $\mc{D}\l(U\r)$ denote the group of all discretely-supported functions from $U$ to $\Z$ (which is exactly a divisor on $U$). This makes $\mc{D}$ into a sheaf since for every family $\l\{D_i\r\}$, the function $D:X\to\Z$ that glues them together is also discretely-supported.\exqed
        \end{itemize}
    \end{example}
    \begin{definition}
        Let $\tpl{\mc{F},\rho}$ and $\tpl{\mc{G},\sigma}$ be two sheaves of Abelian groups on $X$. A \uldef{morphism} \uldef{of sheaves} $\eta:\mc{F}\to\mc{G}$ is a family $\l\{\eta_U\r\}_{U\in\mc{T}}$ of group homomorphisms $\eta_U:\mc{F}\l(U\r)\to\mc{G}\l(U\r)$ such that for every open set $V\subseteq U$, the following diagram commutes.
        \vspace{-0.05in}
        \begin{equation*}
            \begin{tikzcd}
                \mc{F}\l(U\r) \ar[r, "\eta_U"] \ar[d, "\rho^U_V"'] & \mc{G}\l(U\r) \ar[d, "\sigma^U_V"] \\
                \mc{F}\l(V\r) \ar[r, "\eta_V"] & \mc{G}\l(V\r)
            \end{tikzcd}
            \vspace{-0.02in}
        \end{equation*}
    \end{definition}
    \side[-1.09in]{Phrased categorically, a morphism of sheaves is simply a natural transformation $\eta:\mc{F}\Rightarrow\mc{G}$. This makes the collection of all sheaves on $X$ into a category.}
    \begin{example}
        Some examples relating to divisors of a Riemann surface $X$.
        \begin{itemize}
            \item For divisors $D_1$ and $D_2$ of a Riemann surface $X$, we write $D_1\leq D_2$ if $D_1\!\l(p\r)\leq D_2\!\l(p\r)$ for all $p\in X$. This induces a inclusion map $\iota:\HOLO\l[D_1\r]\into\HOLO\l[D_2\r]$ defined\side{In particular, we have the inclusion $\HOLO\into\MERO$.} by $\iota_U\!\l(f\r)\coloneqq f$ for all $f\in\HOLO\l[D_1\r]\l(U\r)$, which makes sense since if the poles of $f$ are bounded by $D_1$, then clearly they are also bounded by $D_2$ provided $D_1\leq D_2$. This map clearly also respect restrictions, so it is indeed a morphism.
            \item For every non-zero meromorphic function $f:X\to\C$, let $D\l(p\r)\coloneqq\ord_p\!\l(f\r)$. This defines a divisor $D:X\to\Z$ by discreteness of zeros and poles, which induces a map $\div:\MERO^\ast\to\mc{D}$.
        \end{itemize}
    \end{example}
    \subsection{Stalks and their Topology}
    Throughout this section, $p\in X$ is a fixed point in a topological space $X$.
    \begin{definition}
        Let $\mc{F}$ be a presheaf of Abelian groups. The \uldef{stalk of $\mc{F}$ at $p$} is the Abelian group
        \begin{equation*}
            \mc{F}_p\coloneqq\raisebox{-2pt}{$\Biggl(\!$}\bigsqcup_{U\ni p}\mc{F}\l(U\r)\!\raisebox{-2pt}{$\Biggr)$}\!\!_{\scalebox{2}{/}\sim_p}
        \end{equation*}
        where $\sim_p$ is the equivalence relation on the disjoint union, defined, for all $f\in\mc{F}\l(U\r)$ and $g\in\mc{F}\l(V\r)$, by $f\sim_p g$ iff there exists an open set $W$ of $X$ with $p\in W\subseteq U\cap V$ such that $\rho^U_W\!\l(f\r)=\rho^V_W\!\l(g\r)$. For any $f\in\mc{F}\l(U\r)$, its equivalence class $\l[f\r]_p$ is called the \uldef{germ of $f$ at $p$}.
    \end{definition}
    \begin{example}
        
    \end{example}
\end{document}
