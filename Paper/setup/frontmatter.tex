\newgeometry{margin = 0.7in, bmargin = 1in, tmargin = 1in}
\pagestyle{empty}
\begin{center}
    \Large\textbf{\textsc{Moduli Spaces of Riemann Surfaces}}
    \\[1.2\baselineskip]
    \begin{center}
        \large{\textsc{Zhaoshen Zhai}}
    \end{center}
    \normalsize\today
\end{center}
\vspace*{0.01\textheight}
\hspace{0.2in}\textit{Directed Reading Program $-$ Winter 2023}
\\[0.3\baselineskip]
\hspace*{0.19in}\textit{Graduate mentor: Kaleb Ruscitti}
\\[0.3\baselineskip]
\begin{center}
    \textsc{\bfseries Abstract}
    \\[1.5\baselineskip]
    \begin{minipage}{0.85\textwidth}
        The theory of Riemann surfaces, first developed by Bernhard Riemann to study algebraic functions, now lies in the confluence of complex analysis, differential geometry, and algebraic geometry. This expository paper aims to introduce this theory, with the goal classifying all compact Riemann surfaces of genus $0$ and $1$. To do so, we first develop the basics of covering space theory, which defines the degree of proper holomorphic maps, and then study the sheaf of holomorphic maps on a Riemann surface and their associated cohomology theory. Together, they form the core technical tools of the paper and allow us to connect the function theory of Riemann surfaces to their complex structure. Lastly, we give a glimpse into the non-compact case, namely the Uniformization Theorem, which gives us a tri-fold classification of all Riemann surfaces.
    \end{minipage}
\end{center}
\vspace*{0.02\textheight}
\begin{center}
    \textsc{\bfseries Contents}
\end{center}

\toccontents
\clearpage
\pagestyle{fancyplain}
\fancyhead[L,C,R]{}
\fancyfoot[L,R]{}
\fancyfoot[C]{\thepage}
\renewcommand{\headrulewidth}{0pt}
\setcounter{page}{1}
