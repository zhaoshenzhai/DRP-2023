\documentclass[../Moduli_Spaces_of_Riemann_Surfaces.tex]{subfiles}
\begin{document}
    \section{Liftings along Covering Maps}\label{CS:sec:liftings_of_maps}
    Using the Homotopy Lifting Property of covering maps, we prove that every map $F:Y\to X$ from a simply-connected space $Y$ admits a lift $\widetilde{F}:Y\to E$ along a covering map $\pi:E\to X$. Unless otherwise stated, $X$, $Y$, and $E$ are all topological spaces and all maps are continuous.
    \begin{definition}
        Let $\pi:E\to X$ and $F:Y\to X$ be maps. A \ul{lift of $F$ (along $\pi$)} is a map $\widetilde{F}:Y\to E$ such that $\pi\circ\widetilde{F}=F$; that is, such that the diagram below commutes.
        \begin{equation*}
            \begin{tikzcd}
                & E \ar[d, "\pi"] \\
                Y \ar[ur, "\tilde{F}"] \ar[r, "F"'] & X
            \end{tikzcd}
        \end{equation*}
    \end{definition}
    \begin{remark}
        If $X$, $Y$, and $E$ are all Riemann surfaces and $\pi:E\to X$ is an unbranched holomorphic map, then any lift $\widetilde{F}:Y\to E$ of a holomorphic $F:Y\to X$ is also holomorphic. Indeed, $\pi$ admits a local inverse $\cchi$, which is holomorphic, so $\widetilde{F}$ is locally a composition of a holomorphic map $F$ with $\cchi$.\exqed
    \end{remark}
    \subsection{Liftings of Paths and Homotopies}
    \begin{proposition}[Homotopy Lifting Property]
        If $\pi:E\to X$ is a covering map, then for any homotopy $F:Y\times\l[0,1\r]\to X$ and any fixed map $\widetilde{f}_0:Y\times\l\{0\r\}\to E$ lifting the restriction of $F$ on $Y\times\l\{0\r\}$, there exists a unique homotopy $\widetilde{F}:Y\times\l[0,1\r]\to E$ lifting $F$ that restricts to $\widetilde{f}_0$ on $Y\times\l\{0\r\}$. In other words, the following diagram commutes.
        \vspace{-0.05in}
        \begin{equation*}
            \begin{tikzcd}[row sep=0.4in]
                Y\times\l\{0\r\} \ar[r, "\widetilde{f}_0"] \ar[d, "\iota"', hookrightarrow] & E \ar[d, "\pi"] \\
                Y\times\l[0,1\r] \ar[r, "F"] \ar[ur, "\widetilde{F}", dashed] & X
            \end{tikzcd}
        \end{equation*}
    \end{proposition}
    \begin{proof}
        Since $\pi$ is a covering map, there exists an open cover $\l\{U_\alpha\r\}$ of $X$, each evenly-covered by $\l\{V_{\alpha\beta}\r\}$. Fix $q_0\in Y$. For each $\tpl{q_0,t_i}\in Y\times\l[0,1\r]$, let $U_i\subseteq X$ be an open set containing $F\l(q_0,t_i\r)$. Continuity of $F$ then furnishes an open set $N_i\times\l(a_i,b_i\r)\ni\tpl{q_0,t_i}$ such that $F\l(N_i\times\l(a_i,b_i\r)\r)\subseteq U_i\subseteq X$. The collection $\l\{N_i\times\l(a_i,b_i\r)\r\}$ covers $\l\{q_0\r\}\times\l[0,1\r]$, so by compactness one obtains an open set $N\coloneqq\bigcap N_i$ containing $q_0$ and a partition $0=t_0<t_1<\cdots<t_n=1$ of $\l[0,1\r]$ such that each $F\l(N\times\l[t_i,t_{i+1}\r]\r)\subseteq U_i$ is evenly-covered. We define $\widetilde{F}:N\times\l[0,t_i\r]\to E$ by induction on $i$; for $i=0$, we let $\widetilde{F}\coloneqq\widetilde{f}_0$ so that $\widetilde{F}$ restricts to $\widetilde{f}_0$ on $N\times\l\{0\r\}$.\\\ \\
        Suppose a lift $\widetilde{F}:N\times\l[0,t_i\r]\to E$ has been constructed for some $i\geq0$. Then, since $F\l(q_0,t_i\r)\in U_i$, there exists a unique open set $V_i\subseteq\pi^{-1}\!\l(U_i\r)$ containing $\widetilde{F}\l(q_0,t_i\r)$ that maps homeomorphically onto $U_i$. Replacing $N\times\l\{t_i\r\}$ by its intersection with $\widetilde{F}^{-1}\!\l(V_i\r)$, if necessary, we may assume that $\widetilde{F}\l(N\times\l\{t_i\r\}\r)\subseteq V_i$. Since $\pi$ is invertible on $V_i$, extend $\widetilde{F}$ so that
        \vspace{-0.05in}
        \begin{equation*}
            \begin{tikzcd}[row sep = 0.4in]
                & V_i \ar[d, "\pi"] \\
                N\times\l[t_i,t_{i+1}\r] \ar[r, "F"'] \ar[ur, "\widetilde{F}"] & U_i
            \end{tikzcd}
        \end{equation*}
        commutes. Our modification of $N\times\l\{t_i\r\}$ ensures that the restriction of $\widetilde{F}$ to $N\times\l\{t_i\r\}$ coincides with this extension, so the functions inductively glue to give a lift $\widetilde{F}$ of $F$ on $N\times\l[0,1\r]$. We now argue that such a lifting is unique when $Y$ is a point\footnote{Here, we are not necessary assuming that $Y=\l\{q_0\r\}$.}; abusing notation, we drop $Y$ from the notation and write $F:\l[0,1\r]\to X$, etc., instead.
        \begin{itemize}
            \item Suppose that $\widetilde{F}'\!:\l[0,1\r]\to E$ is another lift of $F$ such that $\widetilde{F}\l(0\r)=\widetilde{F}'\!\l(0\r)=\widetilde{f}_0\!\l(0\r)$. As above, we may obtain a partition $0=t_0<t_1<\cdots,t_n=1$ of $\l[0,1\r]$ such that each $F\l(\l[t_0,t_{i+1}\r]\r)\subseteq U_i$ is evenly-covered. Proceeding by induction, suppose that $\widetilde{F}=\widetilde{F}'$ on $\l[0,t_i\r]$. Since $\l[t_i,t_{i+1}\r]$ is connected, $\widetilde{F}\l(\l[t_i,t_{i+1}\r]\r)$ is connected too and thus lies in a single open set $V_i\subseteq\pi^{-1}\!\l(U_i\r)$ containing $\widetilde{F}\l(t_i\r)$ that maps homeomorphically onto $U_i$. Similarly for $\widetilde{F}'\!\l(\l[t_i,t_{i+1}\r]\r)$, but since $\widetilde{F}\l(t_i\r)=\widetilde{F}'\!\l(t_i\r)$, they lie in the same open set $V_i$. Since $\pi\circ\widetilde{F}=\pi\circ\widetilde{F}'$ on $\l[t_i,t_{i+1}\r]$ and $\pi$ is injective on $V_i$, we see that $\widetilde{F}=\widetilde{F}'$ on $\l[t_i,t_{i+1}\r]$, as desired.
        \end{itemize}
        Thus, when restricted to $\l\{q\r\}\times\l[0,1\r]$ for each $q\in N$, the lift $\widetilde{F}:N\times\l[0,1\r]\to E$ of $F$ is unique. In general, this shows that if the same construction is repeated for some other $q_0'\in Y$ to obtain a lift $\widetilde{F}'\!:N'\times\l[0,1\r]\to E$ of $F$, and if $N\cap N'\neq\em$, the lifts $\widetilde{F}$ and $\widetilde{F}'$ must agree on $\l(N\cap N'\r)\times\l[0,1\r]$. Thus $\widetilde{F}$ is well-defined on $Y\times\l[0,1\r]$, and is continuous since it is continuous on each $N\times\l[0,1\r]$.\qed
    \end{proof}
    \begin{corollary}\label{CS:cor:lift_curve_homotopy}
        Every covering map lifts paths and homotopies. More precisely:
        \begin{itemize}
            \item[$\blob$] For each path $\gamma:\l[0,1\r]\to X$ starting at some point $p\in X$ and each $\zeta_0\in\pi^{-1}\!\l(p\r)$, there exists a unique path $\widetilde{\gamma}:\l[0,1\r]\to E$ starting at $\zeta_0$ lifting $\gamma$.
                \vspace{-0.05in}
            \item[$\blob$] For each homotopy $\gamma_t:\l[0,1\r]\to X$ of paths and each lift $\widetilde{\gamma}_0:I\to E$ of $\gamma_0$, there exists a unique homotopy $\widetilde{\gamma}_t:I\to E$ of paths starting at $\widetilde{\gamma}_0$ lifting $\gamma_t$.
        \end{itemize}
    \end{corollary}
    \begin{proof}
        In the notation of the preceding proposition, let $Y$ be a singleton and $\l[0,1\r]$, respectively. Note that the resulting homotopy $\widetilde{\gamma}_t$ is a homotopy \textit{of paths\footnote{As opposed to a free homotopy.}} since as $t$ varies, the endpoints $\widetilde{\gamma}_t\!\l(0\r)$ and $\widetilde{\gamma}_t\!\l(1\r)$ are paths in $E$ that lift the constant paths at $\gamma_t\!\l(0\r)$ and $\gamma_t\!\l(1\r)$, respectively. By uniqueness of liftings of paths, we see that $\widetilde{\gamma}_t\!\l(0\r)$ and $\widetilde{\gamma}_t\!\l(1\r)$ are constant paths at the lifts of $\gamma_t\!\l(0\r)$ and $\gamma_t\!\l(1\r)$, respectively, as desired.\qed
    \end{proof}
    \subsection{Liftings of Mappings}
    We return to the problem of the liftings of mappings. The tools that we have developed actually proves a stronger statement\footnote{See \cite[][Proposition 1.33]{hatcher}, which actually characterizes when such a lift exists.} than is needed in this paper, but for sake of brevity we only present a special case. Throughout, let $\pi:E\to X$ be a covering map.
    \begin{lemma}
        If $Y$ is connected, then any two lifts $\widetilde{F}_1,\widetilde{F}_2:Y\to E$ of $F:Y\to X$ agreeing at one point in $Y$ agrees everywhere.
    \end{lemma}
    \begin{proof}
        Let $S\coloneqq\{q\in Y\,|\,\widetilde{F}_1\!\l(q\r)=\widetilde{F}_2\!\l(q\r)\}$, which we claim to be both open and closed. For a fixed $q\in Y$, let $U$ be a neighborhood of $F\l(q\r)$ that is evenly-covered by open sets $V_i\subseteq E$. Let $V_1$ and $V_2$ be sheets above $U$ containing $\widetilde{F}_1\!\l(q\r)$ and $\widetilde{F}_2\!\l(q\r)$, respectively. By continuity of $\widetilde{F}_1$ and $\widetilde{F}_2$, there exists a neighborhood $V$ of $q$ such that $\widetilde{F}_i\!\l(V\r)\subseteq V_i$ for $i=1,2$.
        \begin{itemize}
            \item If $q\in S$, then $V\coloneqq V_1=V_2$. Then, since $p\circ\widetilde{F}_1=p\circ\widetilde{F}_2$ and $p$ is injective on $V$, we see that $\widetilde{F}_1=\widetilde{F}_2$ on $V$. This shows that $S$ is open.
                \vspace{-0.05in}
            \item Otherwise, $V_1\neq V_2$ and hence they are disjoint. Then, since $\widetilde{F}_i\!\l(V\r)\subseteq V_i$ for $i=1,2$, we see that $\widetilde{F}_1\!\l(q'\r)\neq\widetilde{F}_2\!\l(q'\r)$ for all $q'\in V$. This shows that $Y\comp S$ is open, whence $S$ is closed too.\qed
        \end{itemize}
    \end{proof}
    \begin{proposition}
        Fix $q_0\in Y$ and let $\zeta_0\in\pi^{-1}\!\l(F\l(q_0\r)\r)$. If $Y$ is simply-connected and locally path-connected, then every map $F:Y\to X$ admits a unique lift $\widetilde{F}:Y\to E$ such that $\widetilde{F}\l(q_0\r)=\zeta_0$.
    \end{proposition}
    \begin{proof}
        By the lemma, such a lift is unique if it exists. To construct a lift, let $q\in Y$ and let $\gamma$ be a path from $q_0$ to $q$. Then $F_\gamma\coloneqq F\circ\gamma$ is a path starting at $F\l(q_0\r)$, which, by Corollary \ref{CS:cor:lift_curve_homotopy}, admits a unique lift $\widetilde{F}_\gamma$ starting at $\zeta_0$. Define $\widetilde{F}\l(q\r)\coloneqq\widetilde{F}_\gamma\!\l(1\r)$. Assuming that $\widetilde{F}$ is well-defined and continuous, we have that $(\pi\circ\widetilde{F})\l(q\r)=\pi(\widetilde{F}_\gamma\!\l(1\r))=F\l(\gamma\l(1\r)\r)=F\l(q\r)$, so $\widetilde{F}$ lifts $F$. It remains to show that $\widetilde{F}\l(q\r)$ is well-defined for all $q\in Y$, and that the map $\widetilde{F}$ is continuous.
        \begin{itemize}
            \item (Well-definedness). Let $\delta$ be another path from $q_0$ to $q$. By simply-connectedness of $Y$, the paths $\gamma$ and $\delta$ are homotopic, so $F_\gamma$ and $F_\delta$ are homotopic too. Again by Corollary \ref{CS:cor:lift_curve_homotopy}, this homotopy lifts to a homotopy of paths from $\widetilde{F}_\gamma$ to $\widetilde{F}_\delta$ starting at $\zeta_0$, so they have the same endpoint.
                \vspace{-0.2in}
                \begin{center}
                    \begin{tikzpicture}[scale=0.5]
                        \begin{scope}[yshift = -4cm]
                            \fill (0,0) circle (0.02in) node[below left = -0.1]{\footnotesize$q_0$};
                            \draw (0,0) .. controls (1,1.25) and (2,-0.5) .. (3,0) node[midway, above]{\footnotesize$\gamma$};
                            \draw (0,0) .. controls (1,0.5) and (2,-1.25) .. (3,0) node[midway, below]{\footnotesize$\delta$};
                            \fill (3,0) circle (0.02in) node[above right = -0.1]{\footnotesize$q$};
                        \end{scope}
                        \draw[->] (4,-4) -- (6,-4) node[midway, below]{\footnotesize$F$};
                        \begin{scope}[xshift = 7cm, yshift = -4cm]
                            \fill (0,0) circle (0.02in) node[below]{\footnotesize$F\!\l(q_0\r)$};
                            \draw (0,0) .. controls (1,1.25) and (2,-0.5) .. (3,0) node[midway, above]{\footnotesize$F_\gamma$};
                            \draw (0,0) .. controls (1,0.5) and (2,-1.25) .. (3,0) node[midway, below]{\footnotesize$F_\delta$};
                            \fill (3,0) circle (0.02in) node[above right = -0.1]{\footnotesize$F\!\l(q\r)$};
                        \end{scope}
                        \draw[->] (8.5,-1.25) -- (8.5,-2.75) node[midway, right]{\footnotesize$\pi$};
                        \begin{scope}[xshift = 7cm]
                            \fill (0,0) circle (0.02in) node[below left = -0.1]{\footnotesize$\zeta_0$};
                            \draw (0,0) .. controls (1,1.25) and (2,-0.5) .. (3,0) node[midway, above]{\footnotesize$\widetilde{F}_\gamma$};
                            \draw (0,0) .. controls (1,0.5) and (2,-1.25) .. (3,0) node[midway, below]{\footnotesize$\widetilde{F}_\delta$};
                            \fill (3,0) circle (0.02in) node[above right = -0.1]{\footnotesize$\widetilde{F}\!\l(q\r)$};
                        \end{scope}
                        \draw[->] (3,-3) -- (6,-0.75) node[midway, above left]{\footnotesize$\widetilde{F}$};
                    \end{tikzpicture}
                \end{center}
                \vspace{-0.2in}
            \item (Continuity). Let $q\in Y$, $p\coloneqq F\l(q\r)$, $\zeta\coloneqq\widetilde{F}\l(q\r)$, and let $V_0$ be a neighborhood of $\zeta$. For an evenly-covered neighborhood $U$ of $p$, let $V_\zeta$ denote the sheet above $U$ containing $\zeta$. Set $V\coloneqq V_0\cap V_\zeta$, so $\pi$ is a homeomorphism when restricted to $V$. Thus $\pi\l(V\r)$ is open, so by continuity $F^{-1}\!\l(\pi\l(V\r)\r)$ is open too. By local path-connectedness of $Y$, this set contains a path-connected neighborhood $W$ of $q$. We claim that $\widetilde{F}\l(W\r)\subseteq V$, so take $w\in W$ and let $\sigma$ be a path from $q$ to $w$ contained in $W$. Then $F_\sigma$ is a path in $F\l(W\r)\subseteq\pi\l(V\r)$ starting at $p$, which lifts to a path $\widetilde{F}_\sigma$ in $V$ starting at $\zeta$. But since the end point of $\widetilde{F}_\gamma$ constructed above is $\zeta=\widetilde{F}\l(q\r)$, the concatenation of $\widetilde{F}_\gamma\ast\widetilde{F}_\sigma$ is well-defined and is a path starting at $\zeta_0$. This path lifts $F\circ\l(\gamma\ast\sigma\r)$, and since $\gamma\ast\sigma$ is a path from $q_0$ to $w$, we see that $\widetilde{F}\l(w\r)$ is the end point of $\widetilde{F}_\gamma\ast\widetilde{F}_\sigma$. But this end point is $\widetilde{F}_\gamma\!\l(1\r)$, which lies in $V$.\qed
        \end{itemize}
    \end{proof}
    \begin{example}
        Let $\phi:\C/\Gamma\to\C/\Gamma'$ be a holomorphic map between complex tori. By Example \ref{CS:exa:torus_covering_map}, the projection $\pi:\C\to\C/\Gamma$ is an unbranched holomorphic map, and since $\C$ is simply-connected (and locally path-connected), the map $\phi\circ\pi:\C\to\C/\Gamma'$ admits a unique holomorphic lift $\widetilde{\phi}:\C\to\C$ along the projection $\pi'\!:\C\to\C/\Gamma'$.
        \begin{equation*}
            \begin{tikzcd}
                \C \ar[r, "\tilde{\phi}"] \ar[d, "\pi"'] & \C \ar[d, "\pi'"] \\
                \C/\Gamma \ar[r, "\phi"] & \C/\Gamma'
            \end{tikzcd}
        \end{equation*}
        If $\phi$ is a biholomorphism, then lifting $\phi^{-1}$ too gives us a unique biholomorphism $\widetilde{\phi}:\C\to\C$. A classical result from complex analysis then forces $\widetilde{\phi}\l(z\r)=\alpha z+\beta$ for some $\alpha,\beta\in\C$ with $\alpha\neq0$. This result tightly constrains the behaviour of biholomorphisms between complex tori, which will be useful when computing the moduli space of $T^2$.\exqed
    \end{example}
\end{document}
