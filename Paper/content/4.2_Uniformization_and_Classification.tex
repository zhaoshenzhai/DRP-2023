\documentclass[../Moduli_Spaces_of_Riemann_Surfaces.tex]{subfiles}
\begin{document}
    \section{Uniformization and Classification}\label{sec:uniformization}
    Results in this section will only be discussed briefly and all proofs presented are sketches.
    \begin{theorem}[Uniformization]
        Every simply-connected Riemann surface $X$ is biholomorphic to either the Riemann sphere $\RS$, the complex plane $\C$, or the upper-half plane $\H$.
    \end{theorem}
    \begin{proofsketch}\side[0.1in]{This sketch follows \cite{peter}. The existence of such a meromorphic function $f$ is more involved and uses tools from Dolbeault cohomology.}
        As in Theorem \ref{4.1:thm:simply-connect_compact_biholomorphic_Riemann_sphere}, fix $p\in X$ and let $f\in\HOLO\l[p\r]\l(X\r)$ be meromorphic with a single simple pole. Let $F:X\to\RS$ be its associated holomorphic map, so $F\l(p\r)=\infty$. We outline the rest of the proof.\side[0.23in]{These claims are nontrivial and are beyond the scope of this paper.}
        \begin{itemize}
            \item First, it can be shown that $\Im F\l(x\r)\to0$ as `$x\to\infty$' in $X$. That is, for every $\epsilon>0$, there is a large enough compact subset $K$ of $X$ such that $\Im F\l(x\r)<\epsilon$ for all $x\in X\comp K$.
                \vspace{-0.05in}
            \item It can also be shown that $\im F$ is open, contains the `top and bottom halves' of\side[0.01in]{That is, $\l\{x\in X\mid\Im\phi\l(z\r)\neq0\r\}\subseteq\im F$.} $\RS$, and is a biholomorphism onto its image.
        \end{itemize}
        Thus $\im F\iso\RS\comp I$ for some $I\subseteq\R$. By simply-connectedness of $X$, we see that $I$ is connected and hence we have three possibilities.
        \begin{itemize}
            \item If $I=\em$, then $F:X\to\RS$ is a biholomorphism, which reduces to Theorem \ref{4.1:thm:simply-connect_compact_biholomorphic_Riemann_sphere}.
                \vspace{-0.05in}
            \item If $I$ is a singleton, then $\RS\comp I\iso\C$, so $X\iso\C$.
                \vspace{-0.05in}
            \item If $I$ is an interval $\l[a,b\r]$, we may w.l.o.g. take $a=0$ and $b=\infty$. Then the (usual branch of the) square root function sends $\RS\comp\l[0,\infty\r]$ to $\H$.\qed
        \end{itemize}
    \end{proofsketch}
    \begin{remark}
        It turns out that one can construct a simply-connected Riemann surface $\tilde{X}$ from any Riemann surface $X$. Since $\tilde{X}$ is exactly one of three types, this leads to a classification of Riemann surfaces.\exqed
    \end{remark}
    \begin{definition}
        Let $X$ and $E$ be connected topological spaces. A covering map $\pi:E\to X$ if said to be the \uldef{universal covering of $X$} if for every covering $\pi':E'\to X$ on a connected topological space $E'$ and every $e\in E$ and $e'\in E'$ such that $\pi\l(e\r)=\pi'\l(e'\r)$, there exists a unique continuous map $\sigma:E\to E'$ with $\sigma\l(e\r)=e'$ making the below diagram commute.
        \begin{equation*}
            \begin{tikzcd}
                E \ar[rr, "\ex!\sigma", dashed] \ar[dr, "\pi"'] & & E' \ar[dl, "\pi'"] \\
                                                    & X
            \end{tikzcd}
        \end{equation*}
    \end{definition}
    \side[-1.23in]{As with all `universal properties', the universal covering of $X$ is unique up to isomorphism.}
    \vspace{-0.05in}
    \begin{remark}
        Note that $\sigma$ is the lifting of of $\pi$ along $\pi'$. Recall that if $E$ is simply-connected, such a lifting exists and is unique, so in this case \textit{any} covering map is the universal covering of $X$. We quote the following theorem that guarantees the existence of such a simply-connected space.\exqed
    \end{remark}
    \begin{theorem}[{\cite[][Theorem 5.3]{forster}}]
        Suppose $X$ is a connected manifold. Then there exists a connected, simply-connected manifold $\tilde{X}$ and a covering map $\pi:\tilde{X}\to X$.
    \end{theorem}
    \begin{example}
        Recall from Example \ref{2.1:exa:covering_of_torus} that for any lattice $\Gamma\subseteq\C$, the projection $\pi:\C\to\C/\Gamma$ is a covering map. Since $\C$ is simply-connected, we see that $\pi$ is the universal covering of $\C/\Gamma$.\exqed
    \end{example}
    \begin{remark}
        For any Riemann surface $X$, let\side[0.01in]{In fact, every Riemann surface admits a Riemannian metric of constant curvature, either of $1$, $0$, or $-1$. We wish to study the connection between the conformal and metric structures on Riemann surfaces in the future.} $\tilde{X}$ be its simply-connected universal covering. If $\tilde{X}\iso\RS$ (resp. $\C$, $\H$), then $X$ is said to be \textit{elliptic} (resp. \textit{parabolic}, \textit{hyperbolic}).
        \begin{itemize}
            \item Since $\RS$ is simply-connected, it is the universal covering of itself and hence $\RS$ is elliptic.
                \vspace{-0.05in}
            \item Since $\C$ is the universal covering of any torus $\C/\Gamma$, we see that $\C/\Gamma$ is parabolic.
        \end{itemize}
        It turns out that the universal covering for any compact Riemann surfaces with $g>1$ is $\H$, so they are all hyperbolic. This is a curious fact (which we do not understand) that, moreover, has an analogue for three-dimensional real manifolds (called \textit{$3$-manifolds}). Indeed, \textit{Thurston's Geometrization Conjecture}\side[0.02in]{Proven by Grigori Perelman in 2003, for which he was awarded the Fields Medal.} states that all $3$-manifolds can be decomposed into pieces, each having one of eight different geometric structures, and the richest of the eight geometries turns out to be the hyperbolic $3$-manifold.\exqed
    \end{remark}
\end{document}
