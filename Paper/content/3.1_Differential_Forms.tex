\documentclass[../Moduli_Spaces_of_Riemann_Surfaces.tex]{subfiles}
\begin{document}
    \section{Differential Forms}
    Throughout this section, let $W\subseteq X$ be an open subset of a Riemann surface $X$ and fix $p\in W$.
    \subsection{The Cotangent Space}
    For an open set $V\subseteq\C$, we let $\DIFF\l(V\r)$ denote the $\C$-algebra of all functions $f:V\to\C$ that are differentiable w.r.t. the real coordinates $x$ and $y$, which we simple call \ul{differentiable}\side{By `differentiable', we always mean infinitely-differentiable.}. Using the partial derivative operators $\partial/\partial x$ and $\partial/\partial y$, we define
    \begin{equation*}
        \frac{\partial}{\partial z}\coloneqq\frac{1}{2}\l(\frac{\partial}{\partial x}-i\frac{\partial}{\partial y}\r)\ \ \ \ \ \ \ \ \textrm{and}\ \ \ \ \ \ \ \ \frac{\partial}{\partial\bar{z}}\coloneqq\frac{1}{2}\l(\frac{\partial}{\partial x}+i\frac{\partial}{\partial y}\r).
    \end{equation*}
    In this language, the Cauchy-Riemann equations state that $\HOLO\l(V\r)=\ker\partial/\partial\bar{z}$. We now lift these notions to a Riemann surface $X$.
    \begin{definition}
        A function $f:W\to\C$ is said to be \uldef{differentiable at $p$} if there is a chart $\tpl{U,z}$ of $X$ around $p$ such that $f\circ z^{-1}:z\l(U\r)\to\C$ is differentiable at $z\l(p\r)$. If $f$ is differentiable at every point of $W$, then $f$ is said to be \uldef{differentiable on $W$}.
    \end{definition}
    \side[-0.45in]{As with holomorphic functions, differentiability is chart-independent.}
    \vspace{-0.05in}
    \begin{remark}
        Let $\DIFF\l(W\r)$ denote the $\C$-algebra of all differentiable functions $f:W\to\C$ on $W$. Together with the usual restriction mappings, we obtain a sheaf $\DIFF$ of $\C$-algebras consisting of all differentiable functions on $X$.\exqed
    \end{remark}
    \begin{definition}
        Fix a chart $\tpl{U,z}$ of $X$. The \uldef{partial derivative operator w.r.t. $\tpl{U,z}$} is the operator
        \begin{equation*}
            \frac{\partial}{\partial z}:\DIFF\l(U\r)\to\DIFF\l(U\r)\ \ \ \ \ \ \ \ \textrm{\it{mapping}}\ \ \ \ \ \ \ \ f\mapsto\frac{\partial}{\partial z}\!\l(f\circ z^{-1}\r).
        \end{equation*}
        Similarly, we define $\partial/\partial x$, $\partial/\partial y$, and $\partial/\partial\bar{z}$.
    \end{definition}
    \side[-0.75in]{That is, we define $\partial f/\partial z$ by pulling back the regular partial derivative $\partial/\partial z$ for the function $f\circ z^{-1}$ on $\C$. Again, this is chart-independent.}
    \begin{definition}
        Let $\mf{m}_p\subseteq\DIFF_p$ be the ideal of all differentiable functions vanishing at $p$ and let $\mf{m}_p^2$ be its product. The \uldef{cotangent space of $X$ at $p$} is the quotient space $T_p^\ast X\coloneqq\mf{m}_p/\mf{m}^2_p$. If $U\ni p$ is open and $f\in\DIFF\l(U\r)$, we define the \uldef{differential of $f$ at $p$} as
        \begin{equation*}
            \d_pf\coloneqq\l[f-f\l(p\r)\r]_{\mf{m}_p^2}\in T_p^\ast X.
        \end{equation*}
    \end{definition}
    \side[-0.72in]{That is, let $\mf{m}_p$ contain all germs $\l[f\r]_p$ such that $f\l(p\r)=0$ and let $\mf{m}^2_p$ contain all germs $\l[h\r]_p$ such that $h=\sum_if_ig_i$ for some $f_i,g_i\in\mf{m}_p$.}
    \begin{proposition}
        Let $\tpl{U,z}$ be a chart of $X$ around $p$. Then $\l\{\d_px,\d_py\r\}$ and $\l\{\d_pz,\d_p\bar{z}\r\}$ are both bases for $T_p^\ast X$, and if $f\in\DIFF\l(W\r)$, then
        \begin{equation*}
            \begin{aligned}
                \d_pf&=\l.\frac{\partial f}{\partial x}\r|_p\d_px+\l.\frac{\partial f}{\partial y}\r|_p\d_py \\
                     &=\l.\frac{\partial f}{\partial z}\r|_p\d_pz+\l.\frac{\partial f}{\partial\bar{z}}\r|_p\d_p\bar{z}.
            \end{aligned}
        \end{equation*}
    \end{proposition}
    \begin{proof}
        We first show that $\l\{\d_px,\d_py\r\}$ is a basis for $T_p^\ast X$.
        \begin{itemize}
            \item Let $\l[\eta\r]\in T_p^\ast X$, so $\eta=\l[f\r]_p\in\mf{m}_p$ is a differentiable function germ for some $f\in\DIFF\l(W\r)$. Taylor's Theorem in $\C$ then furnishes $\lambda_1,\lambda_2\in\C$ such that\side[0.25in]{There is no constant term since $\l[f\r]_p\in\mf{m}_p$.}
                \begin{equation*}
                    f=\lambda_1\l(x-x\l(p\r)\r)+\lambda_2\l(y-y\l(p\r)\r)+g
                \end{equation*}
                where $g\in\DIFF\l(W\r)$ is such that $\l[g\r]_p\in\mf{m}^2_p$. This lifts to an equality of germs, so, taking the quotient modulo $\mf{m}^2_p$, we see that $\l[\eta\r]=\lambda_1\d_px+\lambda_2\d_py$.
            \item For any $\lambda_1,\lambda_2\in\C$, the linear dependence $\lambda_1\d_px+\lambda_2\d_py=0$ implies that
                \begin{equation*}
                    \lambda_1\l(x-x\l(p\r)\r)+\lambda_2\l(y-y\l(p\r)\r)\in\mf{m}^2_p.
                \end{equation*}
                Taking the partials $\partial/\partial x$ and $\partial/\partial y$ shows that $\lambda_1=\lambda_2=0$.
        \end{itemize}
        Suppose now that $f\in\DIFF\l(W\r)$. By Taylor's Theorem, we have
        \begin{equation*}
            f-f\l(p\r)=\l.\frac{\partial f}{\partial x}\r|_p\!\l(x-x\l(p\r)\r)+\l.\frac{\partial f}{\partial y}\r|_p\!\l(y-y\l(p\r)\r)+g
        \end{equation*}
        where $g\in\DIFF\l(W\r)$ is such that $\l[g\r]\in\mf{m}^2_p$, so lifting this to an equality of germs and taking the quotient modulo $\mf{m}^2_p$ gives us
        \begin{equation*}
            \d_pf=\l.\frac{\partial f}{\partial x}\r|_p\d_px+\l.\frac{\partial f}{\partial y}\r|_p\d_py.
        \end{equation*}
        Finally, we show the corresponding result for $\l\{\d_pz,\d_p\bar{z}\r\}$. Indeed, since $z=x+iy$ as functions in $\DIFF\l(W\r)$, we have that $\partial z/\partial x=1$ and $\partial z/\partial y=i$. Similarly, $\partial\bar{z}/\partial x=1$ and $\partial\bar{z}/\partial y=-i$, so
        \begin{equation*}
            \d_pz=\d_px+i\d_py\ \ \ \ \ \ \ \ \textrm{and}\ \ \ \ \ \ \ \ \d_p\bar{z}=\d_px-i\d_py.
        \end{equation*}
        Thus $\l\{\d_pz,\d_p\bar{z}\r\}$ is linearly-independent, so it is a basis for $T_p^\ast X$. For $f\in\DIFF\l(W\r)$, a computation now shows that
        \begin{equation*}
            \d_pf=\frac{1}{2}\l(\l.\frac{\partial f}{\partial x}\r|_p-i\l.\frac{\partial f}{\partial y}\r|_p\r)\d_pz+\frac{1}{2}\l(\l.\frac{\partial f}{\partial x}\r|_p+i\l.\frac{\partial f}{\partial y}\r|_p\r)\d_p\bar{z}=\l.\frac{\partial f}{\partial z}\r|_p\d_pz+\l.\frac{\partial f}{\partial\bar{z}}\r|_p\d_p\bar{z}.\qedin
        \end{equation*}
    \end{proof}
    \begin{proposition}[Canonical Decomposition]
        Let $\tpl{U,z}$ be a chart of $X$ around $p$. Then the subspaces
        \begin{equation*}
            T_p^\ast X^{\l(1,0\r)}\coloneqq\span\l\{\d_pz\r\}\ \ \ \ \ \ \ \ \textrm{and}\ \ \ \ \ \ \ \ T_p^\ast X^{\l(0,1\r)}\coloneqq\span\l\{\d_p\bar{z}\r\}
        \end{equation*}
        are chart-independent and $T_p^\ast X=T_p^\ast X^{\l(1,0\r)}\oplus T_p^\ast X^{\l(0,1\r)}$.
    \end{proposition}
    \begin{proof}
        If $\tpl{U',z'}$ is another chart of $X$ around $p$, then, since $z'\in\HOLO\l(U\cap U'\r)$, we see that
        \begin{equation*}
            \d_pz'=\l.\frac{\partial z'}{\partial z}\r|_p\d_pz+\l.\frac{\partial z'}{\partial\bar{z}}\r|_p\d_p\bar{z}.
        \end{equation*}
        But $\partial z'/\partial\bar{z}=0$, so $\span\l\{\d_pz\r\}=\span\l\{\d_pz'\r\}$. Similarly, $\partial\bar{z'}/\partial z=0$, so $\span\l\{\d_p\bar{z}\r\}=\span\big\{\d_p\bar{z'}\big\}$. The decomposition then follows by construction.\qed
    \end{proof}
    \begin{remark}
        For\side{For computations, we descend via any chart $\tpl{U,z}$ of $X$ around $p$ where we have
        \begin{equation*}
            \partial_pf=\l.\frac{\partial f}{\partial z}\r|_p\d_pz\ \ \ \ \textrm{and}\ \ \ \ \bar{\partial}_pf=\l.\frac{\partial f}{\partial\bar{z}}\r|_p\d_p\bar{z}.
        \end{equation*}} all $f\in\DIFF\l(W\r)$, let $\partial_pf\in T_p^\ast X^{\l(1,0\r)}$ and $\bar{\partial}_pf\in T_p^\ast X^{\l(0,1\r)}$ be the unique elements such that $\d_pf=\partial_pf+\bar{\partial}_pf$. The above proposition ensures that they are chart-independent.\exqed
    \end{remark}
    \subsection{Differential $1$-forms}
    \begin{definition}
        A \uldef{differential $1$-form on $W$} is a map
        \begin{equation*}
            \omega:W\to\bigcup_{p\in W}T_p^\ast X
        \end{equation*}
        such that $\omega\l(p\r)\in T_p^\ast X$ for every $p\in W$.
    \end{definition}
    \side[-0.83in]{With the induced operations from $T_p^\ast X$, the set of all $1$-forms on $W$ becomes a $\C$-vector space. In fact, it is a $\C$-algebra, for if $f:W\to\C$ is a function, then the map $f\omega$ defined by $\l(f\omega\r)\l(p\r)\coloneqq f\l(p\r)\omega\l(p\r)$ is also a $1$-form on $W$.}
    \vspace{-0.05in}
    \begin{example}
        For $f\in\DIFF\l(W\r)$, the maps $\d f$, $\partial f$, and $\bar{\partial}f$ defined by
        \begin{equation*}
            \l(\d f\r)\l(p\r)\coloneqq\d_pf,\ \ \ \ \ \ \ \ \l(\partial f\r)\l(p\r)\coloneqq\partial_pf,\ \ \ \ \ \ \ \ \textrm{and}\ \ \ \ \ \ \ \ \big(\bar{\partial}f\big)\l(p\r)\coloneqq\bar{\partial}_pf
        \end{equation*}
        for all $p\in W$ are all $1$-forms. Note that if $\tpl{U,z}$ is a chart of $X$, then every $1$-form $\omega$ on $W$ can be written as
        \begin{equation*}
            \omega=f_1\d x+f_2\d y=f_1'\d z+f_2'\d\bar{z}
        \end{equation*}
        for some\side{We note that the functions $f_1,f_2,f_1',f_2'$ are not necessarily continuous.} $f_1,f_2,f_1',f_2':U\to\C$. Indeed, for all $p\in U$, we have $\omega\l(p\r)=f_1\!\l(p\r)\d_px+f_2\!\l(p\r)\d_py$ for some $f_1\!\l(p\r),f_2\!\l(p\r)\in\C$. Varying over all $p\in U$ gives us functions $f_1,f_2:U\to\C$. Similarly for $f_1'$ and $f_2'$.\exqed
    \end{example}
    \begin{definition}
        We define certain subspaces of $1$-forms on $W$ as follows.
        \begin{itemize}
            \item The subspace $\DIFF^{\l(1\r)}\!\l(W\r)$ of all \uldef{differentiable $1$-forms} $\omega$ on $W$ such that, w.r.t. every chart $\tpl{U,z}$ of $X$, $\omega=f\d z+g\d\bar{z}$ for some $f,g\in\DIFF\l(U\cap W\r)$.
            \item The subspace $\DIFF^{\l(1,0\r)}\!\l(W\r)$ (resp. $\DIFF^{\l(0,1\r)}$) of all \uldef{type $\l(1,0\r)$} (resp. \uldef{$\l(0,1\r)$}) \uldef{$1$-forms} $\omega$ on $W$ such that, w.r.t. every chart $\tpl{U,z}$ of $X$, $\omega=f\d z$ (resp. $\omega=f\d\bar{z}$) for some $f\in\DIFF\l(U\cap W\r)$.
            \item The subspace $\Omega\l(W\r)$ of all \uldef{holomorphic $1$-forms} $\omega$ on $W$ such that, w.r.t. every chart $\tpl{U,z}$ of $X$, $\omega=f\d z$ for some $f\in\HOLO\l(U\cap W\r)$.
        \end{itemize}
    \end{definition}
    \begin{remark}
        More work needs to be done to define \textit{meromorphic $1$-forms} on $W$. In fact, we may analogously define the \textit{order of a pole} of a meromorphic $1$-form; see \cite[][Section 9.9]{forster}.\exqed
    \end{remark}
    \begin{example}
        For $f\in\DIFF\l(W\r)$, the form $\d f$ (resp. $\partial f$, $\bar{\partial}f$) is a differentiable (resp. type $\l(1,0\r)$, type $\l(0,1\r)$) $1$-form on $W$. Thus we have the map $\d:\DIFF\l(W\r)\to\DIFF^{\l(1\r)}\!\l(W\r)$, and similarly the maps $\partial$ and $\bar{\partial}$, called the \ul{exterior derivatives on $\DIFF\l(W\r)$}. These exterior derivatives, which are in fact morphisms of sheaves, are studied in the next section.\exqed
    \end{example}
    \subsection{Differential $2$-forms and Exterior Differentiation}
    Define\side{For an in-depth discussion of the tensor product, see \cite[][Chapter 8.2]{aluffi} or \cite{conrad}.} the \textit{exterior power} $\Lambda^2V$ of a $\C$-vector space $V$ as the quotient of the tensor product $V\otimes V$ by the ideal $\mf{a}\coloneqq\ideal{v\otimes v\mid v\in V}$. For completeness, we very briefly define $V\otimes V$.
    \begin{definition}
        Let $V$ be a $\C$-vector space and consider the free vector space $\tpl{F,j}$ over $V\times V$. Letting $S$ denote the span of
        \begin{equation*}
            j\l(v,\lambda v_1+v_2\r)-\lambda j\l(v,v_1\r)-\lambda j\l(v,v_2\r)\ \ \ \ \textrm{and}\ \ \ \ j\l(\lambda v_1+v_2,v\r)-\lambda j\l(v,v_1\r)-\lambda j\l(v,v_2\r),
        \end{equation*}
        for all $v,v_1,v_2\in V$ and $\lambda\in\C$, we define the \uldef{tensor product} of $V$ as the quotient space $V\otimes V\coloneqq F/S$ equipped with the map $\otimes\coloneqq\pi\circ j$, where $\pi:F\to F/S$ is the projection.
    \end{definition}
    \side[-0.92in]{Here, $j:V\times V\to F$ is a function making $\tpl{F,j}$ satisfy the universal property of the free vector space over $V\times V$.}
    \vspace{-0.05in}
    \begin{remark}
        Let $V$ be a $\C$-vector space. For all $v_1,v_2\in V$, define $v_1\wedge v_2\in\Lambda^2V$ to be the equivalence class of $v\otimes v$ modulo $\mf{a}$. It is then immediate from the definition of $V\otimes V$ that
        \begin{equation*}
            \l(v_1+v_2\r)\wedge v_3=\l(v_1\wedge v_3\r)+\l(v_2\wedge v_3\r)\ \ \ \ \ \ \ \ \textrm{and}\ \ \ \ \ \ \ \ \l(\lambda v_1\r)\wedge v_2=\lambda\l(v_1\wedge v_2\r)
        \end{equation*}
        for all $v_1,v_2,v_3\in V$ and $\lambda\in\C$. Moreover,
        \begin{equation*}
            \begin{aligned}
                0&=\l(v_1+v_2\r)\wedge\l(v_1+v_2\r) \\
                 &=\l(v_1\wedge v_1\r)+\l(v_1\wedge v_2\r)+\l(v_2\wedge v_1\r)+\l(v_2\wedge v_2\r) \\
                 &=\l(v_1\wedge v_2\r)+\l(v_2\wedge v_1\r),
            \end{aligned}
        \end{equation*}
        so $v_1\wedge v_2=-\l(v_2\wedge v_1\r)$ for all $v_1,v_2\in V$. Finally, if $\l\{e_i\r\}$ is a basis for $V$, then\side{For a proof, see \cite[][Proposition 12.8]{lee}.} $\l\{e_i\otimes e_j\r\}$ is a basis for $V\otimes V$. Combined with the above, we see that $\l\{e_i\wedge e_j\r\}_{i<j}$ is a basis for $\Lambda^2V$.\exqed
    \end{remark}
    \begin{remark}
        We now specialize for when $V=T_p^\ast X$\side{Recall our notation, where $W\subseteq X$ is an open subset of a Riemann surface $X$ and $p\in W$.} and consider the exterior power $\Lambda^2T_p^\ast X$. Letting $\tpl{U,z}$ be a chart of $X$ around $p$, we see that $\l\{\d_px\wedge\d_py\r\}$ and $\l\{\d_pz\wedge\d_p\bar{z}\r\}$ are both bases for $\Lambda^2T_p^\ast X$. Thus $\dim\Lambda^2T_p^\ast X=1$. Also, observe that
        \begin{equation*}
            \d_pz\wedge\d_p\bar{z}=\l(\d_px+i\d_py\r)\wedge\l(\d_px-i\d_py\r)=-2i\l(\d_px\wedge\d_py\r).\exqedin
        \end{equation*}
    \end{remark}
    \begin{definition}
        A \uldef{differential $2$-form on $W$} is a map
        \begin{equation*}
            \omega:W\to\bigcup_{p\in W}\Lambda^2T_p^\ast X
        \end{equation*}
        such that $\omega\l(p\r)\in\Lambda^2T_p^\ast X$ for every $p\in W$. A $2$-form $\omega$ is said to be \uldef{differentiable} if, w.r.t. every chart $\tpl{U,z}$ of $X$, we have $\omega=f\d z\wedge\d\bar{z}$ for some $f\in\DIFF\l(U\cap W\r)$.
    \end{definition}
    \side[-0.97in]{As with $1$-forms, the set of all $2$-forms on $W$ forms a vector space under the induced operations from $\Lambda^2T_p^\ast X$. Similarly, it is also a $\C$-algebra by defining the map $f\omega$ by $\l(f\omega\r)\l(p\r)\coloneqq f\l(p\r)\omega\l(p\r)$ for every function $f:W\to \C$.}
    \vspace{-0.05in}
    \begin{remark}
        In the above definition, $\d z\wedge\d\bar{z}$ is the $2$-form on $W$ defined by $\l(\d z\wedge\d\bar{z}\r)\l(p\r)\coloneqq\d_pz\wedge\d_p\bar{z}$ for every $p\in W$. In general, if $\omega_1$ an $\omega_2$ are $1$-forms on $W$, we have the $2$-form $\omega_1\wedge\omega_2$ defined by
        \begin{equation*}
            \l(\omega_1\wedge\omega_2\r)\l(p\r)\coloneqq\omega_1\!\l(p\r)\wedge\omega_2\!\l(p\r)
        \end{equation*}
        for every $p\in W$. The $\C$-vector space of all differentiable $2$-forms on $W$ is denoted $\DIFF^{\l(2\r)}\!\l(W\r)$.\exqed
    \end{remark}
    \begin{defprop}
        Let $\omega$ be a differentiable $1$-form on $W$, which, under a chart $\tpl{U,z}$ of $X$, has the form $\omega=f_1\d z+f_2\d\bar{z}$ for some $f_1,f_2\in\DIFF\l(U\cap W\r)$. Then the $2$-forms
        \begin{equation*}
            \begin{gathered}
                \d\omega\coloneqq\d f_1\wedge\d z+\d f_2\wedge\d\bar{z} \\
                \partial\omega\coloneqq\partial f_1\wedge\d z+\partial f_2\wedge\d\bar{z} \\
                \bar{\partial}\omega\coloneqq\bar{\partial}f_1\wedge\d z+\bar{\partial}f_2\wedge\d\bar{z}
            \end{gathered}
        \end{equation*}
        are chart-independent and differentiable, which define the \uldef{exterior derivatives on $\DIFF^{\l(1\r)}\!\l(W\r)$}
        \begin{equation*}
            \d,\partial,\bar{\partial}:\DIFF^{\l(1\r)}\!\l(W\r)\to\DIFF^{\l(2\r)}\!\l(W\r).
        \end{equation*}
    \end{defprop}
    \begin{proof}
        For convenience, we write $z_1\coloneqq z$ and $z_2\coloneqq\bar{z}$, so $\omega=\sum_if_i\d z_i$ and $\d\omega=\sum_i\d f_i\wedge\d z_i$. We only prove the result for $\d$; the others are similar.
        \begin{itemize}
            \item To show that $\d\omega\in\DIFF^{\l(2\r)}\!\l(W\r)$, let $\tpl{V,w}$ be a chart of $X$. Then
                \begin{equation*}
                    \d f_i=\frac{\partial f_i}{\partial w}\d w+\frac{\partial f_i}{\partial\bar{w}}\d\bar{w}\ \ \ \ \ \ \ \ \textrm{and}\ \ \ \ \ \ \ \ \d z_i=\frac{\partial z_i}{\partial w}\d w+\frac{\partial z_i}{\partial\bar{w}}\d\bar{w}
                \end{equation*}
                for $i=1,2$, so
                \begin{equation*}
                    \begin{aligned}
                        \d\omega&=\sum_{j=1}^{2}\l(\frac{\partial f_i}{\partial w}\d w+\frac{\partial f_i}{\partial\bar{w}}\d\bar{w}\r)\wedge\l(\frac{\partial z_i}{\partial w}\d w+\frac{\partial z_i}{\partial\bar{w}}\d\bar{w}\r) \\
                                &=\sum_{j=1}^{2}\l(\frac{\partial f_i}{\partial w}\frac{\partial z_i}{\partial\bar{w}}-\frac{\partial f_i}{\partial\bar{w}}\frac{\partial z_i}{\partial w}\r)\d w\wedge\d\bar{w}\in\DIFF^{\l(2\r)}\!\l(W\r).
                    \end{aligned}
                \end{equation*}
        \end{itemize}
        To show well-definition, let $\tpl{U',z'}$ be another chart of $X$ and write $\omega=\sum_if_i'\d z_i'$.\side[-0.01in]{Again, write $z_1'\coloneqq z'$ and $z_2'\coloneqq\bar{z'}$.} Choose a chart $\tpl{V,w}$ of $X$. Expanding $\d z_i$ and $\d z_i'$ in the basis $\l\{\d w,\d\bar{w}\r\}$ and equating, we obtain by the assumption $\sum_if_i\d z_i=\sum_if_i'\d z_i'$ that
        \begin{equation*}
            \sum_{i=1}^{2}f_i\frac{\partial z_i}{\partial w}=\sum_{i=1}^{2}f_i'\frac{\partial z_i'}{\partial w}\ \ \ \ \ \ \ \ \textrm{and}\ \ \ \ \ \ \ \ \sum_{i=1}^{2}f_i\frac{\partial z_i}{\partial\bar{w}}=\sum_{i=1}^{2}f_i'\frac{\partial z_i'}{\partial\bar{w}}.
        \end{equation*}
        Applying $\partial/\partial\bar{w}$ and $\partial/\partial w$ respectively and subtracting yields
        \begin{equation*}
            \sum_{i=1}^{2}\l(\frac{\partial f_i}{\partial w}\frac{\partial z_i}{\partial\bar{w}}-\frac{\partial f_i}{\partial\bar{w}}\frac{\partial z_i}{\partial w}\r)=\sum_{i=1}^{2}\l(\frac{\partial f_i'}{\partial w}\frac{\partial z_i'}{\partial\bar{w}}-\frac{\partial f_i'}{\partial\bar{w}}\frac{\partial z_i'}{\partial w}\r).
        \end{equation*}
        From our previous calculation of $\d\omega$, the result follows.\qed
    \end{proof}
    \begin{definition}
        A differentiable $1$-form $\omega$ on $W$ is \uldef{closed} if $\d\omega=0$, and is \uldef{exact} if $\omega=\d f$ for some $f\in\DIFF\l(W\r)$.
    \end{definition}
    \begin{proposition}\ 
        \begin{enumerate}
            \item Every exact form is closed.
                \vspace{-0.05in}
            \item Every holomorphic $1$-form is closed.
                \vspace{-0.05in}
            \item Every closed $1$-form of type $\l(1,0\r)$ is holomorphic.
        \end{enumerate}
    \end{proposition}
    \begin{proof}
        Let $\omega$ be a $1$-form on $W$.
        \begin{enumerate}
            \item This is precisely the statement that $\d^2\!f=0$ for all $f\in\DIFF\l(W\r)$, which follows from\side[0.21in]{The same computation also shows $\partial^2\!f=\bar{\partial}^2\!f=0$.}
                \begin{equation*}
                    \d^2\!f=\d\l(1\cdot\d f\r)=\d1\wedge\d f=0.
                \end{equation*}
        \end{enumerate}
        \vspace{-0.05in}
        For $2$ and $3$, suppose that $\omega=f\d z$ for some $f\in\DIFF\l(W\r)$. Then
        \begin{equation*}
            \d\omega=\d f\wedge\d z=\l(\frac{\partial f}{\partial z}\d z+\frac{\partial f}{\partial\bar{z}}\d\bar{z}\r)=\d z=-\frac{\partial f}{\partial\bar{z}}\d z\wedge\d\bar{z}
        \end{equation*}
        Thus $\d\omega=0$ iff $\partial f/\partial\bar{z}=0$, so every holomorphic $1$-form is closed and every closed $1$-form of type $\l(1,0\r)$ is holomorphic.\qed
    \end{proof}
    \subsection{Integration of $2$-forms}
\end{document}
