\documentclass[../Moduli_Spaces_of_Riemann_Surfaces.tex]{subfiles}
\begin{document}
    \section{Stalks and the Étalé Space}
    Throughout this section, $p\in X$ is a fixed point in a topological space $X$.
    \begin{definition}\label{CS:def:stalk}
        Let $\ms{F}$ be a presheaf of Abelian groups on $X$. The \uldef{stalk of $\ms{F}$ at $p$} is the Abelian group
        \begin{equation*}
            \ms{F}_p\coloneqq\raisebox{-2pt}{$\Biggl(\!$}\coprod_{U\ni p}\ms{F}\l(U\r)\!\raisebox{-2pt}{$\Biggr)$}\!\!_{\scalebox{2}{/}\sim_p}
        \end{equation*}
        where $\sim_p$ is the equivalence relation on the disjoint union, defined, for all $f\in\ms{F}\l(U\r)$ and $g\in\ms{F}\l(V\r)$, by $f\sim_p g$ iff there exists an open set $W\in\tau$ with $p\in W\subseteq U\cap V$ such that $\rho^U_W\!\l(f\r)=\rho^V_W\!\l(g\r)$. For any $f\in\ms{F}\l(U\r)$, its equivalence class $\l[f\r]_p$ is called the \uldef{germ of $f$ at $p$}.
    \end{definition}
    \footnote{The relation $\sim_p$ is transitive since $\rho^V_W\circ\rho^U_V=\rho^U_W$ for all $U,V,W\in\tau$ such that $W\subseteq V\subseteq U$.}
    \vspace{-0.05in}
    \begin{example}
        Let\footnote{This construction is analogous to that of the \textit{tangent space} $T_pM$ of a (real) manifold $M$ at some point $p$.}\footnote{This equivalence relation allows us to `evaluate' a function germ $\eta\in\HOLO_p\!\l[D\r]$ as $\eta\l(p\r)\coloneqq f\l(p\r)$ where $U\ni p$ is any open set and $f\in\HOLO\l[D\r]\l(U\r)$ is any function such that $\eta=\l[f\r]_p$.} $D$ be a divisor on a Riemann surface $X$ and consider the stalk $\HOLO_p\!\l[D\r]$. Fix a chart centered at $p$. Since any meromorphic function $f$ admits a Laurent series, we see that the function germ $\l[f\r]_p$ is represented by a Laurent series $\sum_{i=i_0}^{\infty}c_iz^i$ for some $i_0\geq-D\l(p\r)$ and $c_i\in\C$. Conversely, the germ of any Laurent series $\sum_{i=i_0}^{\infty}c_iz^i$ with $i_0\geq-D\l(p\r)$ and $c_i\in\C$ lifts to a meromorphic function germ $\l[f\r]_p$, so this defines a bijection\footnote{This isomorphism depends on the chosen chart map, so it is not canonical.} between $\HOLO_p\!\l[D\r]$ and the set of all such Laurent series.\exqed
    \end{example}
    \begin{remark}
        The sheaf axioms guarantee that if $\ms{F}$ is a sheaf of Abelian groups on $X$ and $U\in\tau$, then an element $f\in\ms{F}\l(U\r)$ is zero iff\footnote{The forward direction is tautological.} all germs $\l[f\r]_p$, for $p\in U$ vanish. Indeed, let $0\in\ms{F}\l(U\r)$ denote the zero element, so $f\sim_p0$ for all $p\in U$ furnishes a family $\l\{W_p\r\}$ of open sets $W_p\subseteq U$ containing $p$ such that $\rho^U_{W_p}\!\l(f\r)=\rho^U_{W_p}\!\l(0\r)$. This family covers $U$, so $f=0$ by the first sheaf axiom.\exqed
    \end{remark}
    \begin{proposition}\label{CS:prp:basis_stalk}
        Let $\ms{F}$ be a presheaf of Abelian groups on $X$. Let $\l|\ms{F}\r|\coloneqq\coprod_{p\in X}\ms{F}_p$ and consider the projection $\pi:\l|\ms{F}\r|\to X$ mapping each $\eta\in\ms{F}_p$ to $p$. Then the system $\mc{B}$ of all sets
        \begin{equation*}
            \l[U,f\r]\coloneqq\bigl\{\l[f\r]_p\,\l|\,p\in U\bigr\}\r.\subseteq\l|\ms{F}\r|
        \end{equation*}
        for $U\in\tau$ and $f\in\ms{F}\l(U\r)$ is a basis for a topology on $\l|\ms{F}\r|$ and $\pi$ is a local homeomorphism.
    \end{proposition}
    \begin{proof}
        We first verify that $\mc{B}$ is a basis.
        \begin{enumerate}
            \item[(1)] Take $\eta\in\l|\ms{F}\r|$, so there exists an open set $U\in\tau$ such that $\eta=\l[f\r]_p$ for some $f\in\ms{F}\l(U\r)$ and $p\in U$. Observe that $\eta\in\l[U,f\r]$.
            \item[(2)] Take $\l[U,f\r],\l[V,g\r]\in\mc{B}$ and $\eta\in\l[U,f\r]\cap\l[V,g\r]$. Then there exists a point $p\in X$ such that $\eta=\l[f\r]_p=\l[g\r]_p$, which furnishes an open set $W\in\tau$ with $p\in W\subseteq U\cap V$ such that $\rho^U_W\!\l(f\r)=\rho^V_W\!\l(g\r)\eqqcolon h$. Then $\eta=\l[h\r]_p$ with $h\in W$, so $\eta\in\l[W,h\r]\subseteq\l[U,f\r]\cap\l[V,g\r]$.
        \end{enumerate}
        To show that $\pi$ is a local homeomorphism, fix $\eta\in\l|\ms{F}\r|$, say with $p\coloneqq\pi\l(\eta\r)$. By (1), there exists some $\l[U,f\r]\in\mc{B}$ containing $\eta$; we claim that $\l.\pi\r|_{\l[U,f\r]}:\l[U,f\r]\to U$ is a homeomorphism.
        \begin{itemize}
            \item For injectivity, take $\psi_1,\psi_2\in\l[U,f\r]$ such that $\pi\l(\psi_1\r)=\pi\l(\psi_2\r)$. Then $\psi_1=\l[f\r]_p$ and $\psi_2=\l[f\r]_q$ for some $p,q\in X$, but since $p=q$, they coincide.
            \item For continuity, it suffices to show that $\l.\pi\r|_{\l[U,f\r]}$ is an open map. Indeed, if $\l[V,g\r]\subseteq\l[U,f\r]$ is open, then $\l.\pi\r|_{\l[U,f\r]}\l(\l[V,g\r]\r)=V$ is open too.\qed
        \end{itemize}
    \end{proof}
    \begin{definition}
        The \uldef{Étalé space} of a presheaf $\ms{F}$ of Abelian groups on $X$ is the topological space $\l|\ms{F}\r|$ equipped the projection $\pi:\l|\ms{F}\r|\to X$.
    \end{definition}
    \begin{definition}
        A presheaf $\ms{F}$ of Abelian groups on $X$ is said to satisfy the \uldef{Identity Theorem} if for all $U\in\tau$ and all $f,g\in\ms{F}\l(U\r)$, if there is some $p\in U$ such that $\l[f\r]_p=\l[g\r]_p$, then $f=g$ (on $U$).
    \end{definition}
    \footnote{In particular, this holds for all $\HOLO\l[D\r]$. In contrast, the sheaf of smooth functions $\DIFF$ (see Section \ref{sec:differential_forms}) does \textit{not} satisfy the Identity Theorem.}
    \begin{proposition}\label{CS:prp:stalk_topology_Hausdorff}
        If $X$ is a locally-connected Hausdorff space and $\ms{F}$ is a presheaf of Abelian groups on $X$ that satisfy the Identity Theorem, then $\l|\ms{F}\r|$ is Hausdorff.
    \end{proposition}
    \begin{proof}
        Take distinct $\eta_1,\eta_2\in\l|\ms{F}\r|$. Two cases occur.
        \begin{itemize}
            \item If $p\coloneqq\pi\l(\eta_1\r)\neq\pi\l(\eta_2\r)\eqqcolon q$, then, since $X$ is Hausdorff, there exist disjoint neighborhoods $U$ of $p$ and $V$ of $q$. On those neighborhoods, $\pi$ is invertible and the sets $\pi^{-1}\!\l(U\r)$ and $\pi^{-1}\!\l(V\r)$ are disjoint neighborhoods of $\eta_1$ and $\eta_2$, respectively.
        \end{itemize}
        Otherwise, set $p\coloneqq\pi\l(\eta_1\r)=\pi\l(\eta_2\r)$ and suppose that each $\eta_i$ is represented by some $f_i\in\ms{F}\l(U_i\r)$. Since $X$ is locally-connected, there exists a connected neighborhood $U\subseteq U_1\cap U_2$ of $p$. Restricting both $f_i$ to $g_i\coloneqq\rho^{U_i}_U\!\l(f_i\r)$, the sets $\l[U,g_i\r]$ are neighborhoods of $\eta_i$. Suppose, for sake of contradiction, that there exists some $\psi\in\l[U,g_1\r]\cap\l[U,g_2\r]$. Setting $q\coloneqq\pi\l(\psi\r)$, we see that $\psi=\l[g_1\r]_q=\l[g_2\r]_q$, from which the Identity Theorem shows that $g_1=g_2$. Note that $f_i\sim_pg_i$, so $\eta_1=\eta_2$, a contradiction. Hence the neighborhoods $\l[U,g_1\r]$ an $\l[U,g_2\r]$ are disjoint, as desired.\qed
    \end{proof}
\end{document}
