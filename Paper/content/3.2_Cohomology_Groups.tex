\documentclass[../Moduli_Spaces_of_Riemann_Surfaces.tex]{subfiles}
\begin{document}
    \section{$\check{\textrm{C}}$ech Cohomology Groups}
    Throughout this section, let $X$ be a topological space, $\ms{F}$ a sheaf of Abelian groups on $X$, and $\mf{A}\coloneqq\l\{U_i\r\}$ an open covering of $X$.
    \subsection{Cochains, Coboundaries, and Cocycles}
    \begin{definition}
        For all $n\in\N$, the \uldef{$n^\textrm{th}$ cochain group of $\ms{F}$ w.r.t. $\mf{A}$} is the direct product
        \begin{equation*}
            \check{C}^n\!\l(\mf{A},\ms{F}\r)\coloneqq\prod_{\tpl{i_0,\dots,i_n}}\ms{F}\l(U_{i_0}\cap\cdots\cap U_{i_n}\r).
        \end{equation*}
    \end{definition}
    \begin{remark}
        For $n=0$, the group $\check{C}^0\!\l(\mf{A},\ms{F}\r)$ contains all tuples $\tpl{f_i}$ where each $f_i$ is defined on $U_i$. For $n=1$, the group $\check{C}^1\!\l(\mf{A},\ms{F}\r)$ contains all tuples $\tpl{f_{ij}}$ where each $f_{ij}$ is defined on the pairwise intersection $U_i\cap U_j$. The following discussion formalizes our rough intuition of `chaining' open sets whose `boundary'\side{Here, `boundary' is not the topological boundary.} are their pairwise intersections.\exqed
    \end{remark}
    \begin{definition}
        For all $n\in\N$, the \uldef{$n^\textrm{th}$ coboundary operator w.r.t. $\mf{A}$} is the map
        \begin{equation*}
            \delta^n\!:\check{C}^n\!\l(\mf{A},\ms{F}\r)\to\check{C}^{n+1}\!\l(\mf{A},\ms{F}\r)\ \ \ \ \textrm{\it{mapping}}\ \ \ \ \l(f_{i_0,\dots,i_n}\r)\mapsto\l(g_{i_0,\dots,i_{n+1}}\r)
        \end{equation*}
        where
        \vspace{-0.05in}
        \begin{equation*}
            g_{i_0,\dots,i_{n+1}}\coloneqq\sum_{k=0}^{n+1}\l(-1\r)^k\rho\l(f_{i_0,\dots,\widehat{i_k},\dots,i_{n+1}}\r).
        \end{equation*}
        Define the \uldef{$n^\textrm{th}$ cocycle} $\check{Z}^n\!\l(\mf{A},\ms{F}\r)\coloneqq\ker\delta^n$ and the \uldef{$n^\textrm{th}$ splitting cocycle} $\check{B}^n\!\l(\mf{A},\ms{F}\r)\coloneqq\im\delta^{n-1}$, whose quotient
        \begin{equation*}
            \check{H}^n\!\l(\mf{A},\ms{F}\r)\coloneqq\check{Z}^n\!\l(\mf{A},\ms{F}\r)/\check{B}^n\!\l(\mf{A},\ms{F}\r)
            \vspace{-0.05in}
        \end{equation*}
        is called the \uldef{$n^\textrm{th}$ cohomology group of $\ms{F}$ w.r.t. $\mf{A}$}.
    \end{definition}
    \side[-0.9in]{The `hat' notation represents a deletion. Also, $\rho$ is the appropriate restriction mapping of $\ms{F}$.\\\ \\
    Note that $\check{B}^0\!\l(\mf{A},\ms{F}\r)=0$ since $\check{C}^{-1}\!\l(\mf{A},\ms{F}\r)=0$.}
    \vspace{-0.05in}
    \begin{remark}
        A calculation shows that $\check{B}^n\!\l(\mf{A},\ms{F}\r)\subseteq\check{Z}^n\!\l(\mf{A},\ms{F}\r)$, so the quotient makes sense. In particular, $\delta^{n+1}\circ\delta^n=0$.\exqed
    \end{remark}
    \begin{remark}
        For $n=0$, we have $\delta^0\!\l(f_i\r)=\tpl{f_j-f_i}$ for all $\tpl{f_i}\in\check{C}^0\!\l(\mf{A},\ms{F}\r)$. This gives us a glueing condition, that if\side{Indeed, if $\tpl{f_i}\in\check{C}^0\!\l(\mf{A},\ms{F}\r)$, then $\rho\l(f_i\r)=\rho\l(f_j\r)$ for all $i,j$. Henceforth, we suppress the restriction maps $\rho$ for ease of notation, but will always mention on which domain the relation is valid on.} $\tpl{f_i}\in\check{Z}^0\!\l(\mf{A},\ms{F}\r)$, then the sheaf axioms furnish a unique $f\in\ms{F}\l(X\r)$ such that $\rho^X_{U_i}\!\l(f\r)=f_i$ for all $i$. Thus
        \begin{equation*}
            \check{H}^0\!\l(\mf{A},\ms{F}\r)=\check{Z}^0\!\l(\mf{A},\ms{F}\r)\iso\ms{F}\l(X\r),
        \end{equation*}
        so $\check{H}^0\!\l(\mf{A},\ms{F}\r)$ is independent of the covering $\mf{A}$ and we may define the \ul{$0^\textrm{th}$ cohomology group of $\ms{F}$} as $\check{H}^0\!\l(X,\ms{F}\r)\coloneqq\ms{F}\l(X\r)$.\exqed
    \end{remark}
    \begin{remark}
        For $n=1$, we have $\delta^1\!\l(f_{ij}\r)=\tpl{f_{jk}-f_{ik}+f_{ij}}$ for all $\tpl{f_{ij}}\in\check{C}^1\!\l(\mf{A},\ms{F}\r)$. Elements $\tpl{f_{ij}}\in\check{Z}^1\!\l(\mf{A},\ms{F}\r)$ satisfy the \textit{cocycle condition}, which states
        \begin{equation*}
            f_{ik}=f_{ij}+f_{jk}
        \end{equation*}
        on $U_i\cap U_j\cap U_k$ for all $i,j,k$. In\side{The construction of the $1^\textrm{st}$ cohomology group $\check{H}^1\!\l(X,\ms{F}\r)$ is more involved. We will not need the general construction for $\check{H}^n\!\l(X,\ms{F}\r)$, but they can be done similarly as in the $n=1$ case. With some machinery, we can also define the $n^\textrm{th}$ \textit{De Rham} and \textit{Dolbeault} cohomology groups, which relate to $\check{H}^n\!\l(X,\ms{F}\r)$. See \cite[][Section IX.4]{miranda}.} particular, it implies that $f_{ii}=0$ for all $i$ and $f_{ij}=-f_{ji}$ on $U_i\cap U_j$ for all $i,j$. Note that every splitting cocycle is a cocycle, but not every cocycle splits. In other words, $\check{H}^1\!\l(\mf{A},\ms{F}\r)$ measures how $1$-cocycles fail to split. The next section defines the $1^\textrm{st}$ cohomology group of $\ms{F}$, independent of the covering $\mf{A}$.\exqed
    \end{remark}
    \subsection{Refinements and $\check{H}^1\!\l(X,\ms{F}\r)$}
    In this section, we specialize to when $n=1$.
    \begin{definition}
        Let $\mf{A}\coloneqq\l\{U_i\r\}_{i\in I}$ and $\mf{B}\coloneqq\l\{V_k\r\}_{k\in K}$ be open coverings of $X$. We say that \uldef{$\mf{B}$ is finer than $\mf{A}$}, and write $\mf{B}\preceq\mf{A}$, if there exists a \uldef{refining map} $r:K\to I$ such that $V_k\subseteq U_{r\l(k\r)}$ for all $k\in K$.
    \end{definition}
    \begin{remark}
        The refining map $r$ lifts to a map $\tilde{r}:\check{Z}^1\!\l(\mf{A},\ms{F}\r)\to\check{Z}^1\!\l(\mf{B},\ms{F}\r)$ by sending $\tpl{f_{ij}}$ into $\tpl{g_{kl}}$ defined by $g_{kl}\coloneqq f_{r\l(k\r),r\l(l\r)}$ on $V_k\cap V_l$ for all $k,l\in K$. Observe that if $\tpl{f_{ij}}\in\check{B}^1\!\l(\mf{A},\ms{F}\r)$, then $\delta^1_\mf{A}\!\l(f_{ij}\r)=0$ and hence $f_{i_1i_3}=f_{i_1i_2}+f_{i_2i_3}$ on $U_{i_1}\cap U_{i_2}\cap U_{i_3}$ for all $i_1,i_2,i_3\in I$. In particular, we have
        \begin{equation*}
            f_{r\l(k_1\r),r\l(k_3\r)}=f_{r\l(k_1\r),r\l(k_2\r)}+f_{r\l(k_2\r),r\l(k_3\r)}
        \end{equation*}
        on $V_{k_1}\cap V_{k_2}\cap V_{k_3}$ for all $k_1,k_2,k_3\in K$ and hence $\delta^1_\mf{B}\!\l(\tilde{r}\l(f_{ij}\r)\r)=0$. Thus $\tilde{r}\l(f_{ij}\r)\in\check{B}^1\!\l(\mf{B},\ms{F}\r)$ and so $\tilde{r}$ sends splitting cocycles into splitting cocycles. Hence we may descent $\tilde{r}$ into the quotient, giving us a map
        \begin{equation*}
            \check{H}\l(r\r):\check{H}^1\!\l(\mf{A},\ms{F}\r)\to\check{H}^1\!\l(\mf{B},\ms{F}\r)\ \ \ \ \textrm{mapping}\ \ \ \ \l[f_{ij}\r]\mapsto\l[\tilde{r}\l(f_{ij}\r)\r].\exqedin
        \end{equation*}
    \end{remark}
    \begin{proposition}\label{3.2:prp:lifted_refinement_independent_of_refining_map}
        In the above notation, the map $\check{H}^\mf{A}_\mf{B}\coloneqq\check{H}\l(r\r)$ is independent of $r$ and is injective.
    \end{proposition}
    \begin{proof}
        Take $\tpl{f_{ij}}\in\check{Z}^1\!\l(\mf{A},\ms{F}\r)$ and suppose that $r':K\to I$ is another refining map. Lifting it to $\tilde{r}'$ similarly, let $\tpl{g_{kl}}\coloneqq\tilde{r}\l(f_{ij}\r)=\tpl{f_{r\l(k\r),r\l(l\r)}}$ and $(g'_{kl})\coloneqq\tilde{r}'\l(f_{ij}\r)=\tpl{f_{r'\l(k\r),r'\l(l\r)}}$. Observe then that
        \begin{equation*}
            \begin{aligned}
                g_{kl}-g_{kl}'&=f_{r\l(k\r),r\l(l\r)}-f_{r'\l(k\r),r'\l(l\r)} \\
                              &=f_{r\l(k\r),r\l(l\r)}+f_{r\l(l\r),r'\l(k\r)}-f_{r\l(l\r),r'\l(k\r)}-f_{r'\l(k\r),r'\l(l\r)} \\
                              &=f_{r\l(k\r),r'\l(k\r)}-f_{r\l(l\r),r'\l(l\r)}
            \end{aligned}
        \end{equation*}
        on $V_k\cap V_l$ for all $k,l\in K$. Since $r$ and $r'$ are refining maps, we see that $V_k\subseteq U_{r\l(k\r)}\cap U_{r'\l(k\r)}$ for all $k\in K$, so we may define $h_k\coloneqq f_{r\l(k\r),r'\l(k\r)}$ on the restriction to $V_k$. Then
        \begin{equation*}
            \tpl{g_{kl}-g_{kl}'}=\tpl{h_k-h_l}=\delta^0\!\l(h_k\r)
        \end{equation*}
        on $V_k\cap V_l$, so $\tpl{g_{ij}}-(g_{ij}')\in\check{B}^1\!\l(\mf{B},\ms{F}\r)$. Thus their equivalence classes coincide, as desired. Now, to show that $\check{H}_\mf{B}^\mf{A}$ is injective, take $\tpl{f_{ij}}\in\ker\check{H}_\mf{B}^\mf{A}$. Thus $\tpl{f_{r\l(k\r),r\l(l\r)}}=\check{H}_\mf{B}^\mf{A}\!\l(f_{ij}\r)$ splits, so there exist $g_k\in\ms{F}\l(V_k\r)$ such that $f_{r\l(k\r),r\l(l\r)}=g_k-g_l$ on $V_k\cap V_l$ for all $k,l\in K$. Then
        \begin{equation*}
            g_k-g_l=f_{r\l(k\r),i}+f_{i,r\l(l\r)}=f_{i,r\l(l\r)}-f_{i,r\l(k\r)}
        \end{equation*}
        on $U_i\cap V_k\cap V_l$ for all $i\in I$ and hence $g_k+f_{i,r\l(k\r)}=g_l+f_{i,r\l(l\r)}$ on the same domain. Fixing $i\in I$ and glueing the family $\l\{g_k+f_{i,r\l(k\r)}\r\}_{k\in K}$ defined on the cover $\l\{U_i\cap V_k\r\}_{k\in K}$ of $U_i$, we obtain an element $h_i\in\ms{F}\l(U_i\r)$ such that $h_i=g_k+f_{i,r\l(k\r)}$ on $U_i\cap V_k$ for all $k\in K$. Observe then that
        \begin{equation*}
            f_{ij}=f_{i,r\l(k\r)}-f_{j,r\l(k\r)}=h_i-g_k-h_j+g_k=h_i-h_j
        \end{equation*}
        on $U_i\cap U_j\cap V_k$. Note that both $f_{ij}$ and $h_i-h_j$ are defined on $U_i\cap U_j$, and since they coincide on the restriction to $V_k$, uniqueness of the glueing gives us $f_{ij}=h_i-h_j$ on $U_i\cap U_j$. Thus $\tpl{f_{ij}}=\delta^0\!\l(h_i\r)$, so $\tpl{f_{ij}}$ splits.\qed
    \end{proof}
    \begin{remark}
        If $\mf{C}\preceq\mf{B}\preceq\mf{A}$ are open coverings of $X$, we have that $\check{H}^\mf{B}_\mf{C}\circ\check{H}^\mf{A}_\mf{B}=\check{H}^\mf{A}_\mf{C}$. This\side{Both constructions are special cases of the so-called \textit{direct limit}; see \cite[][Chapter III]{maclane}.} allows us to give a construction of $\check{H}^1\!\l(X,\ms{F}\r)$ similar to that of Definition \ref{2.2:def:stalk}.\exqed
    \end{remark}
    \begin{definition}
        The \uldef{$1^\textrm{st}$ cohomology group of $\ms{F}$} is the Abelian group
        \begin{equation*}
            \check{H}^1\!\l(X,\ms{F}\r)\coloneqq\raisebox{-2pt}{$\Biggl(\!$}\bigsqcup_{\mf{A}}\check{H}^1\!\l(\mf{A},\ms{F}\r)\!\raisebox{-2pt}{$\Biggr)$}\!\!_{\scalebox{2}{/}\sim}
        \end{equation*}
        where $\sim$ is the equivalence relation on the disjoint union, defined, for all $\xi\in\check{H}^1\!\l(\mf{A},\ms{F}\r)$ and $\xi'\in\check{H}^1\!\l(\mf{A}',\ms{F}\r)$, by $\xi\sim\xi'$ iff there exists a refinement $\mf{B}\preceq\mf{A},\mf{A}'$ such that $\check{H}^\mf{A}_\mf{B}\!\l(\xi\r)=\check{H}^{\mf{A}'}_\mf{B}\!\l(\xi'\r)$.
    \end{definition}
    \side[-1.08in]{Note that $\check{H}^1\!\l(X,\ms{F}\r)$ vanishes iff $\check{H}^1\!\l(\mf{A},\ms{F}\r)=0$ for all open coverings $\mf{A}$ of $X$. Indeed, the converse direction is trivial. For the forward, let $\mf{A}$ be an open covering of $X$. By Proposition \ref{3.2:prp:lifted_refinement_independent_of_refining_map}, the canonical map $\check{H}^1\!\l(\mf{A},\ms{F}\r)\to\check{H}^1\!\l(X,\ms{F}\r)$ is injective. The result follows.}
    \vspace{-0.05in}
    \begin{proposition}
        Let $X$ be a Riemann surface and consider the sheaf of differentiable functions $\DIFF$ on $X$. Then $\check{H}^1\!\l(X,\DIFF\r)=0$.
    \end{proposition}
    \begin{proof}
        Let $\mf{A}\coloneqq\l\{U_i\r\}_{i\in I}$ be an open covering of $X$ and let $\tpl{f_{ij}}\in\check{Z}^1\!\l(X,\DIFF\r)$ be a cocycle; it suffices to show that $\tpl{f_{ij}}$ splits, for then $\check{H}^1\!\l(\mf{A},\DIFF\r)=0$ and we are done by the remark above. To do so, we use the fact that there exists\side{For a proof, see \cite[][Theorem 2.23]{lee}. Note that the functions $\psi_i$ are \textit{not necessarily} holomorphic.} a partition of unity subordinate to $\mf{A}$; that is, a family $\l\{\psi_i\r\}_{i\in I}$ of differentiable functions such that:
        \begin{itemize}
            \item $\Supp\l(\psi_i\r)\coloneqq\bar{\l\{p\in X\mid\psi\l(p\r)\neq0\r\}}\subseteq U_i$ for every $i\in I$.
                \vspace{-0.05in}
            \item Every point in $X$ admits a neighborhood whose intersection with $\l\{\Supp\l(\psi_i\r)\r\}_{i\in I}$ is finite.
                \vspace{-0.05in}
            \item $\sum_{i\in I}\psi_i=1$.
        \end{itemize}
        Consider the function $\psi_jf_{ij}$ on $U_i\cap U_j$, which may be differentiably extended to $U_i$ by zero outside $\Supp\l(\psi_j\r)$. Consider the function $g_i\coloneqq\sum_{j\in I}\psi_jf_{ij}\in\DIFF\l(U_i\r)$, which is legal since there is a neighborhood around every point of $U_i$ such that $\psi_jf_{ij}=0$ for all but finitely-many $j\in I$. Observe that
        \begin{equation*}
            g_i-g_j=\sum_{k\in I}\psi_k\l(f_{ik}-f_{jk}\r)=\sum_{k\in I}\psi_k\l(f_{ik}+f_{kj}\r)=\sum_{k\in I}\psi_kf_{ij}=f_{ij}
        \end{equation*}
        on $U_i\cap U_j$, so $\tpl{f_{ij}}=\tpl{g_i-g_j}=\delta^0\!\l(g_i\r)$ splits.\qed
    \end{proof}
\end{document}
