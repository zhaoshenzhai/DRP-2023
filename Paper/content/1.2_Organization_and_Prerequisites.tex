\documentclass[../Moduli_Spaces_of_Riemann_Surfaces.tex]{subfiles}

\begin{document}
    \section{Organization and Prerequisites}
    We give a brief overview of the organization of this paper.
    \begin{itemize}
        \item Chapter \ref{RS} begins with some definitions and constructions relating to Riemann surfaces and introduces the main examples of interest to this paper: the Riemann sphere and complex tori. We then study the basic behaviours of maps between Riemann surfaces, with a focus on meromorphic functions and the relation to their associated holomorphic maps.
        \item Chapter \ref{CS} studies the covering space theory of Riemann surfaces, with the goal of defining the degree of a proper holomorphic map. This reduces the problem of showing that a simply-connected compact Riemann surface $X$ is biholomorphic to the Riemann sphere to furnishing a global meromorphic function on $X$.
        \item Chapter \ref{CC} builds up sheaf theory and their associated cohomology to prove the existence of such a meromorphic function. The theory of (complex) differential forms is then introduced to study the sheaf of holomorphic functions on the Riemann sphere, where we prove the existence of such a meromorphic function on $X$.
        \item Chapter \ref{MS} ties everything together and uses the tools developed to compute the moduli space of genus $0$ and $1$ surfaces (corresponding to the Riemann sphere and complex tori). This chapter closes with a brief discussion of the Uniformization Theorem and its impacts on the theory of Riemann surfaces.
    \end{itemize}
    To make this paper more accessible, we try to keep the prerequisites at a minimum. However, familiarity with topology and complex analysis is required, and an exposure to the theory of real manifolds is nice to have (but not essential). We also assume that the reader is comfortable with some linear algebra and basic group theory. A more complete list of prerequisites, along with references, will be given at the start of each chapter.
\end{document}
