\documentclass[../Moduli_Spaces_of_Riemann_Surfaces.tex]{subfiles}
\begin{document}
    This chapter assumes that the reader is familiar with the basic notions of liftings and homotopy of curves from algebraic topology, for which we refer the reader to \cite[][Chapter 1]{hatcher}.
    \section{Covering Maps and the Degree}
    We devote this section to develop the tools necessary to define the \textit{degree} of a proper holomorphic map, which, intuitively, is the \textit{number of sheets} in which it covers its image. However, there are points in the image which are not covered `correctly', and they must be taken care of separately.\\\ \\
    Using the theory of degrees, we prove a criterion for a compact Riemann surface $X$ to be biholomorphic to the Riemann sphere $\RS$, which will be used in Section \ref{sec:moduli_space_of_sphere} to calculate the moduli space of $\RS$.
    \subsection{Ramification and Critical Points}
    \begin{definition}
        Let $X$ and $Y$ be Riemann surfaces and let $F:X\to Y$ be a non-constant holomorphic map. A point $p\in X$ is said to be a \uldef{ramification point of $F$} if $\l.F\r|_U$ is not injective for any neighborhood $U$ of $p$, in which case $F\l(p\r)\in X$ is said to be a \uldef{critical value of $F$}. If $F$ has no ramification points, then $F$ is said to be an \uldef{unbranched holomorphic map}.
    \end{definition}
    \side[-0.56in]{It is immediate that $F$ is unbranched iff it is a local homeomorphism. Indeed, if $F$ is unbranched, then for every $p\in X$ there exists a neighborhood $U$ of $p$ such that $\l.F\r|_U$ is injective. By the Open Mapping Theorem, $F$ is open and hence $\l.F\r|_U$ maps $U$ homeomorphically to the open set $F\l(U\r)$. Conversely, if $F$ is a local homeomorphism, then for every $p\in X$ there exists a neighborhood $U$ of $p$ that is mapped homeomorphically onto an open set in $Y$. In particular, $\l.F\r|_U$ is injective, so $F$ is unbranched at $p$.}
    \begin{proposition}\label{2.1:prp:ramification_iff_mult_2}
        Let $X$ and $Y$ be Riemann surfaces and fix $p\in X$. A non-constant holomorphic map $F:X\to Y$ has a ramification point at $p$ iff $\mult_p\!\l(F\r)\geq2$.
    \end{proposition}
    \begin{proof}
        By Theorem \ref{1.2:thm:local_normal_form}, there exist charts $\tpl{U,\phi}$ centered at $p$ and $\tpl{V,\psi}$ centered at $F\l(p\r)$ such that $f\coloneqq\psi\circ F\circ\phi^{-1}$ is the power map $z\mapsto z^m$ where $m\coloneqq\mult_p\!\l(F\r)$. Since $\phi$ and $\psi$ are, in particular, injections, we see that $F$ is locally injective at $p$ iff $f$ is locally injective at $0$. But this occurs precisely when $m=\mult_p\!\l(F\r)<2$, so the result follows.\qed
    \end{proof}
    \begin{example}\label{2.1:exa:covering_of_torus}
        For any lattice $\Gamma\subseteq\C$ the projection $\pi:\C\to\C/\Gamma$ is an unbranched holomorphic map. This follows from our construction of complex tori in Example \ref{1.1:exa:tori} where for every $z\in\C$ a small enough neighborhood $U$ was found so that $\l.\pi\r|_U$ is injective.\exqed
    \end{example}
    \begin{proposition}
        Let $X$, $Y$ and $Z$ be Riemann surfaces and let $F:X\to Y$ be a holomorphic map. Then any lifting $\tilde{F}:X\to Z$ of $F$ w.r.t. an unbranched holomorphic map $\pi:Z\to Y$ is a holomorphic map.
    \end{proposition}
    \side[-0.44in]{Recall that a continuous map $\tilde{F}$ is a \ul{lifting of $F$} \ul{w.r.t. $P$} if the diagram
        \begin{equation*}
            \begin{tikzcd}[ampersand replacement=\&]
                \& Z \ar[d, "\pi"] \\
                X \ar[ur, "\tilde{F}"] \ar[r, "F"'] \& Y
            \end{tikzcd}
        \end{equation*}
        commutes.}
    \vspace{-0.1in}
    \begin{proof}
        Take $p\in X$ and set $r\coloneqq\tilde{F}\l(p\r)$ and $q\coloneqq\pi\l(r\r)=F\l(p\r)$. Since $\pi$ is unbranched, there exists a neighborhood $W$ of $r$ such that $\l.\pi\r|_W:W\to Y$ is holomorphic, so it is biholomorphic onto its image $V\coloneqq\pi\l(W\r)$. Let $\cchi\coloneqq\l.\pi\r|_W^{-1}:V\to W $. Since $\tilde{F}$ is continuous, its inverse image $U\coloneqq\tilde{F}^{-1}\l(W\r)$ is open. Observe that
        \begin{equation*}
            \l.F\r|_U=(\pi\circ\tilde{F})|_U=\l.\pi\r|_W\circ\tilde{F}|_U,
        \end{equation*}
        so $\tilde{F}|_U=\cchi\circ\l.F\r|_U$. Then $p\in U$ and $\tilde{F}|_U$ is a composition of two holomorphic maps, so $\tilde{F}$ is holomorphic at $p$.\qed
    \end{proof}
    \subsection{Proper and Covering Maps}
    In this section, we gather some basic results on the theory of covering maps from topology. Throughout this section and the next, $E$ and $X$ are locally-compact topological spaces.\side[0.28in]{The assumption that $E$ and $X$ are locally compact ensures that all proper maps are closed; that is, then send closed sets to closed sets.}
    \begin{definition}
        A map $\pi:E\to X$ is said to be \uldef{proper} if the preimage of every compact set is compact.
    \end{definition}
    \begin{proposition}\label{1.3:prp:proper_give_neighborhoods}
        Let $\pi:E\to X$ be a proper map. Then for every $p\in X$ and every neighborhood $V$ of $\pi^{-1}\l(p\r)$, there exists a neighborhood $U$ of $p$ such that $\pi^{-1}\!\l(U\r)\subseteq V$.
    \end{proposition}
    \begin{proof}
        Since $V$ is open, the set $E\comp V$ is closed. Since $\pi$ is proper, it is closed and hence $\pi\l(E\comp V\r)$ is closed too. Clearly $p\not\in\pi\l(E\comp V\r)\eqqcolon W$, so $U\coloneqq E\comp W$ is a neighborhood of $p$; we claim that $\pi^{-1}\l(U\r)\subseteq V$. Indeed, for all $\pi\l(e\r)\in U$, we see that $\pi\l(e\r)\not\in\pi\l(E\comp V\r)$ and so $e\not\in E\comp V$.\qed
    \end{proof}
    \begin{definition}
        A map $\pi:E\to X$ is said to be a \uldef{covering map} if every point $p\in X$ has a neighborhood $U$ such that $\pi^{-1}\l(U\r)=\bigcup_{j\in J}V_j$ where $V_j$ are disjoint open sets in $E$, each homeomorphic to $U$ via $\l.\pi\r|_{V_j}$.
    \end{definition}
    \begin{example}
        Let $m\geq2$ be a natural number and consider the power map $f:\C^\ast\to\C^\ast$ mapping $z\mapsto z^m$. We claim that $f$ is a covering map, so take $b\in\C^\ast$ and let $a\in\C^\ast$ be any one of its $m^\textrm{th}$ roots. Since $f$ is a unbranched, there exist neighborhoods $V_0$ of $a$ and $U$ of $b$ such that $\l.f\r|_{V_0}:V_0\to U$ is a homeomorphism. It is clear then that\side{Indeed, for all $c\in f^{-1}\l(U\r)$, $f\l(c\r)\in U$ and so there exists some $a'\in V_0$ such that $f\l(a'\r)=f\l(c\r)$. Then $c=\omega^ja'$ for some $0\leq j\leq m-1$, so $c\in\omega^jV_0$. Conversely, if $c\in\omega^jV_0$ for some $0\leq j\leq m-1$, then $c=\omega^ja'$ for some $a'\in V_0$ and hence $f\l(c\r)=f\l(\omega^ja'\r)=f\l(a'\r)\in U$.}
        \vspace{-0.05in}
        \begin{equation*}
            f^{-1}\l(U\r)=\bigcup_{j=0}^{m-1}\omega^jV_0,
        \end{equation*}
        where $\omega$ is an $m^\textrm{th}$ root of unity, and since $f^{-1}\l(b\r)$ is discrete, the sets $V_j\coloneqq\omega^jV_0$ can be made small enough so that they are pairwise disjoint. Then each $\l.f\r|_{V_j}:V_j\to U$ is a homeomorphism, as desired.\exqed
    \end{example}
    \begin{example}
        For any lattice $\Gamma\subseteq\C$, the projection $\pi:\C\to\C/\Gamma$ is a covering map. Indeed, take $z+\Gamma\in\C/\Gamma$ and let $w\in\C$ be such that $\pi\l(w\r)=z+\Gamma$. Since $\pi$ is unbranched, there exist neighborhoods $V$ of $w$ and $U$ of $z+\Gamma$ such that $\l.\pi\r|_{V}:V\to U$ is a homeomorphism. Then clearly\side{Similarly, for all $z\in\pi^{-1}\!\l(U\r)$, $\pi\l(z\r)\in U$ and so there exists some $w'\in V$ such that $\pi\l(z\r)=\pi\l(w'\r)$. Then $z+\Gamma=w'+\Gamma$, so $z=w'+\lambda$ for some $\lambda\in\Gamma$. Conversely, if $z\in\lambda+V$ for some $\lambda\in\Gamma$, then $z=w'+\lambda$ for some $w'\in V$ and hence $\pi\l(z\r)=\pi\l(\omega'+\lambda\r)=\pi\l(w\r)\in U$.}
        \begin{equation*}
            \pi^{-1}\l(U\r)=\bigcup_{\lambda\in\Gamma}\l(\lambda+V\r)
        \end{equation*}
        where the sets $V_\lambda\coloneqq\lambda+V$ are all disjoint and each $\l.\pi\r|_{V_\lambda}:V_\lambda\to U$ is a homeomorphism.\exqed
    \end{example}
    \begin{proposition}\label{2.1:prp:proper_local_covering}
        Any proper local homeomorphism is a covering map.
    \end{proposition}
    \begin{proof}
        Let $\pi:E\to X$ be a proper local homeomorphism and take $p\in X$. We claim that $\pi^{-1}\l(p\r)$ is finite.
        \begin{itemize}
            \item For each $e\in\pi^{-1}\l(p\r)$, there exist neighborhoods $W_e$ of $e$ and $U$ of $p$ such that $\l.\pi\r|_{W_e}:W_e\to U$ is a homeomorphism. Then the sets $W_e$ must be disjoint, for if $e'\in W_e$ for some $e'\neq e$, then $\l.\pi\r|_{W_e}\!\l(e\r)=p=\l.\pi\r|_{W_e}\!\l(e'\r)$, contradicting that $\l.\pi\r|_{W_e}$ is a homeomorphism. Thus $\pi^{-1}\l(p\r)$ must be finite, lest the cover $\l\{W_e\r\}$ admits no finite subcover.
        \end{itemize}
        Thus $\pi^{-1}\l(p\r)=\l\{e_1,\dots,e_n\r\}$ for some $e_j\in E$. Letting $W_j\coloneqq W_{e_j}$ as above, we see that $\bigcup_{j=1}^{n}W_j$ is a neighborhood of $\pi^{-1}\l(p\r)$. By Proposition \ref{1.3:prp:proper_give_neighborhoods}, there is a neighborhood $U$ of $p$ such that $\pi^{-1}\l(U\r)\subseteq\bigcup_{j=1}^{n}W_j$, so $\pi^{-1}\l(U\r)=\bigcup_{j=1}^{n}V_j$ where the sets $V_j\coloneqq W_j\cap\pi^{-1}\l(U\r)$ are all disjoint and each $\l.\pi\r|_{V_j}:V_j\to U$ is a homeomorphism.\qed
    \end{proof}
    \subsection{Liftings of Curves}
    This section develops some technical tools to define the \textit{number of sheets} of a covering, which in turn is used to define the \textit{degree} of a proper holomorphic map.
    \begin{definition}
        A function $\pi:E\to X$ is said to have the \uldef{curve lifting property} if for every curve $\alpha:\l[0,1\r]\to X$ and every point $e_0\in E$ with $\pi\l(e_0\r)=\alpha\l(0\r)$, there exists a lifting $\tilde{\alpha}:\l[0,1\r]\to E$ w.r.t. $\pi$ such that $\tilde{\alpha}\l(0\r)=e_0$.
    \end{definition}
    \begin{proposition}
        Every covering map $\pi:E\to X$ has the curve lifting property.
    \end{proposition}
    \begin{proof}
        Let\side{The idea of this proof is to split $\alpha\l(\l[0,1\r]\r)$ into (overlapping) paths $\alpha\l(\l[t_{k-1},t_k\r]\r)$, each of which is an open set, and construct the lifting $\tilde{\alpha}$ inductively: Given a lifting $\tilde{\alpha}$ defined up to some boundary $t_{k-1}$, we define it on the next interval $\l[t_{k-1},t_k\r]$ by lifting $\alpha$ (restricted to $\l[t_{k-1},t_k\r]$) via $\cchi$. This gives us a `chain' of paths, which when joined together gives us a global lifting of $\alpha$.\\\ \\
        The base case of this induction simply sets $\tilde{\alpha}\l(0\r)\coloneqq e_0$ in order to start-off this process.} $\alpha:\l[0,1\r]\to X$ be a curve and let $e_0\in E$ be a point such that $\pi\l(e_0\r)=\alpha\l(0\r)$. Consider any open cover $\l\{U_i\r\}$ of $\alpha\l(\l[0,1\r]\r)$ where each $U_i$ is a connected open set in $\alpha\l(\l[0,1\r]\r)$. Thus $\l\{\alpha^{-1}\!\l(U_i\r)\r\}$ is an open cover of $\l[0,1\r]$, so it admits a finite subcover $\l\{\l(t_i,t_{i+1}\r)\r\}_{i=1}^n\coloneqq\l\{\alpha^{-1}\l(U_i\r)\r\}_{i=1}^n$. Reindexing if necessary, we obtain a partition
        \begin{equation*}
            0\eqqcolon t_0<t_1<\cdots<t_n\coloneqq1
        \end{equation*}
        of $\l[0,1\r]$ such that $\alpha\l(\l[t_{i-1},t_i\r]\r)\subseteq U_i$ for all $1\leq i\leq n$. Now, since $\pi$ is a covering map, there exist disjoint open sets $V_{ij}$ in $E$, each homeomorphic to $U_i$ via $\l.\pi\r|_{V_{ij}}$, such that $\pi^{-1}\!\l(U_i\r)=\bigcup_{j\in J_i}V_{ij}$. We now construct a lifting $\l.\tilde{\alpha}\r|_{\l[0,t_k\r]}:\l[0,t_k\r]\to E$ by induction on $k\in\N$.
        \begin{itemize}
            \item The base case for when $k=0$ is trivial by defining $\tilde{\alpha}\l(0\r)\coloneqq e_0$.
        \end{itemize}
        Suppose now that the lifting $\l.\tilde{\alpha}\r|_{\l[0,t_{k-1}\r]}:\l[0,t_{k-1}\r]\to E$ has been constructed for some $k\geq1$. Then\side{
            \begin{equation*}
                \begin{tikzcd}[ampersand replacement=\&]
                    \& E \ar[d, "\pi"] \\
                    \l[0,1\r] \ar[r, "\alpha"'] \ar[ur, "\tilde{\alpha}"] \& X
                \end{tikzcd}
            \end{equation*}
        } $\alpha\l(t_{k-1}\r)=\pi\l(\tilde{\alpha}\l(t_{k-1}\r)\r)\in U_k$, so there exists some $j\in J_k$ such that $\tilde{\alpha}\l(t_{k-1}\r)\in V_{kj}$. Letting $\cchi:U_k\to V_{kj}$ be the inverse of $\l.\pi\r|_{V_{kj}}:V_{kj}\to U_k$, we set
        \begin{equation*}
            \l.\tilde{\alpha}\r|_{\l[t_{k-1},t_k\r]}\coloneqq\cchi\circ\l.\alpha\r|_{\l[t_{k-1},t_k\r]}.
        \end{equation*}
        Clearly\side{$\tilde{\alpha}\l(t_{k-1}\r)=\cchi\l(\alpha\l(t_{k-1}\r)\r)=\cchi\l(\pi\l(\tilde{\alpha}\l(t_{k-1}\r)\r)\r)$ on the appropriate restrictions.}, $\tilde{\alpha}\!\l(t_{k-1}\r)$ agrees with our existing lifting, which makes the piecewise-defined map $\l.\alpha\r|_{\l[0,t_k\r]}$ a lifting of $\l.\alpha\r|_{\l[0,t_k\r]}$ w.r.t. $\pi$.\qed
    \end{proof}
    \begin{corollary}\label{2.1:cor:fiber_cardinalities_coincide}
        Suppose that $X$ is path-connected and let $\pi:E\to X$ be a covering map. Then, for any $p_1,p_2\in X$, the sets $\pi^{-1}\!\l(p_1\r)$ and $\pi^{-1}\!\l(p_2\r)$ are equinumerous.
    \end{corollary}
    \begin{proof}
        Since $X$ is path-connected, there exists a curve $\alpha:\l[0,1\r]\to X$ from $p_1$ to $p_2$. We define a map $\phi:\pi^{-1}\!\l(p_1\r)\to\pi^{-1}\!\l(p_2\r)$ as follows. Every $e\in\pi^{-1}\!\l(p_1\r)$ induces a unique lifting $\tilde{\alpha}:\l[0,1\r]\to E$ such that $\tilde{\alpha}\l(0\r)=e$, and since $\pi\l(\tilde{\alpha}\l(1\r)\r)=\alpha\l(1\r)=p_2$, we have $\tilde{\alpha}\l(1\r)\in\pi^{-1}\!\l(p_2\r)$. Hence we define $\phi\l(e\r)\coloneqq\tilde{\alpha}\l(1\r)$. The uniqueness of liftings ensures that $\phi$ is well-defined and bijective, so $\pi^{-1}\!\l(p_1\r)$ and $\pi^{-1}\!\l(p_2\r)$ are equinumerous.\qed
    \end{proof}
    \subsection{Degrees and Multiplicities}
    Throughout this section, $X$ and $Y$ are Riemann surfaces and $F:X\to Y$ is a non-constant proper holomorphic map.
    \begin{definition}
        The \uldef{degree of $F$}, denoted $\deg F$, is the cardinality of the fiber $F^{-1}\!\l(q\r)$ of any non-critical point $q\in Y$.
    \end{definition}
    \begin{proof}
        (Well-definition): Since $F$ is a proper map, the fiber $F^{-1}\!\l(q\r)$ is compact and is hence finite by Discreteness of Preimages. Being unramified, we see that $F$ is a local homeomorphism, so it is a covering map by Proposition \ref{2.1:prp:proper_local_covering}. Finally, Corollary \ref{2.1:cor:fiber_cardinalities_coincide} shows that $\deg F$ is well-defined.\qed
    \end{proof}
    \begin{remark}
        Let $n\coloneqq\deg F$. Then $n$ is referred to as the \ul{number of sheets of $F$} and $F$ is said to be an \ul{$n$-sheeted holomorphic covering map}.\exqed
    \end{remark}
    \begin{theorem}
        Fix an arbitrary $q\in Y$. Then $\deg F$ is the sum of the multiplicities at each $p\in F^{-1}\!\l(q\r)$ of $F$. That is,
        \begin{equation*}
            \deg F=\sum_{p\in F^{-1}\!\l(q\r)}\mult_p\!\l(F\r).
        \end{equation*}
    \end{theorem}
    \side[-0.73in]{Instead of simply counting the elements in the fiber, we need count them \textit{with multiplicities}.}
    \vspace{-0.1in}
    \begin{proof}
        If $q$ is not a critical point, then Proposition \ref{2.1:prp:ramification_iff_mult_2} shows that $\mult_p\!\l(F\r)=1$ for any $p\in F^{-1}\!\l(p\r)$. Then $\deg F=\l|F^{-1}\!\l(q\r)\r|$, which agrees with our definition.\\\ \\
        Otherwise, $q$ is a critical point of $F$\side{Although $q$ is a critical point of $F$, every point in a small enough neighborhood around it is not a critical point.}. Since $F^{-1}\!\l(q\r)$ is compact, we see that $F^{-1}\!\l(q\r)=\l\{p_1,\dots,p_n\r\}$ for some $p_i\in X$. Fix $1\leq j\leq n$ and set $m_j\coloneqq\mult_{p_j}\!\l(F\r)$. We claim that there exist neighborhoods $U_j$ of $p_j$ and $V_j$ of $q$ such that $\l|F^{-1}\!\l(r\r)\cap U_j\r|=m_j$ for all $r\in V_j\comp\l\{q\r\}$.
        \begin{itemize}
            \item By Theorem \ref{1.2:thm:local_normal_form}, there exist charts $\tpl{U_j,\phi_j}$ of $X$ centered at $p_j$ and $\tpl{V_j,\psi_j}$ of $Y$ centered $q$ such that $F$ acts as the power function $f\l(z\r)\coloneqq z^{m_j}$ on $\phi_j\!\l(U_j\r)$. Take\side{Note that $V_j$ can be taken small enough so that $r$ is \textit{not} a critical value of $F$.} $r\in V_j\comp\l\{q\r\}$ and set $w\coloneqq\psi_j\!\l(r\r)\neq0$. Then $\l|f^{-1}\!\l(w\r)\r|=m_j$, so we have
                \begin{equation*}
                    \l|F^{-1}\!\l(r\r)\cap U_j\r|=\l|\phi_j\l(F^{-1}\l(r\r)\r)\r|=\l|\phi_j\l(F^{-1}\!\l(\psi_j^{-1}\!\l(w\r)\r)\r)\r|=\l|f^{-1}\!\l(w\r)\r|=m_j.
                \end{equation*}
        \end{itemize}
        Since $U_j$ is a neighborhood of $p_j$, we see that $F^{-1}\!\l(V_j\r)\subseteq U_j$ by restricting $V_j$ in accordance with Proposition \ref{1.3:prp:proper_give_neighborhoods}, if necessary. Then, with $V\coloneqq\bigcap_{i=1}^{n}V_i$, we see that $F^{-1}\!\l(V\r)\subseteq\bigcup_{i=1}^{n}U_i$ where the sets $U_i$ are all disjoint. Take any $r\in V\comp\l\{q\r\}$. Then $r\in V_i\comp\l\{q\r\}$ for all $1\leq i\leq n$, so
        \begin{equation*}
            \l|F^{-1}\!\l(r\r)\r|=\l|F^{-1}\!\l(r\r)\cap\bigcup_{i=1}^{n}U_i\r|=\l|\bigcup_{i=1}^{n}\l(F^{-1}\!\l(r\r)\cap U_i\r)\r|=\sum_{i=1}^{n}\l|F^{-1}\!\l(r\r)\cap U_i\r|=\sum_{i=1}^{n}m_i.
        \end{equation*}
        But $r$ is not a critical point of $F$, so the result follows.\qed
    \end{proof}
    \begin{corollary}
        If $X$ is compact, then a holomorphic map $F:X\to Y$ is a biholomorphism iff $\deg F=1$.
    \end{corollary}
    \begin{proof}
        Since $X$ is compact, we see that $F$ is proper surjection. Observe that $F$ is an injective iff it has no critical points, and by Proposition \ref{2.1:prp:ramification_iff_mult_2}, this occurs iff $\mult_p\!\l(F\r)=1$ for all $p\in X$.
        \begin{itemize}
            \item ($\Rightarrow$) If $F$ is an injection, then $\l|f^{-1}\!\l(p\r)\r|=1$ for all $p\in X$. Thus $\deg F=1$.
            \item ($\Leftarrow$): Since $\mult_p\!\l(F\r)\geq1$ for all $p\in X$, the above theorem forces $\mult_p\!\l(F\r)=1$.\qed
        \end{itemize}
    \end{proof}
    \begin{corollary}\label{2.1:cor:exists_meromorphic_implis_biholomorphic_Riemann_sphere}
        If $X$ is compact and there exists a meromorphic function $f:X\to\C$ with a single simple pole, then $X\iso\RS$.
    \end{corollary}
    \begin{proof}
        Let $f:X\to\C$ be a meromorphic function with only a simple pole at $p$ and consider its associated holomorphic map $F:X\to\RS$. By Proposition \ref{1.2:prp:multiplicity_and_order}, we see that $\mult_p\!\l(F\r)=\ord_p\!\l(f\r)=1$ and hence $p$ is unramified. Since $p$ is the only pole of $f$, we see that $\deg F=\l|F^{-1}\!\l(\infty\r)\r|=1$.\qed
    \end{proof}
\end{document}
