\documentclass[../Moduli_Spaces_of_Riemann_Surfaces.tex]{subfiles}
\begin{document}
    We develop the basics of covering space theory. More specifically, Section \ref{CS:sec:covering_maps_degree} develops the notion of the \textit{degree} to derive a criterion for a compact Riemann surface $X$ to be biholomorphic to the Riemann sphere $\RS$, and Section \ref{CS:sec:lifting_criterion} derives the \textit{Lifting Criterion} from algebraic topology. Those tools will be used in Sections \ref{MS:sec:moduli_space_of_sphere} and \ref{MS:cor:moduli_space_torus} to compute the moduli spaces of $S^2$ and $T^2$, respectively. Section \ref{CS:sec:lifting_criterion} requires some background on homotopies of curves, the fundamental group, and induced homomorphisms, for which we refer the reader to \cite[][Chapter 1]{hatcher}.
    \section{Covering Maps and the Degree}\label{CS:sec:covering_maps_degree}
    We devote this section to develop the tools necessary to define the \textit{degree} of a proper holomorphic map, which, intuitively, is the \textit{number of sheets} in which it covers its image. However, there are points in the image which are not covered `evenly', so they must be counted with multiplicity.
    \subsection{Proper and Covering Maps}
    We first gather some basic results on the theory of covering spaces from topology. Throughout this section, let $E$ and $X$ be locally-compact topological spaces. This assumption ensures that proper maps are closed.
    \begin{definition}
        A map $\pi:E\to X$ is said to be \uldef{proper} if the preimage of every compact set is compact.
    \end{definition}
    \begin{proposition}\label{CS:prp:proper_give_neighborhoods}
        For a proper map $\pi:E\to X$, every $p\in X$ and every neighborhood $V$ of $\pi^{-1}\!\l(p\r)$ admits a neighborhood $U$ of $p$ such that $\pi^{-1}\!\l(U\r)\subseteq V$.
    \end{proposition}
    \begin{proof}
        Since $E\comp V$ is closed and $\pi$ is proper, we see that $\pi\l(E\comp V\r)$ is closed too. Clearly $p\not\in\pi\l(E\comp V\r)\eqqcolon W$, so $U\coloneqq X\comp W$ is a neighborhood of $p$. Then $\pi^{-1}\!\l(U\r)\subseteq V$, since for all $\pi\l(\zeta\r)\in U$, we see that $\pi\l(\zeta\r)\not\in\pi\l(E\comp V\r)$ and so $\zeta\not\in E\comp V$.\qed
    \end{proof}
    \begin{definition}
        A map $\pi:E\to X$ is said to be a \uldef{covering map} if every point $p\in X$ admits a neighborhood $U$ such that $\pi^{-1}\!\l(U\r)=\coprod_{j\in J}V_j$ where $V_j$ are disjoint open sets in $E$, each homeomorphic to $U$ via $\l.\pi\r|_{V_j}$. In this case, we say that $U$ is \uldef{evenly-covered by $\l\{V_j\r\}$} and that $E$ is a \uldef{covering space of $X$}.
    \end{definition}
    \begin{example}\label{CS:exa:power_map}
        Let $m\geq2$ be a natural number and consider the power map $f:\C^\ast\to\C^\ast$ mapping $z\mapsto z^m$. We claim that $f$ is a covering map, so take $b\in\C^\ast$ and let $a\in\C^\ast$ be any one of its $m^\textrm{th}$ roots. Since $f$ is a local homeomorphism, there exist neighborhoods $V_0$ of $a$ and $U$ of $b$ such that $\l.f\r|_{V_0}:V_0\to U$ is a homeomorphism. We claim that
        \begin{equation*}
            f^{-1}\!\l(U\r)=\coprod_{j=0}^{m-1}\omega^jV_0,
        \end{equation*}
        where $\omega$ is an $m^\textrm{th}$ root of unity. Indeed, for all $f\l(c\r)\in U$, there exists some $a'\in V_0$ such that $f\l(a'\r)=f\l(c\r)$. Then $c=\omega^ja'$ for some $0\leq j\leq m-1$, so $c\in\omega^jV_0$. Conversely, if $c\in\omega^jV_0$ for some $0\leq j\leq m-1$, then $c=\omega^ja'$ for some $a'\in V_0$ and hence $f\l(c\r)=f\l(\omega^ja'\r)=f\l(a'\r)\in U$. Now, since $f^{-1}\!\l(b\r)$ is discrete, the sets $V_j\coloneqq\omega^jV_0$ can be made small enough so that they are pairwise disjoint. Then each $\l.f\r|_{V_j}:V_j\to U$ is a homeomorphism, as desired.\exqed
    \end{example}
    \begin{example}
        For any lattice $\Gamma\subseteq\C$, the projection $\pi:\C\to\C/\Gamma$ is a covering map. Indeed, take $z+\Gamma\in\C/\Gamma$ and let $w\in\C$ be such that $\pi\l(w\r)=z+\Gamma$. Since $\pi$ is a local homeomorphism\footnote{This follows directly from our construction of complex tori in Example \ref{RS:exa:tori}, where for every $w\in\C$, a small enough neighborhood $V$ was found so that $\l.\pi\r|_V$ is injective.}, there exist neighborhoods $V$ of $w$ and $U$ of $z+\Gamma$ such that $\l.\pi\r|_{V}:V\to U$ is a homeomorphism. We similarly claim that
        \begin{equation*}
            \pi^{-1}\!\l(U\r)=\coprod_{\lambda\in\Gamma}\l(\lambda+V\r).
        \end{equation*}
        Indeed, for all $\pi\l(z\r)\in U$, there exists some $w'\in V$ such that $\pi\l(z\r)=\pi\l(w'\r)$. Then $z+\Gamma=w'+\Gamma$, so $z=w'+\lambda$ for some $\lambda\in\Gamma$. Conversely, if $z\in\lambda+V$ for some $\lambda\in\Gamma$, then $z=w'+\lambda$ for some $w'\in V$ and hence $\pi\l(z\r)=\pi\l(w'+\lambda\r)=\pi\l(w\r)\in U$. Now, the sets $V_\lambda\coloneqq\lambda+V$ are all disjoint and each $\l.\pi\r|_{V_\lambda}:V_\lambda\to U$ is a homeomorphism, as desired.\exqed
    \end{example}
    \begin{proposition}\label{CS:prp:fiber_cardinalities_coincide}
        Let $\pi:E\to X$ be a covering map. If $X$ is connected, then the fibers $\pi^{-1}\!\l(p\r)$ at each $p\in X$ are equinumerous.
    \end{proposition}
    \begin{proof}
        Consider the equivalence relation $\sim$ on $X$ defined by $p\sim p'$ iff the fibers over $p$ and $p'$ are equinumerous. We claim that the equivalence classes are all open, and since they partition $X$, the connectedness of $X$ then shows that there is only one equivalence class, as desired. Indeed, take $p\in X$ and let $U\ni p$ be evenly-covered by $\l\{V_j\r\}$. For any $p'\in U$, the set $\pi^{-1}\!\l(p'\r)\cap V_j$ is a singleton for all $j\in J$, so $\l|\pi^{-1}\!\l(p'\r)\r|=\l|J\r|$. In particular, since $p\in U$, we have $p\sim p'$, as desired.\qed
    \end{proof}
    \begin{proposition}\label{CS:prp:proper_local_covering}
        Any proper local homeomorphism is a covering map.
    \end{proposition}
    \begin{proof}
        Let $\pi:E\to X$ be a proper local homeomorphism and take $p\in X$. We claim that $\pi^{-1}\!\l(p\r)$ is finite.
        \begin{itemize}
            \item For each $\zeta\in\pi^{-1}\!\l(p\r)$, there exist neighborhoods $W_\zeta$ of $\zeta$ and $U$ of $p$ such that $\l.\pi\r|_{W_\zeta}:W_\zeta\to U$ is a homeomorphism. Then the sets $W_\zeta$ must be disjoint, for if $\zeta'\in W_\zeta\cap W_{\zeta'}$ for some $\zeta'\neq\zeta$, then $\l.\pi\r|_{W_\zeta}\!\l(\zeta\r)=p=\l.\pi\r|_{W_\zeta}\!\l(\zeta'\r)$, contradicting that $\l.\pi\r|_{W_\zeta}$ is a homeomorphism. Thus $\pi^{-1}\!\l(p\r)$ must be finite, lest the cover $\l\{W_\zeta\r\}$ admits no finite subcover.
        \end{itemize}
        Thus $\pi^{-1}\!\l(p\r)=\l\{\zeta_1,\dots,\zeta_n\r\}$ for some $\zeta_j\in E$. Letting $W_j\coloneqq W_{\zeta_j}$ as above, we see that $\coprod_{j=1}^{n}W_j$ is a neighborhood of $\pi^{-1}\!\l(p\r)$. By Proposition \ref{CS:prp:proper_give_neighborhoods}, there is a neighborhood $U$ of $p$ such that $\pi^{-1}\!\l(U\r)\subseteq\coprod_{j=1}^{n}W_j$, so $\pi^{-1}\!\l(U\r)=\coprod_{j=1}^{n}V_j$ where the sets $V_j\coloneqq W_j\cap\pi^{-1}\!\l(U\r)$ are all disjoint and each $\l.\pi\r|_{V_j}:V_j\to U$ is a homeomorphism.\qed
    \end{proof}
    \subsection{Ramification Points and the Degree}
    Throughout this section, let $F:Y\to X$ be a (non-constant) proper holomorphic map between Riemann surfaces $X$ and $Y$. We extend Proposition \ref{CS:prp:fiber_cardinalities_coincide} to $F$, which is `almost' a covering map, and define the \textit{degree} of $F$.
    \begin{definition}
        A point $q\in Y$ is said to be a \uldef{ramification/branch point of $F$} if $\l.F\r|_V$ is not injective for any neighborhood $V$ of $q$, in which case $F\l(q\r)\in X$ is said to be a \uldef{critical point of $F$}. If $F$ has no ramification points, then $F$ is said to be an \uldef{unbranched holomorphic map}.
    \end{definition}
    \begin{remark}
        It is immediate that $F$ is unbranched iff it is a local homeomorphism. Indeed, if $F$ is unbranched, then for every $q\in Y$ there exists a neighborhood $V$ of $q$ such that $\l.F\r|_V$ is injective. By the Open Mapping Theorem, $F$ is open and hence $\l.F\r|_V$ maps $V$ homeomorphically to the open set $F\l(V\r)$. Conversely, if $F$ is a local homeomorphism, then for every $q\in Y$ there exists a neighborhood $V$ of $q$ that is mapped homeomorphically onto an open set in $X$. Thus $\l.F\r|_V$ is injective, so $F$ is unbranched at $q$.\\\ \\
        In particular, this shows that every covering map is unbranched. Conversely, Proposition \ref{CS:prp:proper_local_covering} shows that every unbranched proper map is a covering map, so all fibers are equinumerous. On the other hand, if $F$ is branched, then it is a covering map over $X$ with all ramification points removed. Including the ramification points, however, the fibers of $F$ are \textit{not necessarily} equinumerous anymore, but the next best thing happens and we only need to count the fibers \textit{with multiplicity}. First, we need a lemma.\exqed
    \end{remark}
    \begin{lemma}\label{CS:lem:ramification_iff_mult_2}
        For all $q\in Y$, the map $F:Y\to X$ has a ramification point at $q$ iff $\mult_q\!\l(F\r)\geq2$.
    \end{lemma}
    \begin{proof}
        By Theorem \ref{RS:thm:local_normal_form}, there exist charts $\tpl{V,\psi}$ centered at $q$ and $\tpl{U,\phi}$ centered at $F\l(q\r)$ such that $f\coloneqq\phi\circ F\circ\psi^{-1}$ is the power map $z\mapsto z^m$ where $m\coloneqq\mult_q\!\l(F\r)$. Since $\phi$ and $\psi$ are, in particular, injections, we see that $F$ is locally injective at $q$ iff $f$ is locally injective at $0$. But this occurs precisely when $m=\mult_q\!\l(F\r)<2$, so the result follows.\qed
    \end{proof}
    \begin{defthm}\label{CS:thm:degree_sum_of_multiplicities}
        The \uldef{degree of $F$} is the cardinality of any fiber $F^{-1}\!\l(p\r)$ for $p\in X$, counted with multiplicity. That is, $\deg F\coloneqq\sum_{q\in F^{-1}\!\l(p\r)}\mult_q\!\l(F\r)$ is independent of $p\in X$.
    \end{defthm}
    \begin{proof}
        For non-critical points $p\in X$, Lemma \ref{CS:lem:ramification_iff_mult_2} shows that $\mult_q\!\l(F\r)=1$ for any $q\in F^{-1}\!\l(p\r)$. Then $\deg F=\l|F^{-1}\!\l(p\r)\r|$, and since Proposition \ref{CS:prp:proper_local_covering} shows that $F$ is a covering map when all ramification points are removed, it is, by Proposition \ref{CS:prp:fiber_cardinalities_coincide}, constant over all non-critical points.\\\ \\
        Otherwise, let $p$ be a critical point of $F$. Since $F^{-1}\!\l(p\r)$ is compact, it is finite by Discreteness of Preimages, say $F^{-1}\!\l(p\r)=\l\{q_1,\dots,q_n\r\}$ for $q_i\in Y$. Fix $1\leq j\leq n$ and set $m_j\coloneqq\mult_{q_j}\!\l(F\r)$. We claim that there exist neighborhoods $V_j$ of $q_j$ and $U_j$ of $p$ such that $\l|F^{-1}\!\l(r\r)\cap V_j\r|=m_j$ for all $r\in U_j\comp\l\{p\r\}$. Indeed, by Theorem \ref{RS:thm:local_normal_form}, there exist charts $\tpl{V_j,\psi_j}$ of $Y$ centered at $q_j$ and $\tpl{U_j,\phi_j}$ of $X$ centered $p$ such that $F$ acts as the power function $f\l(z\r)\coloneqq z^{m_j}$ on $\psi_j\!\l(V_j\r)$. Since the set of ramification points of $F$ is discrete, we may choose $U_j$ small enough so that every $r\in U_j\comp\l\{p\r\}$ is unramified. Take $r\in U_j\comp\l\{p\r\}$ and set $w\coloneqq\phi_j\!\l(r'\r)\neq0$. Then $\l|f^{-1}\!\l(w\r)\r|=m_j$, so we have
        \begin{equation*}
            \l|F^{-1}\!\l(r\r)\cap V_j\r|=\l|\psi_j\l(F^{-1}\!\l(r\r)\r)\r|=\l|\psi_j\l(F^{-1}\!\l(\phi_j^{-1}\!\l(w\r)\r)\r)\r|=\l|f^{-1}\!\l(w\r)\r|=m_j.
        \end{equation*}
        Now, since $V_j$ is a neighborhood of $q_j$, we see that $F^{-1}\!\l(U_j\r)\subseteq V_j$ by restricting $U_j$ in accordance with Proposition \ref{CS:prp:proper_give_neighborhoods}, if necessary. Then, with $U\coloneqq\bigcap_{i=1}^{n}U_i$, we see that $F^{-1}\!\l(U\r)\subseteq\coprod_{i=1}^{n}V_i$ where the sets $V_i$ are all disjoint. Take any $r\in U\comp\l\{p\r\}$. Then $r\in U_i\comp\l\{p\r\}$ for all $1\leq i\leq n$, so
        \begin{equation*}
            \l|F^{-1}\!\l(r\r)\r|=\l|F^{-1}\!\l(r\r)\cap\bigcup_{i=1}^{n}V_i\r|=\l|\bigcup_{i=1}^{n}\l(F^{-1}\!\l(r\r)\cap V_i\r)\r|=\sum_{i=1}^{n}\l|F^{-1}\!\l(r\r)\cap V_i\r|=\sum_{i=1}^{n}m_i.
        \end{equation*}
        But $r$ is not a critical point of $F$, so $\deg F=\l|F^{-1}\!\l(r\r)\r|=\sum_{i=1}^{n}m_i$ and the result follows.\qed
    \end{proof}
    \begin{example}
        We extend Example \ref{CS:exa:power_map} by considering the same power map $z\mapsto z^m$ for $m\geq2$, this time as a map $f:\C\to\C$. Away from $0$, the map $f$ is a proper local homeomorphism as before, so the cardinality of any fiber is $m$. At $0$, we see that $\mult_0\!\l(f\r)=m\geq2$, so $f$ has a ramification point at $0$. Counting multiplicities, the cardinality of $f^{-1}\!\l(0\r)$ is $m$, so $\deg f=m$.\exqed
    \end{example}
    \begin{corollary}
        If $Y$ is compact, then a holomorphic map $F:Y\to X$ is a biholomorphism iff $\deg F=1$.
    \end{corollary}
    \begin{proof}
        Since $Y$ is compact, we see that $F$ is proper surjection. Observe that $F$ is an injective iff it has no critical points, and by Lemma \ref{CS:lem:ramification_iff_mult_2}, this occurs iff $\mult_q\!\l(F\r)=1$ for all $q\in Y$.
        \begin{itemize}
            \item ($\Rightarrow$) If $F$ is an injection, then $\l|f^{-1}\!\l(q\r)\r|=1$ for all $q\in Y$. Thus $\deg F=1$.
                \vspace{-0.05in}
            \item ($\Leftarrow$): Since $\mult_q\!\l(F\r)\geq1$ for all $q\in Y$, the above theorem forces $\mult_q\!\l(F\r)=1$.\qed
        \end{itemize}
    \end{proof}
    \begin{corollary}\label{CS:cor:exists_meromorphic_implies_biholomorphic_Riemann_sphere}
        If $X$ is compact and there exists a meromorphic function $f:X\to\C$ with a single simple pole, then $X\iso\RS$.
    \end{corollary}
    \begin{proof}
        Let $f:X\to\C$ be a meromorphic function with only a simple pole at $p$ and consider its associated holomorphic map $F:X\to\RS$. By Proposition \ref{RS:prp:multiplicity_and_order}, we see that $\mult_p\!\l(F\r)=\ord_p\!\l(f\r)=1$ and hence $p$ is unramified. Since $p$ is the only pole of $f$, we see that $\deg F=\l|F^{-1}\!\l(\infty\r)\r|=1$. Thus $F$ is a biholomorphism, as desired.\qed
    \end{proof}
    \begin{remark}
        This criterion finally reduces to problem of showing that the moduli space of $S^2$ is a point to showing that every Riemann surface $X$ that is topologically the sphere admits a meromorphic function $f:X\to\C$ with a single simple pole. We dedicate Chapter \ref{CC} to find such a meromorphic function.\exqed
    \end{remark}
\end{document}
