\documentclass[../Moduli_Spaces_of_Riemann_Surfaces.tex]{subfiles}
\begin{document}
    \section{Liftings of Mappings}\label{CS:sec:lifting_criterion}
    Using the Homotopy Lifting Property of covering maps, we establish a criterion for when a map $F:Y\to X$ between topological spaces admits a lift $\widetilde{F}:Y\to E$ along a covering map $\pi:E\to X$. Throughout this section, $X$, $Y$, and $E$ are all topological spaces and all maps are continuous.
    \begin{definition}
        Let $\pi:E\to X$ and $F:Y\to X$ be maps. A \uldef{lift of $F$ (along $\pi$)} is a map $\widetilde{F}:Y\to E$ such that $\pi\circ\widetilde{F}=F$; that is, such that the diagram below commutes.
        \begin{equation*}
            \begin{tikzcd}
                & E \ar[d, "\pi"] \\
                Y \ar[ur, "\tilde{F}"] \ar[r, "F"'] & X
            \end{tikzcd}
        \end{equation*}
    \end{definition}
    \subsection{Liftings of Curves and Homotopies}
    \begin{proposition}[Homotopy Lifting Property]
        If $\pi:E\to X$ is a covering map, then for any homotopy $F:Y\times\l[0,1\r]\to X$ and any fixed map $\widetilde{f}_0:Y\times\l\{0\r\}\to E$ lifting the restriction of $F$ on $Y\times\l\{0\r\}$, there exists a unique homotopy $\widetilde{F}:Y\times\l[0,1\r]\to E$ lifting $F$ that restricts to $\widetilde{f}_0$ on $Y\times\l\{0\r\}$. In other words, the following diagram commutes.
        \begin{equation*}
            \begin{tikzcd}[row sep=0.4in]
                Y\times\l\{0\r\} \ar[r, "\widetilde{f}_0"] \ar[d, "\iota"', hookrightarrow] & E \ar[d, "\pi"] \\
                Y\times\l[0,1\r] \ar[r, "F"] \ar[ur, "\widetilde{F}", dashed] & X
            \end{tikzcd}
        \end{equation*}
    \end{proposition}
    \begin{proof}
        Since $\pi$ is a covering map, there exists an open cover $\l\{U_\alpha\r\}$ of $X$, each evenly-covered by $\l\{V_{\alpha\beta}\r\}$. Fix $q_0\in Y$. For each $\tpl{q_0,t_i}\in Y\times\l[0,1\r]$, let $U_i\subseteq X$ be an open set containing $F\l(q_0,t_i\r)$. Continuity of $F$ then furnishes an open set $N_i\times\l(a_i,b_i\r)\ni\tpl{q_0,t_i}$ such that $F\l(N_i\times\l(a_i,b_i\r)\r)\subseteq U_i\subseteq X$. The collection $\l\{N_i\times\l(a_i,b_i\r)\r\}$ covers $\l\{q_0\r\}\times\l[0,1\r]$, so by compactness one obtains an open set $N\coloneqq\bigcap N_i$ containing $q_0$ and a partition $0=t_0<t_1<\cdots<t_n=1$ of $\l[0,1\r]$ such that each $F\l(N\times\l[t_i,t_{i+1}\r]\r)\subseteq U_i$ is evenly-covered. We define $\widetilde{F}:N\times\l[0,t_i\r]\to E$ by induction on $i$; for $i=0$, we let $\widetilde{F}\coloneqq\widetilde{f}_0$ so that $\widetilde{F}$ restricts to $\widetilde{f}_0$ on $N\times\l\{0\r\}$.\\\ \\
        Suppose a lift $\widetilde{F}:N\times\l[0,t_i\r]\to E$ has been constructed for some $i\geq0$. Then, since $F\l(q_0,t_i\r)\in U_i$, there exists a unique open set $V_i\subseteq\pi^{-1}\!\l(U_i\r)$ containing $\widetilde{F}\l(q_0,t_i\r)$ that maps homeomorphically onto $U_i$. Replacing $N\times\l\{t_i\r\}$ by its intersection with $\widetilde{F}^{-1}\!\l(V_i\r)$, if necessary, we may assume that $\widetilde{F}\l(N\times\l\{t_i\r\}\r)\subseteq V_i$. Since $\pi$ is invertible on $V_i$, extend $\widetilde{F}$ so that
        \begin{equation*}
            \begin{tikzcd}[row sep = 0.4in]
                & V_i \ar[d, "\pi"] \\
                N\times\l[t_i,t_{i+1}\r] \ar[r, "F"'] \ar[ur, "\widetilde{F}"] & U_i
            \end{tikzcd}
        \end{equation*}
        commutes. Our modification of $N\times\l\{t_i\r\}$ ensures that the restriction of $\widetilde{F}$ to $N\times\l\{t_i\r\}$ coincides with this extension, so the functions inductively glue to give a lift $\widetilde{F}$ of $F$ on $N\times\l[0,1\r]$. We now argue that such a lifting is unique when $Y$ is a point\footnote{Here, we are not necessary assuming that $Y=\l\{q_0\r\}$.}; abusing notation, we drop $Y$ from the notation and write $F:\l[0,1\r]\to X$, etc., instead.
        \begin{itemize}
            \item Suppose that $\widetilde{F}'\!:\l[0,1\r]\to E$ is another lift of $F$ such that $\widetilde{F}\l(0\r)=\widetilde{F}'\!\l(0\r)=\widetilde{f}_0\!\l(0\r)$. As above, we may obtain a partition $0=t_0<t_1<\cdots,t_n=1$ of $\l[0,1\r]$ such that each $F\l(\l[t_0,t_{i+1}\r]\r)\subseteq U_i$ is evenly-covered. Proceeding by induction, suppose that $\widetilde{F}=\widetilde{F}'$ on $\l[0,t_i\r]$. Since $\l[t_i,t_{i+1}\r]$ is connected, $\widetilde{F}\l(\l[t_i,t_{i+1}\r]\r)$ is connected too and thus lies in a single open set $V_i\subseteq\pi^{-1}\!\l(U_i\r)$ containing $\widetilde{F}\l(t_i\r)$ that maps homeomorphically onto $U_i$. Similarly for $\widetilde{F}'\!\l(\l[t_i,t_{i+1}\r]\r)$, but since $\widetilde{F}\l(t_i\r)=\widetilde{F}'\!\l(t_i\r)$, they lie in the same open set $V_i$. Since $\pi\circ\widetilde{F}=\pi\circ\widetilde{F}'$ on $\l[t_i,t_{i+1}\r]$ and $\pi$ is injective on $V_i$, we see that $\widetilde{F}=\widetilde{F}'$ on $\l[t_i,t_{i+1}\r]$, as desired.
        \end{itemize}
        Thus, when restricted to $\l\{q\r\}\times\l[0,1\r]$ for each $q\in N$, the lift $\widetilde{F}:N\times\l[0,1\r]\to E$ of $F$ is unique. In general, this shows that if the same construction is repeated for some other $q_0'\in Y$ to obtain a lift $\widetilde{F}'\!:N'\times\l[0,1\r]\to E$ of $F$, and if $N\cap N'\neq\em$, the lifts $\widetilde{F}$ and $\widetilde{F}'$ must agree on $\l(N\cap N'\r)\times\l[0,1\r]$. Thus $\widetilde{F}$ is well-defined on $Y\times\l[0,1\r]$, and is continuous since it is continuous on each $N\times\l[0,1\r]$.\qed
    \end{proof}
    \begin{corollary}
        Every covering map lifts curves and homotopies. More precisely:
        \begin{itemize}
            \item[$\blob$] For each curve $\gamma:\l[0,1\r]\to X$ starting at some point $p\in X$ and each $\zeta_0\in\pi^{-1}\!\l(p\r)$, there exists a unique curve $\widetilde{\gamma}:\l[0,1\r]\to E$ starting at $\zeta_0$ lifting $\gamma$.
                \vspace{-0.05in}
            \item[$\blob$] For each homotopy $\gamma_t:\l[0,1\r]\to X$ of curves and each lift $\widetilde{\gamma}_0:I\to E$ of $\gamma_0$, there exists a unique homotopy $\widetilde{\gamma}_t:I\to E$ of curves starting at $\widetilde{\gamma}_0$ lifting $\gamma_t$.
        \end{itemize}
    \end{corollary}
    \begin{proof}
        In the notation of the preceding proposition, let $Y$ be a singleton and $\l[0,1\r]$, respectively. Note that the resulting homotopy $\widetilde{\gamma}_t$ is a homotopy \textit{of curves} since as $t$ varies, the endpoints $\widetilde{\gamma}_t\!\l(0\r)$ and $\widetilde{\gamma}_t\!\l(1\r)$ are curves in $E$ that lift the constant curves at $\gamma_t\!\l(0\r)$ and $\gamma_t\!\l(1\r)$, respectively. By uniqueness of liftings of curves, we see that $\widetilde{\gamma}_t\!\l(0\r)$ and $\widetilde{\gamma}_t\!\l(1\r)$ are constant curves at the lifts of $\gamma_t\!\l(0\r)$ and $\gamma_t\!\l(1\r)$, respectively, as desired.\qed
    \end{proof}
    \subsection{The Lifting Criterion}
    \begin{lemma}
        
    \end{lemma}
    \begin{proof}
        
    \end{proof}
    \begin{proposition}[Lifting Criterion]
        
    \end{proposition}
    \begin{proof}
        
    \end{proof}
\end{document}
