\documentclass[../Moduli_Spaces_of_Riemann_Surfaces.tex]{subfiles}

\begin{document}
    \section{Organization and Prerequisites}
    We give a brief overview of the organization of this paper.
    \begin{itemize}
        \item Chapter \ref{RS} begins with some definitions and constructions relating to Riemann surfaces and introduces the main examples of interest to this paper: the Riemann sphere and complex tori. We then study the basic behaviours of maps between Riemann surfaces, with a focus on meromorphic functions and their associated holomorphic maps.
            \vspace{-0.05in}
        \item Chapter \ref{CS} studies the covering space theory of Riemann surfaces. The degree of a proper holomorphic map is defined, which is proven to be the cardinality of any fiber, counted with multiplicity. We finish with a proof of the existence of liftings, which will be used to compute the moduli space of $T^2$.
            \vspace{-0.05in}
        \item Chapter \ref{CC} builds up the basics of sheaf theory and their associated cohomology. The theory of (complex) differential forms and integration is then introduced to study the sheaf of holomorphic functions on the Riemann sphere, where we prove the existence of certain global meromorphic functions on $X$.
            \vspace{-0.05in}
        \item Chapter \ref{MS} ties everything together and uses the tools developed to compute the moduli space of genus $0$ and $1$ surfaces ($S^2$ and $T^2$). This chapter closes with a brief discussion of the Uniformization Theorem and its impacts on the theory of Riemann surfaces.
    \end{itemize}
    As for prerequisites, some familiarity with topology and complex analysis is required, and an exposure to the theory of real manifolds will be required in Chapter \ref{CC}. We also assume that the reader is comfortable with some linear algebra and basic group theory. A more complete list of prerequisites, along with references, will be given at the start of each chapter.
\end{document}
