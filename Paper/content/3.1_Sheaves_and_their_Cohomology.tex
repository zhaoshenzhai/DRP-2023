\documentclass[../Moduli_Spaces_of_Riemann_Surfaces.tex]{subfiles}
\begin{document}
    Using the language of \textit{sheaves} and \textit{cohomology}, we prove the existence of certain global meromorphic functions on a compact Riemann surface $X$. Section \ref{CC:sec:sheaves} introduces the language, \ref{CC:sec:differential_forms} studies the sheaf $\DIFF$ of differentiable functions on $X$, which includes differentiation and integration of \textit{forms}, and \ref{CC:sec:global_meromorphic_functions} studies its associated cohomology. The latter sections require some background in linear algebra and the theory of smooth manifolds, for which we refer the reader to \cite{leeSM}.
    \section{Sheaves and their Cohomology}\label{CC:sec:sheaves}
    Unless otherwise stated, let $X$ be a topological space with $\mc{T}$ its system of open sets. Our exposition on sheaves and their cohomology roughly follows \cite[][Sections 6 \& 12]{forster} and \cite[][Chapter IX]{miranda}.
    \subsection{Presheaves, Sheaves, and Stalks}
    \begin{definition}
        A \ul{presheaf of Abelian groups on $X$} is a pair $\tpl{\ms{F},\rho}$ consisting of
        \begin{itemize}
            \item[$\blob$] a family $\ms{F}\coloneqq\l\{\ms{F}\l(U\r)\r\}$ of Abelian groups $\ms{F}\l(U\r)$ for every $U\in\mc{T}$,
                \vspace{-0.05in}
            \item[$\blob$] a family $\rho\coloneqq\l\{\rho^U_V\r\}$ of group homomorphisms $\rho^U_V:\ms{F}\l(U\r)\to\ms{F}\l(V\r)$ for every $U,V\in\mc{T}$ with $V\subseteq U$,
        \end{itemize}
        such that $\rho^U_U=\id_{\ms{F}\l(U\r)}$ and $\rho^V_W\circ\rho^U_V=\rho^U_W$ for every $U,V,W\in\mc{T}$ with $W\subseteq V\subseteq U$.
    \end{definition}
    \begin{remark}
        We may analogously define a presheaf of sets, rings, vector spaces, $\C$-algebras, etc, by requiring that $\ms{F}\l(U\r)$ and $\rho$ are objects and maps of the appropriate `category'.\exqed
    \end{remark}
    \begin{remark}
        Presheaves give us a way of tracking data associated with open sets of a topological space in such a way that makes restricting to a smaller open set $V\subseteq U$ well-behaved. Consider, for instance, a Riemann surface $X$ and the presheaf of all holomorphic functions $\HOLO$ on $X$.
        \begin{itemize}
            \item To every open set $U\subseteq X$ we consider the $\C$-algebra $\HOLO\l(U\r)$ of all holomorphic functions $f:U\to\C$. For any open $V\subseteq U$, we define $\rho^U_V\!\l(f\r)\coloneqq\l.f\r|_V$. The properties then state that restricting to the domain does nothing and that restricting once to $V$ and then to $W\subseteq V$ yields the same function as restricting to $W$ directly, which are obviously true.
        \end{itemize}
        Similarly, we have the presheaf of all meromorphic functions $\MERO$ on $X$. Other examples include the presheaf $\HOLO^\ast$ and $\MERO^\ast$ of \textit{multiplicative} Abelian groups defined respectively by holomorphic functions $f:U\to\C^\ast$ and meromorphic functions on $U$ that do not vanish identically on any connected component of $U$. However, those examples are much more than presheaves since global information about elements in $\ms{F}\l(X\r)$ can be obtained locally by `restricting' to locally $U$. The notion of a \textit{sheaf} makes this precise.\exqed
    \end{remark}
    \begin{definition}
        A presheaf $\ms{F}$ on $X$ is said to be a \ul{sheaf} if for every open set $U\subseteq X$ and every family $\l\{U_i\r\}_{i\in I}$ of open subsets that cover $U$, the following two properties hold:
        \begin{itemize}
            \item[$\blob$] (Identity): For every $f,g\in\ms{F}\l(U\r)$, if $\rho^U_{U_i}\!\l(f\r)=\rho^U_{U_i}\!\l(g\r)$ for every $i\in I$, then $f=g$.
                \vspace{-0.1in}
            \item[$\blob$] (Gluing): For every family $\l\{f_i\r\}_{i\in I}$ with $f_i\in\ms{F}\l(U_i\r)$, if $\rho^{U_i}_{U_i\cap U_j}\!\l(f_i\r)=\rho^{U_j}_{U_i\cap U_j}\!\l(f_j\r)$ for all $i,j\in I$, then there is some $f\in\ms{F}\l(U\r)$ such that $\rho^U_{U_i}\!\l(f\r)=f_i$ for every $i\in I$.
        \end{itemize}
    \end{definition}
    \begin{example}
        It is immediate that $\HOLO$, $\HOLO^\ast$, $\MERO$, and $\MERO^\ast$ are all sheaves on $X$. Indeed, if we have a family $\l\{f_i\r\}$ that agree on all pairwise common domains, then there exists a globally defined function $f$ whose restrictions are $f_i$'s. We only need to show that this globally defined function is of the `right type', but this can be checked easily.\exqed
    \end{example}
    \begin{example}
        We give an example of a presheaf that is \textit{not} a sheaf. Let $X$ be a normed $\R$-vector space. For all $U\in\mc{T}$, let $\ms{B}\l(U\r)$ be the vector space of all bounded functions $f:U\to\R$, which gives us a presheaf $\ms{B}$ on $X$. The problem arises when we consider glueing\footnote{In other words, boundedness is a global property. To check if a function is bounded, it does \textit{not} suffice to check it on an arbitrary neighborhood.}. For instance, let $U_i\coloneqq\l\{p\in X\mid\l\|p\r\|<i\r\}$ and observe that $\l\{U_i\r\}_{i\in\R^+}$ covers $X$. Consider the family $\l\{\id_{U_i}\r\}$, which clearly agrees on pairwise intersections an glues up to the identity $\id_X$. But $\id_X$ is \textit{not} bounded, so $\ms{B}$ is not a sheaf.\exqed
    \end{example}
    \begin{example}
        We give two examples of sheaves relating to \textit{divisors} on a Riemann surface $X$; that is, functions $D:X\to\Z$ whose supports $\l\{p\in X\mid D\l(p\r)\neq0\r\}$ are discrete\footnote{Note that if $X$ is compact, then $D:X\to\Z$ is a divisor iff it has finite support, so the set of divisors of $X$ is the free Abelian group of $X$.} subsets of $X$.
        \begin{itemize}
            \item Let $D$ be a divisor on $X$. For every $U\in\mc{T}$, let $\HOLO\l[D\r]\l(U\r)$ denote the Abelian group of all meromorphic functions $f:X\to\C$ such that $\ord_p\!\l(f\r)\leq D\l(p\r)$ for all $p\in X$. The usual restriction homomorphisms make $\HOLO\l[D\r]$ is a sheaf of Abelian groups. This construction generalizes both $\HOLO$ and $\MERO$. Intuitively, the use of divisors here allow us to `bound' the orders of the poles of $f$ at specific points $p$, thereby restricting how badly-behaved it can be.
            \item For every $U\in\mc{T}$, let $\ms{D}\l(U\r)$ denote the group of all divisors on $U$. This makes $\ms{D}$ into a sheaf since for every family $\l\{D_i\r\}$, the function $D:X\to\Z$ that glues them together is also discretely-supported.\exqed
        \end{itemize}
    \end{example}
    \begin{definition}
        Let $\tpl{\ms{F},\rho}$ and $\tpl{\ms{G},\sigma}$ be two sheaves of Abelian groups on $X$. A \ul{morphism of sheaves} $\eta:\ms{F}\to\ms{G}$ is a family $\l\{\eta_U\r\}_{U\in\mc{T}}$ of group homomorphisms $\eta_U:\ms{F}\l(U\r)\to\ms{G}\l(U\r)$ such that for every $U\in\mc{T}$ and every open set $V\subseteq U$, the following diagram commutes.
        \begin{equation*}
            \begin{tikzcd}
                \ms{F}\l(U\r) \ar[r, "\eta_U"] \ar[d, "\rho^U_V"'] & \ms{G}\l(U\r) \ar[d, "\sigma^U_V"] \\
                \ms{F}\l(V\r) \ar[r, "\eta_V"] & \ms{G}\l(V\r)
            \end{tikzcd}
        \end{equation*}
    \end{definition}
    \begin{example}
        Some examples relating to divisors of a Riemann surface $X$.
        \begin{itemize}
            \item For divisors $D_1$ and $D_2$ of a Riemann surface $X$, we write $D_1\leq D_2$ if $D_1\!\l(p\r)\leq D_2\!\l(p\r)$ for all $p\in X$. This induces an \textit{inclusion morphism} $\iota:\HOLO\l[D_1\r]\into\HOLO\l[D_2\r]$ defined by $\iota_U\!\l(f\r)\coloneqq f$ for all $U\in\mc{T}$ and $f\in\HOLO\l[D_1\r]\l(U\r)$, which makes sense since if $D_1\leq D_2$ and the poles of $f$ are bounded by $D_1$, then they are also clearly bounded by $D_2$. This inclusion also respect restrictions, so it is indeed a morphism of sheaves. In particular, we have the inclusions $\HOLO\into\HOLO\l[D\r]\into\MERO$ for any divisor $D$ of $X$.
            \item For all $U\in\mc{T}$, we associate to each $f\in\MERO^\ast\!\l(U\r)$ the function $\div f:U\to\Z:p\mapsto\ord_p\!\l(f\r)$, which is a divisor by discreteness of zeros and poles. This induces a morphism of sheaves $\div:\MERO^\ast\to\ms{D}$ since for all $U\in\mc{T}$ and all open sets $V\subseteq U$, the restriction of the divisor of any $f\in\MERO^\ast\!\l(U\r)$ coincides with the divisor of the restriction $\l.f\r|_V$.\exqed
        \end{itemize}
    \end{example}
    \begin{definition}\label{CS:def:stalk}
        Let $\ms{F}$ be a presheaf of Abelian groups on $X$ and fix $p\in X$. The \ul{stalk of $\ms{F}$ at $p$} is the Abelian group
        \begin{equation*}
            \ms{F}_p\coloneqq\raisebox{-2pt}{$\Biggl(\!$}\coprod_{U\ni p}\ms{F}\l(U\r)\!\raisebox{-2pt}{$\Biggr)$}\!\!_{\scalebox{2}{/}\sim_p}
        \end{equation*}
        where $\sim_p$ is the equivalence relation\footnotemark on the disjoint union, defined, for all $f\in\ms{F}\l(U\r)$ and $g\in\ms{F}\l(V\r)$, by $f\sim_p g$ iff there exists an open set $W\in\mc{T}$ with $p\in W\subseteq U\cap V$ such that $\rho^U_W\!\l(f\r)=\rho^V_W\!\l(g\r)$. For $f\in\ms{F}\l(U\r)$, its equivalence class $\l[f\r]_p$ is called the \ul{germ of $f$ at $p$}.
    \end{definition}
    \footnotetext{The relation $\sim_p$ is transitive since $\rho^V_W\circ\rho^U_V=\rho^U_W$ for all $U,V,W\in\mc{T}$ such that $W\subseteq V\subseteq U$. This condition on $\rho$ is known as a \textit{directed system}, which all admit a \textit{direct limit} whose construction is formally similar to that of $\ms{F}_p$. We refer the interested reader to \cite[][Chapter III]{maclane}.}
    \vspace{-0.05in}
    \begin{example}
        Let $D$ be a divisor on a Riemann surface $X$ and consider the stalk $\HOLO_p\!\l[D\r]$. Fix a chart centered at $p$. Since every meromorphic function $f$ admits a Laurent series, we see that the function germ $\l[f\r]_p$ is represented by a Laurent series $\sum_{i=i_0}^{\infty}c_iz^i$ for some $i_0\geq-D\l(p\r)$. Conversely, the germ of a Laurent series $\sum_{i=i_0}^{\infty}c_iz^i$ with $i_0\geq-D\l(p\r)$ and whose principal part has positive radius of convergence represents a meromorphic function germ $\l[f\r]_p$, so this defines a bijection between $\HOLO_p\!\l[D\r]$ and the set of all such Laurent series. This isomorphism depends on the chosen chart map, so it is not canonical.\exqed
    \end{example}
    \begin{remark}
        The sheaf axioms guarantee that if $\ms{F}$ is a sheaf of Abelian groups on $X$ and $U\in\mc{T}$, then an element $f\in\ms{F}\l(U\r)$ is zero iff all germs $\l[f\r]_p$, for $p\in U$ vanish. Indeed, let $0\in\ms{F}\l(U\r)$ denote the zero element, so $f\sim_p0$ for all $p\in U$ furnishes a family $\l\{W_p\r\}$ of open sets $W_p\subseteq U$ containing $p$ such that $\rho^U_{W_p}\!\!\l(f\r)=\rho^U_{W_p}\!\!\l(0\r)$. This family covers $U$, so $f=0$ by the first sheaf axiom.\exqed
    \end{remark}
    \subsection{$\check{\textrm{C}}$ech Cohomology Groups}
    We define the \textit{first $\mathit{\check{C}}$\!ech cohomology group} $\check{H}^1\!\l(X,\ms{F}\r)$ of a sheaf $\ms{F}$ of Abelian groups on $X$ by first defining it on an open cover $\mf{A}$ of $X$, which refines via a direct limit, and then prove the \textit{Leray Criterion} to calculate such groups. Throughout this section, let $\ms{F}$ be a sheaf of Abelian groups on $X$ and let $\mf{A}\coloneqq\l\{U_i\r\}$ be an open covering of $X$.
    \begin{definition}
        For all $n\in\N$, the \ul{$n^\textrm{th}$ cochain group of $\ms{F}$ (w.r.t. $\mf{A}$)} is the direct product
        \begin{equation*}
            \check{C}^n\!\l(\mf{A},\ms{F}\r)\coloneqq\prod_{\tpl{i_0,\dots,i_n}}\ms{F}\l(U_{i_0}\cap\cdots\cap U_{i_n}\r).
        \end{equation*}
    \end{definition}
    \begin{remark}
        For $n=0$, the group $\check{C}^0\!\l(\mf{A},\ms{F}\r)$ contains tuples $\tpl{f_i}$ where each $f_i$ is defined on $U_i$. For $n=1$, the group $\check{C}^1\!\l(\mf{A},\ms{F}\r)$ contains tuples $\tpl{f_{ij}}$ where each $f_{ij}$ is defined on $U_i\cap U_j$. Intuitively, $\tpl{f_i}$ induces an element $\tpl{g_{ij}}\in\check{C}^1\!\l(\mf{A},\ms{F}\r)$ by setting $g_{ij}\coloneqq f_j-f_i$, which `chains' $\tpl{f_i}$ on $U_i\cap U_j$ by measuring their difference. The following definition formalizes this intuition.\exqed
    \end{remark}
    \begin{definition}
        For all $n\in\N$, the \ul{$n^\textrm{th}$ coboundary operator} is the map $\delta^n\!:\check{C}^n\!\l(\mf{A},\ms{F}\r)\to\check{C}^{n+1}\!\l(\mf{A},\ms{F}\r)$ mapping $\l(f_{i_0,\dots,i_n}\r)$ to the cochain $\l(g_{i_0,\dots,i_{n+1}}\r)$ defined by\footnotemark
        \begin{equation*}
            g_{i_0,\dots,i_{n+1}}\coloneqq\sum_{k=0}^{n+1}\l(-1\r)^k\rho\l(f_{i_0,\dots,\widehat{i_k},\dots,i_{n+1}}\r).
        \end{equation*}
        Define the \ul{$n^\textrm{th}$ cocycle group} $\check{Z}^n\!\l(\mf{A},\ms{F}\r)\coloneqq\ker\delta^n$ and the \ul{$n^\textrm{th}$ splitting cocycle group} $\check{B}^n\!\l(\mf{A},\ms{F}\r)\coloneqq\im\delta^{n-1}$, whose quotient
        \begin{equation*}
            \check{H}^n\!\l(\mf{A},\ms{F}\r)\coloneqq\check{Z}^n\!\l(\mf{A},\ms{F}\r)/\check{B}^n\!\l(\mf{A},\ms{F}\r)
        \end{equation*}
        is called the \ul{$n^\textrm{th}$ cohomology group of $\ms{F}$ (w.r.t. $\mf{A}$)}.
    \end{definition}
    \footnotetext{The `hat' notation represents a deletion. Also, $\rho$ is the appropriate restriction mapping of $\ms{F}$.}
    \vspace{-0.05in}
    \begin{remark}
        A calculation shows that $\check{B}^n\!\l(\mf{A},\ms{F}\r)\subseteq\check{Z}^n\!\l(\mf{A},\ms{F}\r)$, so the quotient makes sense. In particular, $\delta^{n+1}\circ\delta^n=0$.\exqed
    \end{remark}
    \begin{remark}
        For $n=0$, we have $\delta^0\!\l(f_i\r)=\tpl{f_j-f_i}$ for all $\tpl{f_i}\in\check{C}^0\!\l(\mf{A},\ms{F}\r)$. This gives us a glueing condition, that if $\tpl{f_i}\in\check{Z}^0\!\l(\mf{A},\ms{F}\r)$, then\footnote{Henceforth, we suppress the restriction maps $\rho$ for ease of notation, but will always mention on which domain the relation is valid on.} $\rho\l(f_i\r)=\rho\l(f_j\r)$ for all $i,j$ and hence the sheaf axioms furnish a unique $f\in\ms{F}\l(X\r)$ such that $\rho\l(f\r)=f_i$ for all $i$. Thus
        \begin{equation*}
            \check{H}^0\!\l(\mf{A},\ms{F}\r)=\check{Z}^0\!\l(\mf{A},\ms{F}\r)\iso\ms{F}\l(X\r),
        \end{equation*}
        so $\check{H}^0\!\l(\mf{A},\ms{F}\r)$ is independent of the covering $\mf{A}$ and we may define the \ul{$0^\textrm{th}$ cohomology group of $\ms{F}$} as $\check{H}^0\!\l(X,\ms{F}\r)\coloneqq\ms{F}\l(X\r)$.\exqed
    \end{remark}
    \begin{remark}
        For $n=1$, we have $\delta^1\!\l(f_{ij}\r)=\tpl{f_{jk}-f_{ik}+f_{ij}}$ for all $\tpl{f_{ij}}\in\check{C}^1\!\l(\mf{A},\ms{F}\r)$. Elements $\tpl{f_{ij}}\in\check{Z}^1\!\l(\mf{A},\ms{F}\r)$ satisfy the \textit{cocycle condition}, which states $f_{ik}=f_{ij}+f_{jk}$ on $U_i\cap U_j\cap U_k$ for all $i,j,k$. In particular, it implies that $f_{ii}=0$ for all $i$ and $f_{ij}=-f_{ji}$ on $U_i\cap U_j$ for all $i,j$. Note that every splitting cocycle is a cocycle, but not every cocycle splits. In other words, $\check{H}^1\!\l(\mf{A},\ms{F}\r)$ measures how $1$-cocycles fail to split. We now define the $1^\textrm{st}$ cohomology group of $\ms{F}$ by `refining' the open cover $\mf{A}$.\exqed
    \end{remark}
    \begin{definition}
        Let $\mf{A}\coloneqq\l\{U_i\r\}_{i\in I}$ and $\mf{B}\coloneqq\l\{V_k\r\}_{k\in K}$ be open coverings of $X$. We say that \ul{$\mf{B}$ is finer than $\mf{A}$}, and write $\mf{B}\preceq\mf{A}$, if there exists a \ul{refining map} $r:K\to I$ such that $V_k\subseteq U_{r\l(k\r)}$ for all $k\in K$.
    \end{definition}
    \begin{remark}
        The refining map $r$ induces a map $\tilde{r}:\check{Z}^1\!\l(\mf{A},\ms{F}\r)\to\check{Z}^1\!\l(\mf{B},\ms{F}\r)$ by sending $\tpl{f_{ij}}$ into $\tpl{g_{kl}}$ defined by $g_{kl}\coloneqq f_{r\l(k\r),r\l(l\r)}$ on $V_k\cap V_l$ for all $k,l\in K$. Observe that if $\delta^1_\mf{A}\!\l(f_{ij}\r)=0$, then $f_{i_1i_3}=f_{i_1i_2}+f_{i_2i_3}$ on $U_{i_1}\cap U_{i_2}\cap U_{i_3}$ for all $i_1,i_2,i_3\in I$. In particular, we have $f_{r\l(k_1\r),r\l(k_3\r)}=f_{r\l(k_1\r),r\l(k_2\r)}+f_{r\l(k_2\r),r\l(k_3\r)}$ on $V_{k_1}\cap V_{k_2}\cap V_{k_3}$ for all $k_1,k_2,k_3\in K$, so $\delta^1_\mf{B}\!\l(\tilde{r}\l(f_{ij}\r)\r)=0$. Thus $\tilde{r}$ sends splitting cocycles into splitting cocycles, so we may descend $\tilde{r}$ into the quotient, giving us a map
        \begin{equation*}
            \check{H}\l(r\r):\check{H}^1\!\l(\mf{A},\ms{F}\r)\to\check{H}^1\!\l(\mf{B},\ms{F}\r)\ \ \ \ \textrm{mapping}\ \ \ \ \l[f_{ij}\r]\mapsto\l[\tilde{r}\l(f_{ij}\r)\r].\exqedin
        \end{equation*}
    \end{remark}
    \begin{proposition}\label{CC:prp:lifted_refinement_independent_of_refining_map}
        In the above notation, the map $\check{H}^\mf{A}_\mf{B}\coloneqq\check{H}\l(r\r)$ is independent of $r$ and is injective.
    \end{proposition}
    \begin{proof}
        Take $\tpl{f_{ij}}\in\check{Z}^1\!\l(\mf{A},\ms{F}\r)$ and suppose that $r':K\to I$ is another refining map. Inducing the map $\tilde{r}'$ similarly, let $\tpl{g_{kl}}\coloneqq\tilde{r}\l(f_{ij}\r)=\tpl{f_{r\l(k\r),r\l(l\r)}}$ and $(g'_{kl})\coloneqq\tilde{r}'\l(f_{ij}\r)=\tpl{f_{r'\l(k\r),r'\l(l\r)}}$. Observe then that
        \begin{equation*}
            \begin{aligned}
                g_{kl}-g_{kl}'&=f_{r\l(k\r),r\l(l\r)}-f_{r'\l(k\r),r'\l(l\r)} \\
                              &=f_{r\l(k\r),r\l(l\r)}+f_{r\l(l\r),r'\l(k\r)}-f_{r\l(l\r),r'\l(k\r)}-f_{r'\l(k\r),r'\l(l\r)} \\
                              &=f_{r\l(k\r),r'\l(k\r)}-f_{r\l(l\r),r'\l(l\r)}
            \end{aligned}
        \end{equation*}
        on $V_k\cap V_l$ for all $k,l\in K$. Since $r$ and $r'$ are refining maps, we see that $V_k\subseteq U_{r\l(k\r)}\cap U_{r'\l(k\r)}$ for all $k\in K$, so we may define $h_k\coloneqq f_{r\l(k\r),r'\l(k\r)}$ on the restriction to $V_k$. Then $\tpl{g_{kl}-g_{kl}'}=\tpl{h_k-h_l}=\delta^0\!\l(h_k\r)$ on $V_k\cap V_l$, so $\tpl{g_{ij}}-(g_{ij}')\in\check{B}^1\!\l(\mf{B},\ms{F}\r)$. Thus their equivalence classes coincide, as desired. Now, to show that $\check{H}_\mf{B}^\mf{A}$ is injective, take $\tpl{f_{ij}}\in\ker\check{H}_\mf{B}^\mf{A}$. Thus $\tpl{f_{r\l(k\r),r\l(l\r)}}=\check{H}_\mf{B}^\mf{A}\!\l(f_{ij}\r)$ splits, so there exist $g_k\in\ms{F}\l(V_k\r)$ such that $f_{r\l(k\r),r\l(l\r)}=g_k-g_l$ on $V_k\cap V_l$ for all $k,l\in K$. Then
        \begin{equation*}
            g_k-g_l=f_{r\l(k\r),i}+f_{i,r\l(l\r)}=f_{i,r\l(l\r)}-f_{i,r\l(k\r)}
        \end{equation*}
        on $U_i\cap V_k\cap V_l$ for all $i\in I$ and hence $g_k+f_{i,r\l(k\r)}=g_l+f_{i,r\l(l\r)}$ on the same domain. Fixing $i\in I$ and glueing the family $\l\{g_k+f_{i,r\l(k\r)}\r\}_{k\in K}$ defined on the cover $\l\{U_i\cap V_k\r\}_{k\in K}$ of $U_i$, we obtain an element $h_i\in\ms{F}\l(U_i\r)$ such that $h_i=g_k+f_{i,r\l(k\r)}$ on $U_i\cap V_k$ for all $k\in K$. Observe then that
        \begin{equation*}
            f_{ij}=f_{i,r\l(k\r)}-f_{j,r\l(k\r)}=h_i-g_k-h_j+g_k=h_i-h_j
        \end{equation*}
        on $U_i\cap U_j\cap V_k$. Note that both $f_{ij}$ and $h_i-h_j$ are defined on $U_i\cap U_j$, and since they coincide on the restriction to $V_k$, uniqueness of the glueing gives us $f_{ij}=h_i-h_j$ on $U_i\cap U_j$. Thus $\tpl{f_{ij}}=\delta^0\!\l(h_i\r)$, so $\tpl{f_{ij}}$ splits.\qed
    \end{proof}
    \begin{remark}
        If $\mf{C}\preceq\mf{B}\preceq\mf{A}$ are open coverings of $X$, we have that $\check{H}^\mf{B}_\mf{C}\circ\check{H}^\mf{A}_\mf{B}=\check{H}^\mf{A}_\mf{C}$. This allows us to give a construction of $\check{H}^1\!\l(X,\ms{F}\r)$ that is formally similar to that of stalks (see Definition \ref{CS:def:stalk} and its associated footnote).\exqed
    \end{remark}
    \begin{definition}
        The \ul{$1^\textrm{st}$ cohomology group of $\ms{F}$} is the Abelian group
        \begin{equation*}
            \check{H}^1\!\l(X,\ms{F}\r)\coloneqq\raisebox{-2pt}{$\Biggl(\!$}\coprod_{\mf{A}}\check{H}^1\!\l(\mf{A},\ms{F}\r)\!\raisebox{-2pt}{$\Biggr)$}\!\!_{\scalebox{2}{/}\sim}
        \end{equation*}
        where $\sim$ is the equivalence relation on the disjoint union, defined, for all $\xi\in\check{H}^1\!\l(\mf{A},\ms{F}\r)$ and $\xi'\in\check{H}^1\!\l(\mf{A}',\ms{F}\r)$, by $\xi\sim\xi'$ iff there exists a refinement $\mf{B}\preceq\mf{A},\mf{A}'$ such that $\check{H}^\mf{A}_\mf{B}\!\l(\xi\r)=\check{H}^{\mf{A}'}_\mf{B}\!\!\l(\xi'\r)$.
    \end{definition}
    \begin{remark}
        Note that $\check{H}^1\!\l(X,\ms{F}\r)$ vanishes iff $\check{H}^1\!\l(\mf{A},\ms{F}\r)=0$ for all open coverings $\mf{A}$ of $X$. The converse direction is trivial, and for the forward, let $\mf{A}$ be an open covering of $X$. By Proposition \ref{CC:prp:lifted_refinement_independent_of_refining_map}, the canonical maps $\check{H}^1\!\l(\mf{A},\ms{F}\r)\to\check{H}^1\!\l(\mf{B},\ms{F}\r)$ are injective for all open coverings $\mf{B}\preceq\mf{A}$. Descending into the quotient, the induced map $\check{H}^1\!\l(\mf{A},\ms{F}\r)\to\check{H}^1\!\l(X,\ms{F}\r)$ is also injective, as desired.\exqed
    \end{remark}
    % \begin{proposition}\label{prp:3.2:vanishing_of_sheaf_differentiable}
    %     Let $X$ be a Riemann surface and consider the sheaf of differentiable functions $\DIFF$ on $X$. Then $\check{H}^1\!\l(X,\DIFF\r)=0$.
    % \end{proposition}
    % \begin{proof}
    %     Let $\mf{A}\coloneqq\l\{U_i\r\}_{i\in I}$ be an open covering of $X$ and let $\tpl{f_{ij}}\in\check{Z}^1\!\l(\mf{A},\DIFF\r)$ be a cocycle; it suffices to show that $\tpl{f_{ij}}$ splits, for then $\check{H}^1\!\l(\mf{A},\DIFF\r)=0$ and we are done by the remark above. To do so, we use the fact that there exists\footnote{For a proof, see \cite[][Theorem 2.23]{leeSM}. Note that the functions $\psi_i$ are \textit{not necessarily} holomorphic.} a partition of unity subordinate to $\mf{A}$; that is, a family $\l\{\psi_i\r\}_{i\in I}$ of differentiable functions such that:
    %     \begin{itemize}
    %         \item $\Supp\l(\psi_i\r)\coloneqq\bar{\l\{p\in X\mid\psi\l(p\r)\neq0\r\}}\subseteq U_i$ for every $i\in I$.
    %             \vspace{-0.05in}
    %         \item Every point in $X$ admits a neighborhood whose intersection with $\l\{\Supp\l(\psi_i\r)\r\}_{i\in I}$ is finite.
    %             \vspace{-0.05in}
    %         \item $\sum_{i\in I}\psi_i=\id$.
    %     \end{itemize}
    %     Consider the function $\psi_jf_{ij}$ on $U_i\cap U_j$, which may be differentiably extended to $U_i$ by zero outside $\Supp\l(\psi_j\r)$. Consider the function $g_i\coloneqq\sum_{j\in I}\psi_jf_{ij}\in\DIFF\l(U_i\r)$, which is legal since there is a neighborhood around every point of $U_i$ such that $\psi_jf_{ij}=0$ for all but finitely-many $j\in I$. Observe that
    %     \begin{equation*}
    %         g_i-g_j=\sum_{k\in I}\psi_k\l(f_{ik}-f_{jk}\r)=\sum_{k\in I}\psi_k\l(f_{ik}+f_{kj}\r)=\sum_{k\in I}\psi_kf_{ij}=f_{ij}
    %     \end{equation*}
    %     on $U_i\cap U_j$, so $\tpl{f_{ij}}=\tpl{g_i-g_j}=\delta^0\!\l(g_i\r)$ splits.\qed
    % \end{proof}
    \begin{proposition}[Leray]\label{CC:prp:leray}
        If $\mf{A}\coloneqq\l\{U_i\r\}_{i\in I}$ is an open covering of $X$ such that $\check{H}^1\!\l(U_i,\ms{F}\r)$ vanishes for every $i\in I$, then
        \begin{equation*}
            \check{H}^1\!\l(X,\ms{F}\r)\iso\check{H}^1\!\l(\mf{A},\ms{F}\r).
        \end{equation*}
        Such a covering $\mf{A}$ of $X$ is called a \ul{Leray covering of $X$}.
    \end{proposition}
    \begin{proof}
        Let $\mf{B}\coloneqq\l\{V_k\r\}_{k\in K}$ be an open covering of $X$ with $\mf{B}\preceq\mf{A}$, so there exists a refining map $r:K\to I$. We claim that $\check{H}^\mf{A}_\mf{B}$ is an isomorphism, from which the result follows by descending into the quotient. By Proposition \ref{CC:prp:lifted_refinement_independent_of_refining_map}, this map is injective, and to show that it is surjective, we must show that every cocycle $\tpl{f_{kl}}\in\check{Z}^1\!\l(\mf{B},\ms{F}\r)$ admits a cocycle $\tpl{F_{ij}}\in\check{Z}^1\!\l(\mf{A},\ms{F}\r)$ such that
        \begin{equation*}
            \tpl{F_{r\l(k\r),r\l(l\r)}}-\tpl{f_{kl}}\in\check{B}^1\!\l(\mf{B},\ms{F}\r).
        \end{equation*}
        For each $i\in I$, consider the open cover $U_i\cap\mf{B}\coloneqq\l\{U_i\cap V_k\r\}_{k\in K}$ of $U_i$. Since $\check{H}^1\!\l(U_i,\ms{F}\r)=0$, we see that $\check{H}^1\!\l(U_i\cap\mf{B},\ms{F}\r)=0$. Restricting to $U_i$, we see that $\tpl{f_{kl}}\in\check{Z}^1\!\l(U_i\cap\mf{B},\ms{F}\r)$ and hence there exist $g_{ik}\in\ms{F}\l(U_i\cap V_k\r)$ such that $f_{kl}=g_{ik}-g_{il}$ on $U_i\cap V_k\cap V_l$ for all $i\in I$ and $k,l\in K$. Using this result on two fixed $i,j\in I$ and equating, we see that $g_{jk}-g_{ik}=g_{jl}-g_{il}$ on $U_i\cap U_j\cap V_k\cap V_l$. This glues to an element $F_{ij}\in\ms{F}\l(U_i\cap U_j\r)$ such that $F_{ij}=g_{jk}-g_{ik}$ on $U_i\cap U_j\cap V_k$ for all $k\in K$, and a computation shows that $\tpl{F_{ij}}\in\check{Z}^1\!\l(\mf{A},\ms{F}\r)$. Observe then that\footnote{Here, we instantiated the relation $f_{kl}=g_{ik}-g_{il}$ with $i\coloneqq r\l(l\r)$.}
        \begin{equation*}
            F_{r\l(k\r),r\l(l\r)}-f_{kl}=\l(g_{r\l(l\r),k}-g_{r\l(k\r),k}\r)-\l(g_{r\l(l\r),k}-g_{r\l(l\r),l}\r)=g_{r\l(l\r),l}-g_{r\l(k\r),k}
        \end{equation*}
        on $V_k\cap V_l$, so setting $h_k\coloneqq g_{r\l(k\r),k}\in\ms{F}\l(V_k\r)$ shows that $\tpl{F_{r\l(k\r),r\l(l\r)}}-\tpl{f_{kl}}$ splits in $\mf{B}$.\qed
    \end{proof}
\end{document}
