\documentclass[../Moduli_Spaces_of_Riemann_Surfaces.tex]{subfiles}
\begin{document}
    We begin with some basic definitions and constructions relating to Riemann surfaces that will be used throughout this paper. This chapter requires some background in topology and complex analysis, all of which can be found in classical texts such as \cite{munkres} and \cite{lang}. For an introduction to topology focused on (real) manifolds, see \cite{leeTM} or \cite{tu}.
    \section{Charts and Atlases}\label{sec:charts_and_atlases}
    We first formalize what we mean for a topological space to `locally look like a patch of $\C$'. In this section, let $X$ be a connected second-countable Hausdorff space.
    \begin{definition}
        A \uldef{complex chart of $X$} is a pair $\tpl{U,\phi}$ where $\phi:U\to V$ is a homeomorphism from an open subset $U\subseteq X$ onto an open subset $V\subseteq\C$. Two charts $\l(U_1,\phi_1\r)$ and $\l(U_2,\phi_2\r)$ are said to be \uldef{compatible} if either $U_1\cap U_2=\em$, or the map
        \begin{equation*}
            \phi_2\circ\phi_1^{-1}:\phi_1\l(U_1\cap U_2\r)\to\phi_2\l(U_1\cap U_2\r),
        \end{equation*}
        called the \uldef{transition map}, is biholomorphic. A \uldef{complex atlas on $X$} is a collection $\mf{A}\coloneqq\l\{\l(U_i,\phi_i\r)\r\}_{i\in I}$ of pairwise compatible complex charts that cover $X$.
    \end{definition}
    \begin{remark}
        Charts provide \textit{local coordinates} for every point in $X$ in such a way that the transition maps $\phi_j\circ\phi_i^{-1}$ respect the analytic structure of $\C$. Within the same atlas $\mf{A}$, those charts give us different coordinate representations for points in $U_i\cap U_j$, and since no chart is distinguished from the others, we can only define notions using local coordinates if they are invariant under the transition map.
        \begin{equation*}
            \begin{tikzcd}[column sep=0.1in]
                & U_i\cap U_j \ar[dl, "\phi_i"'] \ar[dr, "\phi_j"] \\
                \phi_i(U_i\cap U_j) \ar[rr, "\phi_j\circ\phi_i^{-1}"'] & & \phi_j(U_i\cap U_j)
            \end{tikzcd}
        \end{equation*}
        It is a classical result in complex analysis that the inverse of a holomorphic map is also holomorphic, so $\phi_j\circ\phi_i^{-1}$ is biholomorphic iff $\phi_i\circ\phi_j^{-1}$ is, which is convenient when checking that a collection of charts form an atlas. Lastly, we remark that it is sometimes convenient to write $\tpl{U,z}$ for $\tpl{U,\phi}$, which can be decomposed into $z=x+iy$ by taking the real and imaginary parts of $\phi$.\exqed
    \end{remark}
    \begin{definition}
        Two complex atlases $\mf{A}$ and $\mf{B}$ on $X$ are said to be \uldef{equivalent} if every chart of $\mf{A}$ is compatible with every chart in $\mf{B}$.
    \end{definition}
    \begin{remark}
        By Zorn's Lemma, every atlas $\mf{A}$ of a manifold $X$ is contained in a unique maximal atlas on $X$ (see, for instance, \cite[][Proposition 1.17]{leeSM}). Moreover, two atlases are equivalent iff they are contained in the same maximal atlas, which justifies the following definition.\exqed
    \end{remark}
    \begin{definition}
        A \uldef{complex structure} on $X$ is a  maximal atlas $\mf{A}$ on $X$, or, equivalently, an equivalence class of complex atlases on $X$. The pair $\tpl{X,\mf{A}}$ is then called a \uldef{Riemann surface}.
    \end{definition}
    \begin{remark}
        Every Riemann surface can be regarded as a (connected) $2$-dimensional real manifold by $\textrm{`}$forgetting$\textrm{'}$ its complex structure. Since orientations are invariant under biholomorphisms, and in particular transition maps, the local orientation of $\C$ pulls-back via charts to a local orientation at each point $p\in X$. Since charts cover $X$, these local orientations induce a global orientation on $X$. Thus all Riemann surfaces are orientable, so, by the Classification of Surfaces, the closed Riemann surfaces are classified by their genus. Note, however, that this is a \textit{topological} classification, and does not give any information about the complex structure on $X$.\exqed
    \end{remark}
    \begin{example}
        Some elementary examples of Riemann surfaces.
        \begin{itemize}
            \item The complex plane $\C$, equipped with its standard topology, can be given a complex structure $\mf{A}$ by choosing the atlas containing a single chart $\tpl{\C,\id_\C}$. We may, however, also give $\C$ a different complex structure $\mf{A}'$ by choosing the chart map $\phi:z\mapsto\bar{z}$ instead. Indeed, $\mf{A}\neq\mf{A}'$ since the map $\phi\circ\id_\C^{-1}=\phi$ is not holomorphic and hence the atlases $\l\{\tpl{\C,\id_\C}\r\}$ and $\l\{\tpl{\C,\phi}\r\}$ are not equivalent. This example generalizes to any domain $D\subseteq\C$.
            \item Let $D\subseteq\C$ be a domain and consider any holomorphic function $f:D\to\C$. Then the graph $\Gamma\!_f\coloneqq\l\{\tpl{z,f\l(z\r)}\mid z\in D\r\}$, equipped with the subspace topology inherited from $\C^2$, can be given a complex structure by choosing the chart map $\pi:\Gamma\!_f\to D:\tpl{z,f\l(z\r)}\mapsto z$. More generally, the set $X$ of roots of an irreducible\footnote{The irreducibility of the polynomial ensures that its set of roots is connected. Its proof requires some algebraic geometry, which we take for granted.} polynomial $f\in\C\l[z,w\r]$ where every root has at least one non-vanishing partial derivative, called a \textit{smooth affine plane curve}, is a Riemann surface. Indeed, if $\partial f/\partial w$ is non-zero at $p=\tpl{z_0,w_0}$, then the Implicit Function Theorem furnishes a holomorphic function $g\l(z\r)$ defined on a neighborhood of $z_0$ such that $X=\Gamma_{\!g}$ on some neighborhood $U\ni p$. Then, as above, the projection $\pi_z:U\to\C$ is a homeomorphism onto its image, giving us the desired chart map.\exqed
        \end{itemize}
    \end{example}
    \subsection{The Riemann Sphere $\RS$}
    A particularly important Riemann surface is the Riemann sphere $\RS$, which admits several constructions. Here, we equip standard constructions of topological spheres with three complex structures, which \textit{a priori} need not be biholomorphic (in the sense of Definition $\ref{RS:def:biholomorphic_Riemann_surfaces}$), but in fact are; see Example \ref{RS:exa:biholomorphisms_between_Riemann_spheres} for a proof. In fact, \textit{any} Riemann surface that is topologically the sphere is the Riemann sphere, which we prove in Theorem \ref{MS:thm:simply-connect_compact_biholomorphic_Riemann_sphere}.
    \begin{example}[One-point Compactification of $\C$]\label{RS:exa:one_point_compactification_of_C}
        Let $\infty$ be a symbol not belonging to $\C$ and set $\C_\infty\!\coloneqq\C\cup\l\{\infty\r\}$. We declare a set $U\subseteq\C_\infty\!$ to be open if either $U\subseteq\C$ is open or $U=K^c\cup\l\{\infty\r\}$ for some compact subset $K\subseteq\C$. This makes $\C_\infty\!$, equipped with the collection $\mc{T}$ of all such open sets, a second-countable Hausdorff space. Indeed, the fact that $\mc{T}$ is a topology on $\C_\infty\!$ follows from De Morgan's Laws and the Heine-Borel Theorem; it is Hausdorff since any $p\in\C$ can be separated from $\infty$ by neighborhoods $B\l(p,r\r)$ and $\bar{B\l(p,r\r)}^c\cup\l\{\infty\r\}$, respectively; and it is second-countable since we may append, to any countable basis for the standard topology of $\C$, the countable collection $\big\{\bar{B\l(0,r\r)}^c\cup\l\{\infty\r\}\!\big\}_{r\in\Q^+}$. To give $\C_\infty$ a complex structure, we employ two charts
        \begin{equation*}
            \begin{alignedat}{2}
                U_1&\coloneqq\C_\infty\!\comp\l\{\infty\r\}=\C\ \ \ \ \ \ \ \ \ \ \ \ \ \ \ \ \ \ \ \ &&\phi_1:U_1\to\C:z\mapsto z\ \ \ \ \l(\phi_1\coloneqq\id_\C\r) \\
                U_2&\coloneqq\C_\infty\!\comp\l\{0\r\}=\C^\ast\cup\l\{\infty\r\}&&\phi_2:U_2\to\C:z\mapsto
                \begin{dcases}
                    1/z & \textrm{if }z\in\C^\ast \\
                    0 & \textrm{else.}
                \end{dcases}
            \end{alignedat}
        \end{equation*}
        Clearly $\phi_1$ is a homeomorphism. Since $\phi_2$ is invertible with $\phi_2^{-1}\!\l(z\r)\coloneqq1/z$ for all $z\in\C^\ast$ and $\phi_2^{-1}\!\l(0\r)\coloneqq\infty$, and
        \begin{equation*}
            \lim\limits_{z\to\infty}\phi_2\!\l(z\r)=0=\phi_2\!\l(\infty\r)\ \ \ \ \ \ \ \ \textrm{and}\ \ \ \ \ \ \ \ \lim\limits_{z\to0}\phi_2^{-1}\!\l(z\r)=\infty=\phi_2^{-1}\l(0\r),
        \end{equation*}
        we see that $\phi_2$ is a homeomorphism too. Furthermore, $\phi_2\circ\phi_1^{-1}:\C^\ast\to\C^\ast:z\mapsto1/z$ is holomorphic, so the atlas $\l\{\tpl{U_1,\phi_1},\tpl{U_2,\phi_2}\r\}$ defines a complex structure on $\C_\infty\!$.\exqed
    \end{example}
    \begin{example}[Stereographic Projection]\label{RS:exa:stereographic_projection}
        Consider the unit sphere $S^2\subseteq\R^3$ as a topological subspace of $\R^3$, which makes it a second-countable Hausdorff space. Letting $\tpl{x,y,w}$ be the standard coordinates of $\R^3$ and identifying the plane $w=0$ as $\C$, we employ the charts
        \begin{equation*}
            \begin{alignedat}{2}
                U_1&\coloneqq S^2\comp\l\{\l(0,0,1\r)\r\}\ \ \ \ \ \ \ \ \ \ \ \ \ \ \ \ &&\phi_1:U_1\to\C:\tpl{x,y,w}\mapsto\frac{x+iy}{1-w} \\
                U_2&\coloneqq S^2\comp\l\{\l(0,0,-1\r)\r\}&&\phi_2:U_2\to\C:\tpl{x,y,w}\mapsto\frac{x-iy}{1+w}.
            \end{alignedat}
        \end{equation*}
        Clearly $\phi_1$ and $\phi_2$ are continuous, and it can be verified that they are invertible with continuous inverses
        \begin{equation*}
            \phi_1^{-1}\!\l(z\r)\coloneqq\tpl{\frac{2\Re z}{\l|z\r|^2+1},\frac{2\Im z}{\l|z\r|^2+1},\frac{\l|z\r|^2-1}{\l|z\r|^2+1}}\ \ \ \ \textrm{and}\ \ \ \ \phi_2^{-1}\!\l(z\r)\coloneqq\tpl{\frac{2\Re z}{\l|z\r|^2+1},\frac{-2\Im z}{\l|z\r|^2+1},\frac{1-\l|z\r|^2}{\l|z\r|^2+1}}.
        \end{equation*}
        Observe that $U_1\cap U_2=S^2\comp\l\{\tpl{0,0,\pm1}\r\}$ and $\phi_2\circ\phi_1^{-1}:\C^\ast\to\C^\ast:z\mapsto1/z$, which is holomorphic, so the atlas $\l\{\tpl{U_1,\phi_1},\tpl{U_2,\phi_2}\r\}$ defines a complex structure on $\RS$.\exqed
    \end{example}
    \begin{example}[Complex Projective Line]\label{RS:exa:complex_projective_line}
        Consider the equivalence relation $\sim$ on $\C^2\comp\l\{\tpl{0,0}\r\}$ defined by $\tpl{z_1,w_1}\sim\tpl{z_2,w_2}$ iff $\tpl{z_1,w_1}=\lambda\tpl{z_2,w_2}$ for some $\lambda\in\C^\ast$. Set $\P^1\coloneqq\l(\C^2\comp\l\{\tpl{0,0}\r\}\r)/\!\sim$ and equip it with the quotient topology. Since $\sim$ is an open equivalence relation\footnote{See \cite[][Section 7.5]{tu} for details on the quotient topology and open equivalence relations.} whose graph is closed in $\big(\C^2\comp\l\{\tpl{0,0}\r\}\big)^2$, we see that $\P^1$ is a second-countable Hausdorff space. Denoting the equivalence class of $\tpl{z,w}$ by $\proje{z:w}$, we employ the charts
        \begin{equation*}
            \begin{alignedat}{2}
                U_1&\coloneqq\P^1\comp\l\{\proje{0:w}\mid w\in\C\r\}\ \ \ \ \ \ \ \ \ \ \ \ &&\phi_1:U_1\to\C:\proje{z:w}\mapsto w/z \\
                U_2&\coloneqq\P^1\comp\l\{\proje{z:0}\mid z\in\C\r\}&&\phi_2:U_2\to\C:\proje{z:w}\mapsto z/w.
            \end{alignedat}
        \end{equation*}
        Clearly $\phi_2$ and $\phi_2$ are continuous, and it is easily verified that they are invertible with continuous inverses
        \begin{equation*}
            \phi_1^{-1}\!\l(z\r)\coloneqq\proje{1:z}\ \ \ \ \ \ \ \ \textrm{and}\ \ \ \ \ \ \ \ \phi_2^{-1}\!\l(z\r)\coloneqq\proje{z:1}.
        \end{equation*}
        Furthermore, $\phi_2\circ\phi_1^{-1}:\C^\ast\to\C^\ast:z\mapsto1/z$ is holomorphic, so the atlas $\l\{\tpl{U_1,\phi_1},\tpl{U_2,\phi_2}\r\}$ defines a complex structure on $\P^1$.\exqed
    \end{example}
    \subsection{Complex Tori}
    Recall that a torus is any manifold homeomorphic to $T^2\coloneqq S^1\times S^1$, which admit representations as quotients $\C/\Gamma$ by lattices $\Gamma\coloneqq\Z\omega_1\oplus\Z\omega_2$ for any linearly independent vectors $\omega_1,\omega_2\in\C$ over $\R$. By definition, there is only one torus up to homeomorphism, but it turns out that we can equip it with many different complex structures. They arise from quotienting $\C$ by different lattices, and we shall derive a criterion on the lattices $\Gamma_1\coloneqq\Z\omega_1\oplus\Z\omega_2$ and $\Gamma_2\coloneqq\Z\eta_1\oplus\Z\eta_2$ for the tori $\C/\Gamma_1$ and $\C/\Gamma_2$ to be biholomorphic.
    \begin{example}[Complex Tori]\label{RS:exa:tori}
        Let $\omega_1,\omega_2\in\C$ be linearly independent over $\R$ and consider the lattice $\Gamma\coloneqq\Z\omega_1\oplus\Z\omega_2$. Identifying $S^1$ with the unit circle in $\C$, the quotient $\C/\Gamma$ is a torus in the topological sense since the map
        \begin{equation*}
            \phi:\C/\Gamma\to S^1\times S^1\ \ \ \ \ \ \ \ \textrm{mapping}\ \ \ \ \ \ \ \ \l[z\r]\mapsto\big(e^{2\pi i\lambda_1},e^{2\pi i\lambda_2}\big)
        \end{equation*}
        where $z=\lambda_1\omega_1+\lambda_2\omega_2$ for unique $\lambda_1,\lambda_2\in\R$, is a homeomorphism. Indeed, $\phi$ is well-defined since for any $\lambda_1\omega_1+\lambda_2\omega_2\sim\mu_1\omega_1+\mu_2\omega_2$ in $\C$, we have $\l(\lambda_1-\mu_1\r)\omega_1+\l(\lambda_2-\mu_2\r)\omega_2\in\Gamma$ and so $\lambda_i-\mu_i\in\Z$ for $i=1,2$. The fact that it is a homeomorphism is clear. This makes $\C/\Gamma$ a second-countable Hausdorff space, which we now endow with the following complex structure.\\\ \\
        Since $\Gamma$ is discrete, there exists some $\epsilon>0$ such that $\epsilon<\l|\omega\r|/2$ for every non-zero $\omega\in\Gamma$.\footnote{This exposition follows \cite[][Section I.2]{miranda}.} Fix any such $\epsilon$, which ensures that no two points in any open ball with radius $\epsilon$ can be equivalent. Indeed, take any $z\in\C$ and $w_1,w_2\in B\l(z,\epsilon\r)\eqqcolon V_z$. For $w_1\sim w_2$, we need some $n,m\in\Z$ such that $w_1-w_2=n\omega_1+m\omega_2$. But
        \begin{equation*}
            \l|w_1-w_2\r|\leq\l|z-w_1\r|+\l|z-w_2\r|<2\epsilon<\l|n\omega_1+m\omega_2\r|
        \end{equation*}
        for any $n,m\in\Z$, so this is impossible. Fixing any such $\epsilon$ gives us a family $\l\{V_z\r\}_{z\in\C}$ of open sets in $\C$ for which the projections $\l.\pi\r|_{V_z}:V_z\to\pi\l(V_z\r)$ are homeomorphisms. Letting $U_z\coloneqq\pi\l(V_z\r)$ and $\phi_z:U_z\to V_z$ be the inverse of $\l.\pi\r|_{V_z}$, we obtain complex charts $\tpl{U_z,\phi_z}$ for all $z\in\C$. We claim that the collection $\mf{A}\coloneqq\l\{\tpl{U_z,\phi_z}\r\}_{z\in\C}$ form an atlas, for which it suffices to take $\tpl{U_1,\phi_1},\tpl{U_2,\phi_2}\in\mf{A}$ and show that the transition map $T\coloneqq\phi_2\circ\phi_1^{-1}:\phi_1\!\l(U\r)\to\phi_2\!\l(U\r)$, where $U\coloneqq U_1\cap U_2$, is holomorphic. Observe that the diagram
        \begin{equation*}
            \begin{tikzcd}[column sep=0.1in]
                & U \ar[dl, "\phi_1"] \ar[dr, "\phi_2"'] \\
                V_1=\phi_1(U) \ar[ur, "\l.\pi\r|_{V_1}", bend left=20, start anchor={[xshift=-20px]}] \ar[rr, "T"'] & & \phi_2(U)=V_2 \ar[ul, "\l.\pi\r|_{V_2}"', bend right=20, start anchor={[xshift=20px]}]
            \end{tikzcd}
        \end{equation*}
        commutes, so $\l.\pi\r|_{V_2}\circ T=\l.\pi\r|_{V_1}$ on $\phi_1\!\l(U\r)$. Then $\pi\l(T\l(z\r)\r)=\pi\l(z\r)$ for every $z\in\phi_1\!\l(U\r)$, so $T\l(z\r)\sim z$ and hence $\ell\l(z\r)\coloneqq T\l(z\r)-z\in\Gamma$. This holds for all $z\in\phi_1\!\l(U\r)$, so we obtain a continuous function $\ell:\phi_1\!\l(U\r)\to\Gamma:z\mapsto T\l(z\r)-z$. Note that $\Gamma\subseteq\C$ is equipped with the subspace topology, but since it is discrete, every $L\subseteq\Gamma$ is open. In particular, fix $z_0\in\phi_1\!\l(U\r)$ and set $\gamma_0\coloneqq T\l(z_0\r)-z_0$. With $L\coloneqq\l\{\gamma_0\r\}$, continuity of $\ell$ shows that $\ell^{-1}\!\l(L\r)$ is open. Thus $\ell\l(B\l(z_0,\delta_1\r)\r)\subseteq\l\{\gamma_0\r\}$ for some $\delta_1>0$, so $\ell\l(w\r)=\gamma_0$ for all $w\in B\l(z_0,\delta_1\r)$. Thus $T\l(z\r)=z+\gamma_0$ for all $z$ in a local neighborhood around $z_0$, so $T$ is locally biholomorphic. Repeating this for all $z_0\in\phi_1\!\l(U\r)$, we see that $T$ is holomorphic on $\phi_1\!\l(U\r)$.\exqed
    \end{example}
\end{document}
