\documentclass[../Moduli_Spaces_of_Riemann_Surfaces.tex]{subfiles}
\begin{document}
    For any genus $g$, we let\side{This defines $\mc{M}_g$ as a set, but it turns out that they can all be equipped with a natural complex structure.} $\mc{M}_g$ denote the \textit{moduli space} of compact Riemann surfaces of genus $g$; that is, the set of all Riemann surfaces of genus $g$ up to biholomorphism. Using the language and machinery developed in Chapters \ref{cpt:Riemann_surfaces} and \ref{cpt:covering_spaces_analytic_continuation}, we compute $\mc{M}_0$ and $\mc{M}_1$, which are the moduli spaces of the sphere $S^2$ and the torus $T^2$, respectively.\\\ \\
    We conclude with a brief discussion of the \textit{Uniformization Theorem} and the \textit{Classification of Riemann Surfaces}.
    \section{Case for $g=0$ and $g=1$}
    \subsection{Moduli Space of $S^2$}\label{sec:moduli_space_of_sphere}
    We show that the moduli space of the sphere $S^2$ is a point. That is, there\side{It turns out that this is an easy corollary of the Riemann-Roch Theorem, but its proof is beyond the scope of this paper. Here, we present a more elementary proof.} is a \textit{unique} complex structure on the sphere.
    \begin{theorem}\label{3.1:thm:simply-connect_compact_biholomorphic_Riemann_sphere}
        Every simply-connected compact Riemann surface $X$ is biholomorphic to the Riemann sphere $\RS$.
    \end{theorem}
    \begin{proof}
        Fix an arbitrary point $p\in X$ and let $\tpl{U,\phi}$ be any chart of $X$ centered at $p$. Consider the function $f_0\in\HOLO\l[p\r]\l(U\r)$ defined by $f_0\!\l(q\r)\coloneqq1/\phi\l(q\r)$ for all $q\in U\comp\l\{p\r\}$. Applying Theorem \ref{2.3:thm:existence_of_analytic_continuation} to $\eta_0\coloneqq\l[f_0\r]_p$ and using Corollary \ref{2.3:thm:analytically_continue_to_global_function}, we obtain a function $f\in\HOLO\l[p\r]\l(X\r)$ such that $\l.f\r|_U=f_0$. Since $\ord_p\!\l(f_0\r)=1$, we see that $f$ is meromorphic on $X$ with a single simple pole, so $X\iso\RS$ by Corollary \ref{2.1:cor:exists_meromorphic_implis_biholomorphic_Riemann_sphere}.\qed
    \end{proof}
    \begin{remark}
        This is part of the \textit{Uniformization Theorem}, which states that every simply-connected Riemann surface is biholomorphic to either the Riemann sphere $\RS$, the complex plane $\C$, or the upper-half plane $\H$ of $\C$. A brief discussion and proof sketch is given in Section \ref{sec:uniformization}.\exqed
    \end{remark}
    \subsection{Moduli Space of $T^2$}
    We show that the moduli space of the torus $T^2$ is $\H/\PSL{2}{\Z}$ where $\H$ is the upper-half plane of $\C$ and $\PSL{2}{\Z}$ is the \textit{modular group} consisting of all functions $\gamma:\H\to\H$ mapping
    \begin{equation*}
        \tau\mapsto\frac{a\tau+b}{c\tau+d}
    \end{equation*}
    for some $a,b,c,d\in\Z$ with $ad-bc=1$.
    \begin{lemma}\label{2.1:lem:moduli_space_torus_1}
        Let $\Gamma,\Gamma'\subseteq\C$ be two lattices and suppose $\alpha\Gamma\subseteq\Gamma'$ for some $\alpha\in\C^\ast$. Then $z\mapsto\alpha z$ descends to a holomorphic map $\phi:\C/\Gamma\to\C/\Gamma'$, which is biholomorphic iff $\alpha\Gamma\subseteq\Gamma'$.
    \end{lemma}\side[-0.3in]{This gives a simple criterion for when two tori are biholomorphic.}\vspace{-0.08in}
    \begin{proof}
        Let $\Gamma\coloneqq\Z\omega_1\oplus\Z\omega_2$ and $\Gamma'\coloneqq\Z\omega_1'\oplus\Z\omega_2'$. Define $\phi\l(z+\Gamma\r)\coloneqq\alpha z+\Gamma'$ for all $z\in\C$, which is clearly holomorphic if it is well-defined in the first place. To verify, take $z_1,z_2\in\C$ such that $z_1+\Gamma=z_2+\Gamma$. Then $z_1-z_2\in\Gamma$, so $z_1-z_2=m\omega_1+n\omega_2$ for some $n,m\in\Z$. Observe that
        \begin{equation*}
            \alpha z_1-\alpha z_2=\alpha\l(z_1-z_2\r)=m\l(\alpha\omega_1\r)+n\l(\alpha\omega_2\r)\in\alpha\Gamma\subseteq\Gamma',
        \end{equation*}
        so $\alpha z_1+\Gamma'=\alpha z_2+\Gamma'$. This shows that $\phi$ is well-defined. Furthermore, it is invertible with holomorphic inverse
        \begin{equation*}
            \phi^{-1}\!\l(z+\Gamma'\r)\coloneqq z/\alpha+\Gamma
        \end{equation*}
        iff $\phi^{-1}$ is well-defined, in which case $\phi$ is a biholomorphism. We claim that this occurs iff $\alpha\Gamma=\Gamma'$.
        \begin{itemize}
            \item ($\Rightarrow$): It suffices to show that $\Gamma'\subseteq\alpha\Gamma$, so take $m\omega_1'+n\omega_2'\in\Gamma'$. Then
                \begin{equation*}
                    \phi^{-1}\!\l(m\omega_1'+n\omega_2'+\Gamma'\r)=\l(m\omega_1'+n\omega_2'\r)\!/\alpha+\Gamma,
                \end{equation*}
                but since $m\omega_1'+n\omega_2'+\Gamma'=0+\Gamma'$ and $\phi^{-1}\!\l(0+\Gamma'\r)=0+\Gamma$, we see that $\l(m\omega_1'+n\omega_2'\r)\!/\alpha\in\Gamma$.
            \item ($\Leftarrow$): Take $z_1,z_2\in\C$ such that $z_1+\Gamma'=z_2+\Gamma'$, so $z_1-z_2\in\Gamma'\subseteq\alpha\Gamma$ and hence
                \begin{equation*}
                    z_1/\alpha-z_2/\alpha=\l(z_1-z_2\r)\!/\alpha\in\Gamma.
                \end{equation*}
                Then $z_1/\alpha+\Gamma=z_2/\alpha+\Gamma$, so $\phi^{-1}$ is well-defined.\qed
        \end{itemize}
    \end{proof}
    \begin{lemma}
        Any torus $\C/\Gamma$ is biholomorphic to $X_\tau\!\coloneqq\C/\l(\Z+\tau\Z\r)$ for some $\tau\in\H$.
    \end{lemma}\side[-0.15in]{This reduces the analysis to just tori of the form $X_\tau$, which is considerably more simpler.}\vspace{-0.08in}
    \begin{proof}
        Let $\Gamma\coloneqq\Z\omega_1\oplus\Z\omega_2$ and set $\alpha\coloneqq1/\omega_1$ and $\tau\coloneqq\omega_2/\omega_1$. Then $\Im\tau\neq0$, lest $\omega_1,\omega_2$ be linearly dependent over $\R$. Without loss of generality, suppose that $\Im\tau>0$; if not, take $\tau\coloneqq\bar{\omega_2}/\omega_1$. Then, since
        \begin{equation*}
            \alpha\l(m\omega_1+n\omega_2\r)=\alpha\omega_1\l(m+n\omega_2/\omega_1\r)=m+n\tau
        \end{equation*}
        for all $m,n\in\Z$, we see that $\alpha\Gamma=\Z\oplus\Z\tau$. By Lemma \ref{2.1:lem:moduli_space_torus_1}, the map $z\mapsto\alpha z$ descends to a biholomorphism $\phi:\C/\Gamma\to\C/\l(\Z\oplus\Z\tau\r)=X_\tau$, so $\C/\Gamma\iso X_\tau$.\qed
    \end{proof}
    \begin{theorem}
        For any $\tau,\tau'\in\H$, the tori $X_{\tau}$ and $X_{\tau'}$ are biholomorphic iff there exists some $\gamma\in\PSL{2}{\Z}$ such that $\tau'=\gamma\l(\tau\r)$.
    \end{theorem}
    \begin{corollary}
        The moduli space of $T^2$ is $\H/\PSL{2}{\Z}$.
    \end{corollary}
    \begin{proof}
        The backwards direction is relatively straightforward. Indeed, note that
        \begin{equation*}
            \tau'=\frac{a\tau+b}{c\tau+d}\ \ \ \ \ \ \ \ \Rightarrow\ \ \ \ \ \ \ \ \tau=\frac{b-d\tau'}{c\tau'-a}
        \end{equation*}
        for any $a,b,c,d\in\Z$ with $ad-bc=1$, so let $\alpha\coloneqq c\tau'-a$. Then, with $\Gamma\coloneqq\Z\oplus\Z\tau$ and $\Gamma'\coloneqq\Z\oplus\Z\tau'$, we proceed by proving that $\alpha\Gamma=\Gamma'$, from which the result follows from Lemma \ref{2.1:lem:moduli_space_torus_1}.
        \begin{itemize}
            \item ($\subseteq$): For any $m,n\in\Z$, our choice of $\alpha$ shows that
                \begin{equation*}
                    m\alpha+n\alpha\tau=m\l(c\tau'-a\r)+n\l(b-d\tau'\r)=\l(nb-ma\r)+\l(mc-nd\r)\tau'\in\Z\oplus\Z\tau',
                \end{equation*}
                so $\alpha\l(\Z\oplus\Z\tau\r)\subseteq\Z\oplus\Z\tau'$.
            \item ($\supseteq$): For any $m,n\in\Z$, the condition that $ad-bc=1$ shows that
                \begin{equation*}
                    \l(m+n\tau'\r)\!/\alpha=\frac{\l(na-mc\r)\tau+\l(nb-md\r)}{a\l(c\tau+d\r)-c\l(a\tau+b\r)}=\l(nb-md\r)+\l(na-mc\r)\tau\in\Z\oplus\Z\tau,
                \end{equation*}
                so $\Z\oplus\Z\tau'\subseteq\alpha\l(\Z\oplus\Z\tau\r)$.
        \end{itemize}
        For the forward direction, let $\phi:X_\tau\to X_{\tau'}$ be a biholomorphism, which lifts to a biholomorphic mapping $\tilde{\phi}:\C\to\C$ such that
        \begin{equation*}
            \begin{tikzcd}
                \C \ar[r, "\tilde{\phi}"] \ar[d, "\pi"'] & \C \ar[d, "\pi'"] \\
                \C/\Gamma \ar[r, "\phi"] & \C/\Gamma'
            \end{tikzcd}           
        \end{equation*}
        commutes. Fix\side{This proof follows \cite[][Proposition 1.3.2]{diamond}. For an alternative proof, see \cite[][Lemma 2.8]{i&t}} $\lambda\in\Gamma$ and consider the map $f_\lambda\!\l(z\r)\coloneqq\tilde{\phi}\l(z+\lambda\r)-\tilde{\phi}\l(z\r)$. Then, since $z+\lambda+\Gamma=z+\Gamma$, we see that $\phi\l(z+\lambda+\Gamma\r)=\phi\l(z+\Gamma\r)$ and hence the commutativity of the diagram forces $\tilde{\phi}\l(z+\lambda\r)+\Gamma'=\tilde{\phi}\l(z\r)+\Gamma'$. Thus $f_\lambda\!\l(z\r)\in\Gamma'$ for all $z\in\C$, so, since $f_\lambda$ is a continuous map into a discrete set, it must be constant. Differentiating gives us $f_\lambda'\l(z\r)=\tilde{\phi}'\l(z+\lambda\r)-\tilde{\phi}'\l(z\r)=0$, so $\tilde{\phi}'\l(z+\lambda\r)=\tilde{\phi}'\l(z\r)$ for all $z\in\C$. But $\lambda\in\Gamma$ is arbitrary, so $\tilde{\phi}'$ is $\Gamma$-periodic. Thus $\tilde{\phi}'$ is a bounded entire function and hence is constant by Liouville's Theorem. This shows that $\tilde{\phi}\l(z\r)=\alpha z+\beta$ for some $\alpha,\beta\in\C$ with $\alpha\neq0$. Without loss of generality, assume that $\beta=0$. We now claim that $\alpha\Gamma=\Gamma'$.
        \begin{itemize}
            \item Indeed, for all $z\in\alpha\Gamma$, we have $z/\alpha\in\Gamma$ and so $z/\alpha+\Gamma=0+\Gamma$. Applying $\phi$ to both sides and comparing gives
                \begin{equation*}
                    0+\Gamma'=\phi\l(0+\Gamma\r)=\phi\l(z/\alpha+\Gamma\r)=\tilde{\phi}\l(z/\alpha\r)+\Gamma'=z+\Gamma',
                \end{equation*}
                so $z\in\Gamma'$. The converse is similar.
        \end{itemize}
        Observe then that $\tilde{\phi}\l(\tau\r)=\alpha\tau=b-d\tau'$ and $\tilde{\phi}\l(1\r)=\alpha=c\tau'-a$ for some $a,b,c,d\in\Z$, so
        \begin{equation*}
            \tau=\frac{b-d\tau'}{c\tau'-a}\ \ \ \ \ \ \ \ \textrm{and hence}\ \ \ \ \ \ \ \ \tau'=\frac{a\tau+b}{c\tau+d}.
        \end{equation*}
        A computation now shows that $\alpha=-\l(ad-bc\r)/\l(c\tau+d\r)$, so $ad-bc\neq0$. Then, since
        \begin{equation*}
            \begin{bmatrix}
                \alpha\tau \\ \alpha
            \end{bmatrix}=
            \begin{bmatrix}
                b & -d \\
                -a & c
            \end{bmatrix}
            \begin{bmatrix}
                1 \\ \tau'
            \end{bmatrix},
        \end{equation*}
        we solve for $\tau'$ to obtain
        \begin{equation*}
            \tau'=-\frac{b\alpha+a\alpha\tau}{ad-bc}=\l(\frac{-b}{ad-bc}\r)\alpha+\l(\frac{-a}{ad-bc}\r)\alpha\tau
        \end{equation*}
        But $\tau'\in\alpha\Gamma$, which forces $ad-bc=\pm1$. A little algebra now shows that\side{Let $\tau\coloneqq e+fi$ and $\tau'\coloneqq g+hi$ and expand.}
        \begin{equation*}
            \Im\tau'=\frac{ad-bc}{\l|c\tau+d\r|^2}\l(\Im\tau\r)>0,
        \end{equation*}
        so $ad-bc=1$.\qed
    \end{proof}
\end{document}
